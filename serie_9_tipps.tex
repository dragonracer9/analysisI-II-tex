\documentclass[12pt]{article}

\usepackage[margin=1in]{geometry}
\usepackage[utf8]{inputenc}
\usepackage[ngerman]{babel}
\usepackage{parskip}

\usepackage{amsmath}
\usepackage{amssymb}
\usepackage{amsfonts}
\usepackage{enumitem}
\usepackage{mathtools}
\usepackage{mathrsfs}
\usepackage{setspace}
\usepackage{xargs}
\usepackage{xcolor}
\usepackage{witharrows}
\usepackage{commath}
\usepackage{physics}
\usepackage{tikz}
\usetikzlibrary{patterns,hobby}
\usepackage{pgfplots}
\pgfplotsset{compat=1.6}
\usepackage[hidelinks]{hyperref}
\usepackage{graphicx}
\usepackage{float}
\usepackage[centerlast,small,sc]{caption}
\usepackage{subcaption} % causes weird error with \setlength{\captionmargin}{20pt}

\newcommand{\dx}{\mathrm{d}x}
\newcommand{\ddx}{\frac{\mathrm{d}}{\mathrm{d}x}}
\newcommand{\dt}{\mathrm{d}t}
\newcommand{\du}{\mathrm{d}u}
%\newcommand{\dv}{\mathrm{d}v}
\newcommand{\dy}{\mathrm{d}y}
\newcommand{\dz}{\mathrm{d}z}

\newcommand{\Rn}{\mathbb{R}^n}
\newcommand{\Rm}{\mathbb{R}^m}
\newcommand{\Rk}{\mathbb{R}^k}
\newcommand{\und}{\text{ und }}
\newcommand{\oder}{\text{ oder }}
\newcommand{\bydef}{\underset{def.}{=}}
\newcommand{\BH}{\underset{\textrm{B-H}}{=}}

\newcommand{\Follows}{\Longrightarrow\ }
\newcommand{\sameas}{\Longleftrightarrow}
\newcommandx{\Laplace}[2][1=f(t), 2=s]{\mathscr{L}\{#1\}(#2)}
\newcommandx{\LaplaceInv}[2][1=F(s), 2=t]{\mathscr{L}^{-1}\{#1\}(#2)}
\DeclareMathOperator{\arccosh}{Arcosh}
\DeclareMathOperator{\arcsinh}{Arsinh}
\DeclareMathOperator{\arctanh}{Artanh}
\DeclareMathOperator{\arcsech}{arcsech}
\DeclareMathOperator{\arccsch}{arcCsch}
\DeclareMathOperator{\arccoth}{arcCoth} 

\def\doubleunderline#1{\underline{\underline{#1}}}
\def\ez{\begin{flushright}\underline{ez.}\end{flushright}}
%\[
%   \Laplace[\cos(x)]=\int_{t=0}^{\infty}f(t)e^{-st}dt
%\]

\newcommand{\R}{\mathbb{R}} % gives blackboard R
\newcommand{\Z}{\mathbb{Z}}
\newcommand{\N}{\mathbb{N}}
\newcommand{\Q}{\mathbb{Q}}
\newcommand{\C}{\mathbb{C}}

\newcommand{\inttext}{\shortintertext}

\newenvironment{definition}[2][Definition]{\begin{trivlist}
        \item[\hskip \labelsep {\bfseries #1}\hskip \labelsep {\bfseries #2.}]}{\flushright{$\square$}\end{trivlist}}
\newenvironment{lemma}[2][Theorem]{\begin{trivlist}
        \item[\hskip \labelsep {\bfseries #1}\hskip \labelsep {\bfseries #2.}]}{\flushright{$\square$}\end{trivlist}}
\newenvironment{exercise}[2][Exercise]{\begin{trivlist}
        \item[\hskip \labelsep {\bfseries #1}\hskip \labelsep {\bfseries #2.}]}{\end{trivlist}}
\newenvironment{problem}[2][\textcolor{blue}{Tipps \& Tricks zu}]{\begin{trivlist}
        \item[\hskip \labelsep {\bfseries #1}\hskip \labelsep {\bfseries \textcolor{blue}{#2}.}]}{\end{trivlist}}
\newenvironment{question}[2][\textcolor{red}{Aufgabe}]{\begin{trivlist}
        \item[\hskip \labelsep {\bfseries \textcolor{red}{#1}}\hskip \labelsep {\bfseries \textcolor{red}{#2}.}]}{\end{trivlist}}
\newenvironment{remark}[2][Bemerkung]{\begin{trivlist}
        \item[\hskip \labelsep {\bfseries #1}\hskip \labelsep {\bfseries #2.}]}{\end{trivlist}}

\begin{document}
\title{Problem Set 8, Tips}
\author{Vikram Damani\\
        Analysis I}

\maketitle
Aufgaben in \textcolor{red}{rot} markiert, Tipps \& Tricks in \textcolor{blue}{blau}.

\section{Theorie}

\begin{definition}{[Zykloide]}
    Die Zykloide ist die Kurve, die ein Punkt auf einem Kreis beschreibt, der auf einer anderen Kurve abrollt.

    Die Kurve ist die Summe aus zwei Bewegungen: Die Bewegung des Mittelpunks des Kreises und die Bewegung des Punktes auf dem Kreis.

    \begin{figure}[htbp!]
        \centering
        \begin{tikzpicture}
            \coordinate (O) at (0,0);
            \coordinate (A) at (0,3);
            \def\r{1} % radius
            \def\c{1.4} % center
            \coordinate (C) at (\c, \r);
          
          
            \draw[-latex] (O) -- (A) node[anchor=south] {$y$};
            \draw[-latex] (O) -- (2.6*pi,0) node[anchor=west] {$x$};
            \draw[red,domain=-0.5*pi:2.5*pi,samples=50, line width=1] 
                 plot ({\x - sin(\x r)},{1 - cos(\x r)});
            \draw[blue, line width=1] (C) circle (\r);
            \draw[] (C) circle (\r);
          
            % coordinate x 
            \def\x{0.4} % coordinate x
            \def\y{0.83} % coordinate y
            \def\xa{0.3} % coordinate x for arc left
            \def\ya{1.2} % coordinate y for arc left
            \coordinate (X) at (\x, 0 );
            \coordinate (Y) at (0, \y );
            \coordinate (XY) at (\x, \y );
          
            \node[anchor=north] at (X) {$x$} ;
          
            % draw center of circle
            \draw[fill=blue] (C) circle (1pt);
          
            % draw radius of the circle
            \draw[] (C) -- node[anchor=south] {\; $a$} (XY);
          
            % bottom of circle, radius to the bottom
            \coordinate (B) at (\c, 0);
            \draw[] (C) -- (B) node[anchor=north] {$a \, t$};
          
            % projections of point XY
            \draw[dotted] (XY) -- (X);
            \draw[dotted] (XY) -- (Y) node[anchor=east, xshift=1mm] {$\quad y$};
          
            % arc theta
            % start arc
            \coordinate (S) at (\c, 0.4);
            \draw[->] (S) arc (-90:-165:0.6);
            \node[xshift=-2mm, yshift=-2mm] at (C) {\scriptsize $t$};
          
            % arc above
            \coordinate (AA) at (\xa, \ya);
            \draw[-latex, rotate=25] (AA) arc (-220:-260:1.3);
          
            % arc below
            \def\xb{2.5} % coordinate x for arc bottom
            \def\yb{0.8} % coordinate y for arc bottom
            \coordinate (AB) at (\xb, \yb);
            \draw[-latex, rotate=-10] (AB) arc (-5:-45:1.3);
          
          
          
            % XY dot
            \draw[fill=black] (XY) circle (1pt);
          
          
            % top label
            \coordinate (T) at (pi, 2);
            \node[anchor=south] at (T)  {$(\pi a, 2 a )$} ;
            \draw[fill=black] (T) circle (1pt);
          
            % equations
            \coordinate (E) at ( 4,1.2);
            \coordinate (F) at ( 4,0.9);
            \node[] at (E) {\scriptsize $x=a(t - \sin t)$};
            \node[] at (F) {\scriptsize $y=a(1 - \cos t)$};
          
            % label 2pi a
            \coordinate (TPA) at (2*pi, 0);
            \node[anchor=north] at (TPA) {$2 \pi a$};
          
          
            \end{tikzpicture}
        \caption{Zykloide}
    \end{figure}
%% CITATION: https://tex.stackexchange.com/a/275450


    Wir betrachten einen Kreis mit Radius $a$. Der Kreismittelpunkt bewegt sich hier auf einer Geraden: 
    \begin{align*}
        x(t) &= at\\
        y(t) &= a
    \end{align*}
    Der Punkt auf dem Kreis bewegt sich dann auf der Zykloide:
    \begin{align*}
        x(t) &= a(t - \sin t)\\
        y(t) &= a(1 - \cos t)\\\\
        \sameas \begin{pmatrix}
            x(t)\\
            y(t)
        \end{pmatrix}
        &= \begin{pmatrix}
            at\\
            a
        \end{pmatrix}
        - a\begin{pmatrix}
            \sin t\\
            \cos t
        \end{pmatrix}\\
        &= \vec{m}(t)-a\begin{pmatrix}
            \sin t\\
            \cos t
        \end{pmatrix}
    \end{align*}
    Hier ist $\vec{m}(t)$ der Vektor, der die Bewegung des Kreismittelpunkts beschreibt.

    Allgemeiner gilt für einen Kreis mit Radius $a$, wenn der Punkt nicht auf dem Kreis liegt, sondern auf einem Abstand $b$ vom Kreismittelpunkt:
    \begin{align*}
        x(t) &= at - b\sin(t) \\
        y(t) &= a - b\cos(t) 
    \end{align*}
    \begin{align*}
        \text{Falls }&b<a: \text{ Verkürzte Zykloide}\\
        \text{Falls }&b>a: \text{ Verlängerte Zykloide}
    \end{align*}

    \begin{remark}{[Mehr als Geraden]}
            Wie vorher erwähnt, ist die Zykloide die Kurve die durch die Bewegung eines Punktes auf einem Kreis entsteht, der auf einer anderen Kurve abrollt. Hieberi muss die andere Kurve nicht unbedingt eine Gerade sein, sondern kann auch eine beliebige Kurve sein. Dann ist $\vec{m}(t)$ (die Bewegung des Kreismittelpunkts) gegeben durch die Kurve auf der der Kreis abrollt.
    \end{remark}
\end{definition}

\begin{definition}{[Tangenten]}
    \begin{figure}[htbp!]
        \centering
        \begin{tikzpicture}
            \begin{axis}[
                xmin=-1,xmax=5,
                ymin=-1,ymax=4.5,
            xtick={3,4},
            xticklabels={$x_1$,$x_2$},
            ytick={4,3.5},
            yticklabels={$y_1$,$y_2$},
            xlabel={$x$},  
            ylabel={$y$},
            axis lines=middle]
            \draw [thick] (axis cs:-1,2) to [ curve through ={(axis cs:1,4)  . . (axis cs:3,4) . . (axis cs:4,3.5)  }] (axis cs:5.5,3.3);
            \draw [thick, color=red,->] (axis cs:3,4) -- (axis cs:4,3.5) node[anchor=south west] {\scriptsize $\vec{v_2}-\vec{v_1}$};
            \draw[dashed] (axis cs:3,4) -- (axis cs:3,0) ;
            \draw[dashed] (axis cs:4,3.5) -- (axis cs:4,0) ;
            \draw[dashed] (axis cs:3,4) -- (axis cs:0,4) ;
            \draw[dashed] (axis cs:4,3.5) -- (axis cs:0,3.5);
            \draw[thick,color=blue,->] (axis cs:0,0) -- (axis cs:3,4) node[anchor=south east] {\scriptsize $\vec{v_1}$};
            \draw[thick,color=blue,->] (axis cs:0,0) -- (axis cs:4,3.5) node[anchor=north west] {\scriptsize $\vec{v_2}$};
        \end{axis}
        \end{tikzpicture}
    \end{figure}
    Die Richtung der Tangente an eine ebene Kurve $(x(t), y(t))$ an der Stelle $t_0$ ist gegeben durch:
    \begin{align*}
        \begin{pmatrix}
            \dot{x}(t_0)\\
            \dot{y}(t_0)
        \end{pmatrix}
    \end{align*}
\end{definition}

\end{document}