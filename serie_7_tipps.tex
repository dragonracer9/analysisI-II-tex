\documentclass[12pt]{article}

\usepackage[margin=1in]{geometry}
\usepackage[utf8]{inputenc}
\usepackage{parskip}

\usepackage{amsmath}
\usepackage{amssymb}
\usepackage{amsfonts}
\usepackage{enumitem}
\usepackage{mathtools}
\usepackage{mathrsfs}
\usepackage{setspace}
\usepackage{xargs}
\usepackage{xcolor}
\usepackage{witharrows}
\usepackage{commath}
\usepackage{physics}
\usepackage[hidelinks]{hyperref}

\newcommand{\dx}{\mathrm{d}x}
\newcommand{\ddx}{\frac{\mathrm{d}}{\mathrm{d}x}}
\newcommand{\dt}{\mathrm{d}t}
\newcommand{\du}{\mathrm{d}u}
%\newcommand{\dv}{\mathrm{d}v}
\newcommand{\dy}{\mathrm{d}y}
\newcommand{\dz}{\mathrm{d}z}

\newcommand{\Rn}{\mathbb{R}^n}
\newcommand{\Rm}{\mathbb{R}^m}
\newcommand{\Rk}{\mathbb{R}^k}
\newcommand{\und}{\text{ und }}
\newcommand{\oder}{\text{ oder }}
\newcommand{\bydef}{\underset{def.}{=}}
\newcommand{\BH}{\underset{\textrm{B-H}}{=}}

\newcommand{\Follows}{\Longrightarrow\ }
\newcommand{\sameas}{\Longleftrightarrow}
\newcommandx{\Laplace}[2][1=f(t), 2=s]{\mathscr{L}\{#1\}(#2)}
\newcommandx{\LaplaceInv}[2][1=F(s), 2=t]{\mathscr{L}^{-1}\{#1\}(#2)}
\DeclareMathOperator{\arccosh}{Arcosh}
\DeclareMathOperator{\arcsinh}{Arsinh}
\DeclareMathOperator{\arctanh}{Artanh}
\DeclareMathOperator{\arcsech}{arcsech}
\DeclareMathOperator{\arccsch}{arcCsch}
\DeclareMathOperator{\arccoth}{arcCoth} 

\def\doubleunderline#1{\underline{\underline{#1}}}
\def\ez{\begin{flushright}\underline{ez.}\end{flushright}}
%\[
%   \Laplace[\cos(x)]=\int_{t=0}^{\infty}f(t)e^{-st}dt
%\]

\newcommand{\R}{\mathbb{R}} % gives blackboard R
\newcommand{\Z}{\mathbb{Z}}
\newcommand{\N}{\mathbb{N}}
\newcommand{\Q}{\mathbb{Q}}
\newcommand{\C}{\mathbb{C}}

\newcommand{\inttext}{\shortintertext}

\newenvironment{definition}[2][Definition]{\begin{trivlist}
        \item[\hskip \labelsep {\bfseries #1}\hskip \labelsep {\bfseries #2.}]}{\flushright{$\square$}\end{trivlist}}
\newenvironment{lemma}[2][Theorem]{\begin{trivlist}
        \item[\hskip \labelsep {\bfseries #1}\hskip \labelsep {\bfseries #2.}]}{\flushright{$\square$}\end{trivlist}}
\newenvironment{exercise}[2][Exercise]{\begin{trivlist}
        \item[\hskip \labelsep {\bfseries #1}\hskip \labelsep {\bfseries #2.}]}{\end{trivlist}}
\newenvironment{problem}[2][\textcolor{blue}{Tipps \& Tricks zu}]{\begin{trivlist}
        \item[\hskip \labelsep {\bfseries #1}\hskip \labelsep {\bfseries \textcolor{blue}{#2}.}]}{\end{trivlist}}
\newenvironment{question}[2][\textcolor{red}{Aufgabe}]{\begin{trivlist}
        \item[\hskip \labelsep {\bfseries \textcolor{red}{#1}}\hskip \labelsep {\bfseries \textcolor{red}{#2}.}]}{\end{trivlist}}
\newenvironment{remark}[2][Bemerkung]{\begin{trivlist}
        \item[\hskip \labelsep {\bfseries #1}\hskip \labelsep {\bfseries #2.}]}{\end{trivlist}}

\begin{document}
\title{Problem Set 7, Tips}
\author{Vikram Damani\\
        Analysis I}

\maketitle
Aufgaben in \textcolor{red}{rot} markiert, Tipps \& Tricks in \textcolor{blue}{blau}.

\section{Theorie}

\begin{definition}{[Fundamentalsatz der Algebra]}
        Sei $p(x)=a_n x^n+a_{n-1}x^{n-1}+ \cdots +a_1x+a_0$ ein Polynom mit Koeffizienten $a_0,a_1,\ldots,a_n\in\C$. Eine Zahl $x_0$ heißt Nullstelle von $p(x)$, falls $p(x_0)=0$.

        Jedes Polynom $p(x)$ vom Grad $n\geq 1$ hat genau $n$ Nullstellen, gezählt mit
        Vielfachheit. Das Polynom $p(x)$ lässt sich also schreiben als
        \begin{equation}
                p(x)=a_n(x-x_1)(x-x_2)\ldots(x-x_n)
        \end{equation}
        wobei $x_1,x_2,\ldots,x_n$ die Nullstellen von $p(x)$ sind.

        \textbf{Bemerkung:} Die Nullstellen von einem Polynom $p(x)$ mit reellen Koeffizienten $a_k\in\R$ sind nicht notwendigerweise reell. Es gilt jedoch, dass komplexe Nullstellen stets als komplex konjugierte Paare auftreten, d.h. wenn $x_0\in\C$ eine Nullstelle von $p(x)$ ist, dann ist auch $\overline{x_0}$ eine Nullstelle von $p(x)$.
\end{definition}

\begin{remark}{[$\arcsinh$]}
        Die Funktion $\arcsinh(x)$ ist die Umkehrfunktion von $\sinh(x)$, d.h. $\sinh(\arcsinh(x))=x$.
        Es gilt aus der Vorlesung, dass $\arcsinh(x) = \ln(x + \sqrt{x^2 + 1})$.
\end{remark}

\begin{definition}{[Exponentialfunktion]}
        Die Exponentialfunktion
        \begin{equation}
                \exp: \R\to\left(0,\infty\right),\,x\mapsto\exp(x)
        \end{equation} ist definiert als
        \begin{equation}
                \exp(x)=\sum_{n=0}^{\infty}\frac{x^n}{n!}
        \end{equation}
        und erfüllt die Funktionalgleichung
        \begin{equation}
                \exp(x+y)=\exp(x)\exp(y)
        \end{equation}
        für alle $x,y\in\R$. Die Ableitung der Exponentialfunktion ist die Exponentialfunktion selbst, d.h.
        \begin{equation}
                \ddx{f}(x)=f(x).
        \end{equation}
        Die Exponentialfunktion ist stetig, streng monoton wachsend und bijektiv. Es gilt $\exp(0)=1$ und $\exp(1)=e$.
\end{definition}

\begin{definition}{[Logarithmus]}
        Der natürliche Logarithmus ist die Umkehrfunktion der Exponentialfunktion, d.h. $\exp(\ln(x))=x$.
        Der natürliche Logarithmus ist definiert als
        \begin{equation}
                \ln: \left(0,\infty\right)\to\R,\,x\mapsto\ln(x)
        \end{equation}
        und erfüllt die Funktionalgleichung
        \begin{equation}
                \ln(xy)=\ln(x)+\ln(y)
        \end{equation}
        für alle $x,y\in\left(0,\infty\right)$. Der natürliche Logarithmus ist stetig, streng monoton wachsend und bijektiv. Es gilt $\ln(1)=0$ und $\ln(e)=1$.

        \begin{remark}{[Ableitung des Logarithmus]}
                Die Ableitung der Umkehrfunktion $f^{-1}$ ist gegeben durch
                \begin{equation}\label{eq:1}
                        \ddx{f^{-1}(x)}=\frac{1}{f'(f^{-1}(x))}.
                \end{equation}
                Daher ist die Ableitung des natürlichen Logarithmus
                \begin{equation}
                        \ddx{\ln(x)}\underset{(\ref{eq:1})}{=}\frac{1}{e^{\ln(x)}}\underset{def.}{=}\frac{1}{x}.
                \end{equation}
        \end{remark}

        \begin{remark}{[Logarithmus zur Basis $a$]}
                Da $a^x=e^{x\ln(a)}$ gilt, ist der Logarithmus zur Basis $a$ definiert als
                \begin{equation}
                        \log_a: \left(0,\infty\right)\to\R,\,x\mapsto\log_a(x)=\frac{\ln(x)}{\ln(a)}.
                \end{equation}
                Es gilt $\log_a(1)=0$ und $\log_a(a)=1$. Die Ableitung des Logarithmus zur Basis $a$ ist
                \begin{equation}
                        \ddx{\log_a(x)}=\frac{1}{x\ln(a)}.
                \end{equation}
        \end{remark}
\end{definition}

\begin{definition}{[Landau-o]}
        Seien $f,g: \R\to\R$ Funktionen. Dann schreiben wir $f(x)=o(g(x))$ für $x\to \infty$, falls
        \begin{equation}
                \lim_{x\to \infty}\frac{f(x)}{g(x)}=0.
        \end{equation}

        \begin{remark}{[Intuition]}
                Die Schreibweise $f(x)=o(g(x))$ bedeutet, dass $f(x)$ im Vergleich zu $g(x)$ für $x\to \infty$ asymptotisch vernachlässigbar klein ist.
        \end{remark}
        \begin{remark}{[Berechnung]}
                Um zu zeigen, dass $f(x)=o(g(x))$ für $x\to \infty$, ist es wichtig, gut mit Bernoulli-Hôpital umgehen zu können.
        \end{remark}
        \begin{remark}{[$\ln$, $x^k$ und $e^x$]}
                Es gilt $\ln(x)=o(x^k)$ für $x\to\infty$ für jedes $k>0$, da
                \begin{equation}
                        \lim_{x\to\infty}\frac{\ln(x)}{x^k}\BH{}\lim_{x\to\infty}\frac{1/x}{kx^{k-1}}=\lim_{x\to\infty}\frac{1}{kx^k}=0.
                \end{equation}
                Es gilt auch $x^k=o(e^x)$ für $x\to\infty$ für jedes $k>0$, da
                \begin{equation}
                        \lim_{x\to\infty}\frac{x^k}{e^x}\BH{}\lim_{x\to\infty}\frac{kx^{k-1}}{e^x}\BH\lim_{x\to\infty}\frac{k(k-1)x^{k-2}}{e^x}\BH\cdots\BH\lim_{x\to\infty}\frac{k!}{e^x}=0.
                \end{equation}
        \end{remark}
        \textbf{Beispiel:} Sei $f(x)=x^2$ und $g(x)=x^3$. Dann gilt $f(x)=o(g(x))$ für $x\to 0$, da
        \begin{equation}
                \lim_{x\to 0}\frac{x^2}{x^3}=\lim_{x\to 0}\frac{1}{x}=0.
        \end{equation}
\end{definition}

\begin{definition}{[Landau-O]}
        Seien $f,g: \R\to\R$ Funktionen. Dann schreiben wir $f(x)=O(g(x))$ für $x\to \infty$, falls es eine Konstante $A\in\R^{+}$ gibt, sodass
        \begin{equation}
                \lim_{x\to \infty}\abs{\frac{f(x)}{g(x)}}=A\neq0.
        \end{equation}

        \begin{remark}{[Intuition]}
                Die Schreibweise $f(x)=O(g(x))$ bedeutet, dass $f(x)$ eine asymptotische obere Schranke von $g(x)$ für $x\to \infty$ ist.\footnote{\url{https://de.wikipedia.org/wiki/Landau-Symbole}}
        \end{remark}
\end{definition}

\section{Tipps \& Tricks}

\begin{question}{1}
        Es sei das Gebiet
        \begin{equation}
                B=\left\{z\in\C\setminus\{0\}\mid\Im\left(\frac{z+2}{iz}\right)>0\right\}
        \end{equation}
        gegeben.
        \begin{enumerate}[label=(\alph*)]
                \item Skizzieren Sie das Gebiet $B$ in der komplexen Ebene.
                \item Das Polynom $z^3+\frac{7}{2}z^2+7z+6$ hat eine komplexe Nullstelle mit Realteil
                      gleich $-1$. Bestimmen Sie alle Nullstellen dieses Polynoms. Wie lauten die
                      Nullstellen in Polarform?
                \item Welche dieser Nullstellen befinden sich in $B$?
        \end{enumerate}
\end{question}

\begin{problem}{1} Wie letzte Woche, muss man hier eine Teilmenge der komplexen Zahlen skizzieren. Umformen der Ungleichung liefert eine Ungleichung in $x$ und $y$. Diese kann man mit quadratischer Ergänzung in eine schöne Form
        bringen.
\begin{enumerate}[label=(\alph*)]
        \item Einsetzen von $z=x+iy$ in die Ungleichung und Umformen ergibt eine Ungleichung in $x$ und $y$. Der Imaginärteil ist ein Bruch. Kann der Nenner negativ sein? Was beduetet das für die Ungleichung?
        \item Da eine komplexe Nullstelle mit Realteil $-1$ gegeben ist, ist die Nullstelle eine Zahl der Form $-1\pm{iy}$ (da komplexe Nullstellen stets als konjugierte Paare auftreten). Polynomdivision durch diese beiden Nullstellen liefert die letze Nullstelle und ein quadratisches Polynom als Rest. Wie bestimmt man nun $y$?
\end{enumerate}
\end{problem}

\begin{question}{2}
        Es bezeichne $p$ ein Polynom fünften Grades mit reellen Koeffizienten. Weiter bezeichne $r_p\geq 0$ bzw. $c_p\geq 0$ die mit Vielfachheit gezählten reellen bzw.\ komplexen, nicht-reellen Nullstellen von $p$.
        \begin{enumerate}[label=(\alph*)]
                \item Was sind $r_p$ und $c_p$ für das Polynom $p(x)=x^5+1$?
                \item Sei nun $p$ wieder ein allgemeines Polynom fünften Grades mit reellen
                      Koeffizienten. Zeigen Sie, dass immer gilt: $r_p+c_p=5$.
                \item Wieso ist $r_p=2$ und $c_p=3$ niemals möglich?
                \item Finden Sie alle möglichen Werte von $r_p$. Geben Sie weiterhin jeweils ein
                      Polynom $p$ an, welches genau $r_p$ mit Vielfachheit gezählte reelle
                      Nullstellen hat.
        \end{enumerate}
\end{question}

\begin{problem}{2} Von den Überlegungen her ist diese Aufgabe sehr ähnlich zu einer typischen Prüfungsaufgabe. Es lohnt sich daher, diese Aufgabe gut zu verstehen. Hier ist der Fundamentalsatz der Algebra sehr wichtig.        
\end{problem}

\begin{question}{3}
        Die hyperbolischen Funktionen sinh und cosh sind wie folgt definiert:
        \begin{equation}
                \sinh: \R\to\R,\,x\mapsto\frac{e^x-e^{-x}}{2},\quad\cosh: \R\to\R,\,x\mapsto\frac{e^x+e^{-x}}{2}.
        \end{equation}
        Beweisen Sie folgende Identitäten für alle $x\in\R$:
        \begin{enumerate}[label=(\alph*)]
                \item $\cosh^2(x)-\sinh^2(x)=1$
                \item $\sinh(x+y)=\sinh(x)\cosh(y)+\cosh(x)\sinh(y)$
        \end{enumerate}
        Beweisen Sie die folgenden Identitäten für alle $x\in\R$, an denen beide Seiten definiert sind:
        \begin{enumerate}[label=(\alph*)]
                \item $2\cosh^2\left(\frac{x}{2}\right)=\cosh(x)+1$
                \item $\arccosh(x)=\ln\left(x+\sqrt{x^2-1}\right)$
                \item $\arcsinh'(x)=\frac{1}{\sqrt{x^2+1}}$
                \item $\arccosh'(x)=\frac{1}{\sqrt{x^2-1}}$
        \end{enumerate}
\end{question}

\begin{problem}{3} Die Identitäten (a), (b) und (c) lassen sich durch direktes Einsetzen beweisen. (d) geht ganz analog zu der Herleitung von $\arcsinh(x)$ in der Vorlesung. Welchen Ast der Wurzelfunktion in $y\pm\sqrt{y^2-1}$ muss man hier wählen? (e) und (f) lassen sich durch die allgemeine Ableitungsregel für Umkehrfunktionen beweisen.
\end{problem}

\begin{question}{4}
        Ordnen Sie die folgenden sechs Funktionen nach ihren Grössenordnungen, wenn $x\to +\infty$ strebt.
        \begin{enumerate}[label=(\alph*)]
                \item $\ln\left(\ln(x^2)\right)$
                \item $\ln\left(e^x-x\right)$
                \item $x^2$
                \item $x^{1/5}$
                \item $\ln\left(10x^{1/2}\right)$
                \item $e^x$
        \end{enumerate}
\end{question}

\begin{problem}{4} Hier muss man gut mit Bernoulli-Hôpital umgehen können.
\end{problem}
\end{document}