\documentclass[12pt]{article}

\usepackage[margin=1in]{geometry}
\usepackage[utf8]{inputenc}
\usepackage{parskip}

\usepackage{amsmath}
\usepackage{amssymb}
\usepackage{amsfonts}
\usepackage{enumitem}
\usepackage{mathtools}
\usepackage{mathrsfs}
\usepackage{setspace}
\usepackage{xargs}
\usepackage{xcolor}
\usepackage{witharrows}

\newcommand{\Follows}{\Longrightarrow\ }
\newcommand{\sameas}{\Longleftrightarrow}
\newcommandx{\Laplace}[2][1=f(t), 2=s]{\mathscr{L}\{#1\}(#2)}
\DeclareMathOperator{\sech}{sech}
\DeclareMathOperator{\csch}{csch}
\DeclareMathOperator{\arcsec}{arcsec}
\DeclareMathOperator{\arccot}{arcCot}
\DeclareMathOperator{\arccsc}{arcCsc}
\DeclareMathOperator{\arccosh}{Arcosh}
\DeclareMathOperator{\arcsinh}{Arsinh}
\DeclareMathOperator{\arctanh}{Artanh}
\DeclareMathOperator{\arcsech}{arcsech}
\DeclareMathOperator{\arccsch}{arcCsch}
\DeclareMathOperator{\arccoth}{arcCoth} 

\def\doubleunderline#1{\underline{\underline{#1}}}
\def\ez{\begin{flushright}\underline{ez.}\end{flushright}}
%\[
%   \Laplace[\cos(x)]=\int_{t=0}^{\infty}f(t)e^{-st}dt
%\]

\newcommand{\R}{\mathbb{R}} % gives blackboard R
\newcommand{\Z}{\mathbb{Z}}
\newcommand{\N}{\mathbb{N}}
\newcommand{\Q}{\mathbb{Q}}
\newcommand{\C}{\mathbb{C}}

\newcommand{\inttext}{\shortintertext}

\newenvironment{definition}[2][Definition]{\begin{trivlist}
        \item[\hskip \labelsep {\bfseries #1}\hskip \labelsep {\bfseries #2.}]}{\flushright{$\square$}\end{trivlist}}
\newenvironment{lemma}[2][Theorem]{\begin{trivlist}
        \item[\hskip \labelsep {\bfseries #1}\hskip \labelsep {\bfseries #2.}]}{\flushright{$\square$}\end{trivlist}}
\newenvironment{exercise}[2][Exercise]{\begin{trivlist}
        \item[\hskip \labelsep {\bfseries #1}\hskip \labelsep {\bfseries #2.}]}{\end{trivlist}}
\newenvironment{problem}[2][\textcolor{blue}{Tipps \& Tricks zu}]{\begin{trivlist}
        \item[\hskip \labelsep {\bfseries #1}\hskip \labelsep {\bfseries \textcolor{blue}{#2}.}]}{\end{trivlist}}
\newenvironment{question}[2][\textcolor{red}{Aufgabe}]{\begin{trivlist}
        \item[\hskip \labelsep {\bfseries \textcolor{red}{#1}}\hskip \labelsep {\bfseries \textcolor{red}{#2}.}]}{\end{trivlist}}
\newenvironment{remark}[2][Bemerkung]{\begin{trivlist}
        \item[\hskip \labelsep {\bfseries #1}\hskip \labelsep {\bfseries #2.}]}{\end{trivlist}}

\begin{document}
\title{Problem Set 5, Tips}
\author{Vikram Damani\\
    Analysis I}

\maketitle
Aufgaben in \textcolor{red}{rot} markiert, Tipps \& Tricks in \textcolor{blue}{blau}.

\section{Theorie}

\begin{definition}{[Fundamentalsatz der Algebra]}
    Sei $p(x)=a_n x^n+a_{n-1}x^{n-1}+ \cdots +a_1x+a_0$ ein Polynom mit Koeffizienten $a_0,a_1,\ldots,a_n\in\C$. Eine Zahl $x_0$ heißt Nullstelle von $p(x)$, falls $p(x_0)=0$.

    Jedes Polynom $p(x)$ vom Grad $n\geq 1$ hat genau $n$ Nullstellen, gezählt mit Vielfachheit. Das Polynom $p(x)$ lässt sich also schreiben als
    \begin{equation}
        p(x)=a_n(x-x_1)(x-x_2)\ldots(x-x_n)
    \end{equation}
    wobei $x_1,x_2,\ldots,x_n$ die Nullstellen von $p(x)$ sind.

    \textbf{Bemerkung:} Die Nullstellen von einem Polynom $p(x)$ mit reellen Koeffizienten $a_k\in\R$ sind nicht notwendigerweise reell. Es gilt jedoch, dass komplexe Nullstellen stets als Komplex konjugierte Paare auftreten, d.h. wenn $x_0\in\C$ eine Nullstelle von $p(x)$ ist, dann ist auch $\overline{x_0}$ eine Nullstelle von $p(x)$.
\end{definition}

\begin{remark}{[$\arcsinh$]}
    Die Funktion $\arcsinh(x)$ ist die Umkehrfunktion von $\sinh(x)$, d.h. $\sinh(\arcsinh(x))=x$.
    Es gilt aus der Vorlesung, dass $\arcsinh(x) = \ln(x + \sqrt{x^2 + 1})$.
\end{remark}

\end{document}