\documentclass[12pt]{article}

\usepackage[margin=1in]{geometry}
\usepackage[utf8]{inputenc}
\usepackage[ngerman]{babel}
\usepackage{parskip}

\usepackage{amsmath}
\usepackage{amssymb}
\usepackage{amsfonts}
\usepackage{enumitem}
\usepackage{mathtools}
\usepackage{mathrsfs}
\usepackage{setspace}
\usepackage{xargs}
\usepackage{xcolor}
\usepackage{witharrows}
\usepackage{commath}
\usepackage{physics}
\usepackage{tikz}
\usetikzlibrary{patterns,hobby}
\usepackage{pgfplots}
\pgfplotsset{compat=1.6}
\usepgfplotslibrary{fillbetween}
\usetikzlibrary{patterns}
\usepackage[outline]{contour} % glow around text
\contourlength{1.0pt}

\tikzset{>=latex} % for LaTeX arrow head
\colorlet{myred}{red!85!black}
\colorlet{myblue}{blue!80!black}
\colorlet{mydarkred}{myred!80!black}
\colorlet{mydarkblue}{myblue!60!black}
\tikzstyle{xline}=[myblue,thick]
\def\tick#1#2{\draw[thick] (#1) ++ (#2:0.09) --++ (#2-180:0.18)}
\tikzstyle{myarr}=[myblue!50,-{Latex[length=3,width=2]}]
\def\Nr{100}

\usepackage[hidelinks]{hyperref}
\usepackage{graphicx}
\usepackage{float}
\usepackage[centerlast,small,sc]{caption}
\usepackage{subcaption} % causes weird error with \setlength{\captionmargin}{20pt}
\usepackage{amsthm}
\usepackage{cancel}


\renewcommand\qedsymbol{$\blacksquare$}


\newcommand{\dx}{\mathrm{d}x}
\newcommand{\ddx}{\frac{\mathrm{d}}{\mathrm{d}x}}
\newcommand{\dt}{\mathrm{d}t}
\newcommand{\du}{\mathrm{d}u}
%\newcommand{\dv}{\mathrm{d}v}
\newcommand{\dy}{\mathrm{d}y}
\newcommand{\dz}{\mathrm{d}z}

\newcommand{\Rn}{\mathbb{R}^n}
\newcommand{\Rm}{\mathbb{R}^m}
\newcommand{\Rk}{\mathbb{R}^k}
\newcommand{\und}{\text{ und }}
\newcommand{\oder}{\text{ oder }}
\newcommand{\bydef}{\underset{def.}{=}}
\newcommand{\BH}{\underset{\textrm{B-H}}{=}}

\newcommand{\Follows}{\Longrightarrow\ }
\newcommand{\sameas}{\Longleftrightarrow}
\newcommandx{\Laplace}[2][1=f(t), 2=s]{\mathscr{L}\{#1\}(#2)}
\newcommandx{\LaplaceInv}[2][1=F(s), 2=t]{\mathscr{L}^{-1}\{#1\}(#2)}
\DeclareMathOperator{\arccosh}{Arcosh}
\DeclareMathOperator{\arcsinh}{Arsinh}
\DeclareMathOperator{\arctanh}{Artanh}
\DeclareMathOperator{\arcsech}{arcsech}
\DeclareMathOperator{\arccsch}{arcCsch}
\DeclareMathOperator{\arccoth}{arcCoth} 

\def\doubleunderline#1{\underline{\underline{#1}}}
\def\ez{\begin{flushright}\underline{ez.}\end{flushright}}
%\[
%   \Laplace[\cos(x)]=\int_{t=0}^{\infty}f(t)e^{-st}dt
%\]

\newcommand{\R}{\mathbb{R}} % gives blackboard R
\newcommand{\Z}{\mathbb{Z}}
\newcommand{\N}{\mathbb{N}}
\newcommand{\Q}{\mathbb{Q}}
\newcommand{\C}{\mathbb{C}}

\newcommand{\inttext}{\shortintertext}

\newenvironment{definition}[2][Definition]{\begin{trivlist}
        \item[\hskip \labelsep {\bfseries #1}\hskip \labelsep {\bfseries #2.}]}{\flushright{$\square$}\end{trivlist}}
\newenvironment{lemma}[2][Theorem]{\begin{trivlist}
        \item[\hskip \labelsep {\bfseries #1}\hskip \labelsep {\bfseries #2.}]}{\flushright{$\square$}\end{trivlist}}
\newenvironment{notation}[2][Notation]{\begin{trivlist}
        \item[\hskip \labelsep {\bfseries #1}\hskip \labelsep {\bfseries #2.}]}{\end{trivlist}}
\newenvironment{problem}[2][\textcolor{blue}{Tipps \& Tricks zu}]{\begin{trivlist}
        \item[\hskip \labelsep {\bfseries #1}\hskip \labelsep {\bfseries \textcolor{blue}{#2}.}]}{\end{trivlist}}
\newenvironment{question}[2][\textcolor{red}{Aufgabe}]{\begin{trivlist}
        \item[\hskip \labelsep {\bfseries \textcolor{red}{#1}}\hskip \labelsep {\bfseries \textcolor{red}{#2}.}]}{\end{trivlist}}
\newenvironment{remark}[2][Bemerkung]{\begin{trivlist}
        \item[\hskip \labelsep {\bfseries #1}\hskip \labelsep {\bfseries #2.}]}{\end{trivlist}}

\begin{document}
\title{Problem Set 11, Tips}
\author{Vikram R. Damani\\
    Analysis I}

\maketitle
Aufgaben in \textcolor{red}{rot} markiert, Tipps \& Tricks in \textcolor{blue}{blau}.

\section{Theorie}

\begin{definition}{[Bestimmtes Integral]}
    Sei $f: [a,b] \to \R$. Dann ist die Fläche unter dem Graphen von $f$ über dem Intervall $[a,b]$ definiert als
    \begin{align}
        \colorbox{pink}{\mathstrut$A\triangleq\displaystyle\int_{a}^{b}f(x)dx$}
    \end{align}
    das \textbf{bestimmte Integral} von $f$ von $a$ bis $b$.

    \begin{remark}{[Bedingungen für Konvergenz]}
        $f$ darf auf $[a,b]$ an einzelnen, isolierten Stellen unstetig sein.\footnote{Die Menge der Unstetigkeitsstellen von $f$ auf $[a,b]$ muss eine Menge mit Lebesgue-Mass Null sein.} Die Grenzen $a$ und $b$ müssen endlich sein.

        Stetige Funktionen sind in der Regel integrierbar.
    \end{remark}
\end{definition}

\begin{definition}{[Riemannsumme und Riemannintegral]}
    Sei $f: [a,b] \to \R$ eine Funktion. Wir betrachten eine Zerlegung des Intervalls $[a,b]$ in die \textbf{Teilintervalle} $\mathbf{[x_{i-1},x_i]}$ mit \textbf{Intervallbreite}\footnote{Die Intervalle müssen nicht unbedingt regelmässig sein.} $\Delta x_k=x_k-x_{k-1}$ für $i=1,\ldots,n$ und wählen Stützpunkte $\xi_k\in[x_{k-1},x_k]$

    Die \textbf{Riemannsumme} von $f$ bezüglich der Zerlegung
    $\{x_0,x_1,\ldots,x_n\}$ und der Stützpunkte $\xi_k$ ist definiert als
    \begin{align}
        f(\xi_1)\Delta x_1+f(\xi_2)\Delta x_2 + \cdots + f(\xi_n)\Delta x_n & =\sum_{i=1}^{n}f(\xi_i)(x_i-x_{i-1}) \\ &\triangleq\colorbox{pink}{\mathstrut$\displaystyle\sum_{i=1}^{n}f(\xi_i)\Delta x_i$}.
    \end{align}
    \begin{figure}[htbp]
        \centering
        \begin{tikzpicture}[scale=1]
            \def\a{1.7}
            \def\b{5.7}
            \def\c{3.7}
            \def\L{0.49} % width of interval

            \pgfmathsetmacro{\Va}{2*sin(\a r+1)+4} \pgfmathresult
            \pgfmathsetmacro{\Vb}{2*sin(\b r+1)+4} \pgfmathresult
            \pgfmathsetmacro{\Vc}{2*sin(\c r+1)+4} \pgfmathresult

            \draw[->,thick] (-0.5,0) -- (7,0) coordinate (x axis) node[below] {$x$};
            \draw[->,thick] (0,-0.5) -- (0,7) coordinate (y axis) node[left] {$y$};
            \foreach \f in {1.7,2.2,...,6.2} {\pgfmathparse{2*sin(\f r+1)+4} \pgfmathresult
                    \draw[fill=blue!20] (\f-\L/2,\pgfmathresult |- x axis) -- (\f-\L/2,\pgfmathresult) -- (\f+\L/2,\pgfmathresult) -- (\f+\L/2,\pgfmathresult |- x axis) -- cycle;}
            \node at (\a-\L/2,-5pt) {\footnotesize{$a=x_0$}};
            \node at (\b+\L/2+\L,-5pt) {\footnotesize{$b=x_n$}};
            \draw[blue] (\c-\L/2,0) -- (\c-\L/2,\Vc) -- (\c+\L/2,\Vc) -- (\c+\L/2,0);
            \draw[dashed] (\c,0) node[below] {\footnotesize{$\xi_i$}} -- (\c,\Vc) -- (0,\Vc) node[left] {$f(\xi_i)$};
            \node at (\a+5*\L/2,-5pt) {\footnotesize{$x_{i-1}$}};
            \node at (\a+7*\L/2,-5pt) {\footnotesize{$x_i$}};
            \node at (\a+5*\L,-5pt) {\footnotesize{$x_{i+1}$}};
            \draw[blue,thick,smooth,samples=100,domain=1.45:6.2] plot(\x,{2*sin(\x r+1)+4});
            \filldraw[black] (\c,\Vc) circle (.03cm);
        \end{tikzpicture}
    \end{figure}
    Eine Funktion $f: [a,b] \to \R$ heisst \textbf{Riemann-integrierbar}, falls sich die Summe
    für jede Zerlegung $\{x_0,x_1,\ldots,x_n\}$ und jede Wahl der Stützpunkte
    $\{\xi_1,\xi_2,\ldots,\xi_n\},\,\,\xi_i\in[x_{i-1},x_i]$ einem festen Grenzwert $I$ annähert, vorausgesetzt, die Zerlegung ist hinreichend fein. Im Grenzwert $n\to\infty$ und $\Delta x_i\to 0$ gilt
    \begin{align}
        \lim_{n\to\infty,\,\max_{i=1,\ldots,n}(\Delta x_i)\to 0}\sum_{i=1}^{n}f(\xi_i)\Delta x_i = \lim_{n\to\infty, \,\Delta x_i\to 0}\sum_{i=1}^{n}f(\xi_i)\Delta x_i  =I.
    \end{align}
    Dann ist $I$ das bestimmte Integral von $f$ zwischen $a$ und $b$ und wir schreiben
    \begin{align}
        \colorbox{pink}{\mathstrut$\displaystyle\int_{a}^{b}f(x)dx=I$}.
    \end{align}
\end{definition}

\begin{figure}[htbp]
    \centering
    \centering
    \begin{tikzpicture}[scale=1]
        \begin{axis}[axis lines=middle,
                axis line style = thick,
                xlabel=$x$,
                ylabel=$y$,
                enlargelimits,
                ytick=\empty,
                xtick={1,4},
                xticklabels={a,b}]

            \path[name path=axis] (axis cs:0,0) -- (axis cs:4,0);

            \addplot[name path=F,thick,blue,domain={-.2:5}] {0.5*x^2-2*x+5} node[pos=.8, above, anchor=north west]{$f+g$};

            \addplot[name path=G,green,thick,domain={-.2:5}] {-0.1*x^2+2} node[pos=.1, below]{$g$};

            \addplot[pattern=north west lines, pattern color=green!50]fill between[of=G and axis, soft clip={domain=1:4}]
            ;
            \addplot[pattern=north east lines, pattern color=blue!50]fill between[of=F and axis, soft clip={domain=1:4}]
            ;
            \addplot[pattern=horizontal lines, pattern color=magenta!50]fill between[of=F and G, soft clip={domain=1:4}]
            ;
            \node[coordinate,pin=30:{$\mathcolor{blue}{A}$}] at (axis cs:3.8,3){};

        \end{axis}
    \end{tikzpicture}
    \caption{Beispiel für eine Fläche $\mathcolor{blue}{A}=\int_{a}^{b}(f+g)\dx=\mathcolor{magenta}{\int_{a}^{b}f\dx}+\mathcolor{green}{\int_{a}^{b}g\dx}$ unter dem Graphen von $f+g$ auf dem Intervall $[a,b]$.}
\end{figure}

\begin{definition}{[Rechenregeln für Integrale]}
    Seien $f,g: [a,b] \to \R$ stetige Funktionen und $\lambda \in \R$. Dann gelten folgende Rechenregeln:
    \begin{enumerate}
        \item \colorbox{pink}{$\displaystyle\int_{a}^{b}f(x)dx=-\displaystyle\int_{b}^{a}f(x)dx$} \hfill (Symmetrie)
        \item \colorbox{pink}{$\displaystyle\int_{a}^{b}(\lambda f(x)+g(x))dx=\lambda \displaystyle\int_{a}^{b}f(x)dx+\displaystyle\int_{a}^{b}g(x)dx$} \hfill (Linearität)
        \item \colorbox{pink}{$\displaystyle\int_{a}^{b}f(x)dx+\displaystyle\int_{b}^{c}f(x)dx=\displaystyle\int_{a}^{c}f(x)dx$} \hfill (Additivität)
    \end{enumerate}
\end{definition}

\begin{figure}[htbp]
    \centering
    \def\xmax{2.2} % max x axis
    \def\ymax{1.6} % max y axis
    \begin{subfigure}[t]{0.45\textwidth}
        \centering
        % CONSTANT
        \begin{tikzpicture}[scale=2.0]
            \def\a{0.17*\xmax} % first limit
            \def\b{0.76*\xmax} % last limit
            \def\k{0.65*\ymax} % constant value
            \fill[myblue!10] (\a,0) rectangle (\b,\k);
            \draw[->,thick] (0,-0.15*\ymax) -- (0,\ymax) node[left] {$y$};
            \draw[->,thick] (-0.15*\ymax,0) -- (\xmax,0) node[right=1,below] {$x$};
            \draw[xline,line cap=round] (0,\k) -- (0.9*\xmax,\k)
            node[mydarkblue,left=7,above=0,scale=0.9] {$y=k$};
            \fill[mydarkblue] (\a,\k) circle(0.04) (\b,\k) circle(0.04);
            \draw[dashed] (\a,0) --++ (0,1.07*\k);
            \draw[dashed] (\b,0) --++ (0,1.07*\k);
            \tick{\a,0}{90} node[below=-2.5,scale=0.8] {\strut$a$};
            \tick{\b,0}{90} node[below=-2.5,scale=0.8] {\strut$b$};
            \tick{0,\k}{0} node[left=-1,scale=0.8] {$k=\xi$};
        \end{tikzpicture}
        \caption{Fläche unter einer konstanten Funktion auf dem Intervall $[a,b]$.}
    \end{subfigure}
    % \begin{subfigure}[t]{0.45\textwidth}
    %     \centering
    %     % LINEAR
    %     \begin{tikzpicture}[scale=2.0]
    %         \def\a{0.17*\xmax}  % first limit
    %         \def\b{0.76*\xmax}  % last limit
    %         \def\k{\ymax/\xmax} % slope coefficient
    %         \fill[myblue!10] (\a,0) -- (\a,\k*\a) -- (\b,\k*\b) |- cycle;
    %         \draw[->,thick] (0,-0.15*\ymax) -- (0,\ymax+0.1) node[left] {$y$};
    %         \draw[->,thick] (-0.15*\ymax,0) -- (\xmax+0.1,0) node[right=1,below] {$x$};
    %         \draw[xline,line cap=round]
    %         (-0.1*\ymax,-0.1*\k*\ymax) -- (0.9*\xmax,0.9*\k*\xmax)
    %         node[mydarkblue,above left=-3,scale=0.9] {$y=kx$};
    %         \fill[mydarkblue]
    %         (\a,\k*\a) circle(0.04) (\b,\k*\b) circle(0.04);
    %         \draw[dashed] (\a,0) --++ (0,1.25*\k*\a);
    %         \draw[dashed] (\b,0) --++ (0,1.10*\k*\b);
    %         \tick{\a,0}{90} node[below=-2,scale=0.8] {\strut$a$};
    %         \tick{\b,0}{90} node[below=-2,scale=0.8] {\strut$b$};
    %     \end{tikzpicture}
    %     \caption{filler}
    % \end{subfigure}
    \begin{subfigure}[t]{0.45\textwidth}
        \centering
        % SINE
        \def\xmax{3} % max x axis
        \begin{tikzpicture}[scale=2.0]
            \message{^^JSine}
            \def\ymax{1.0} % max y axis
            \def\T{0.9*\xmax} % period
            \def\A{0.88*\ymax} % amplitude
            \fill[myblue!10,samples=\Nr,smooth,variable=\x,domain=0:\T/2]
            plot(\x,{\A*sin(360/(\T)*\x});
            \draw[->,thick] (0,-\ymax+0.25) -- (0,\ymax+0.1) node[left] {$y$};
            \draw[->,thick] (-0.15*\ymax,0) -- (\xmax,0) node[right=1,below] {$x$};
            \draw[xline,samples=\Nr,smooth,variable=\x,domain=0:0.8*\xmax]
            plot(\x,{\A*sin(360/(\T)*\x)});
            \draw[dashed] (0,2*\A/pi) --++ (0.53*\T,0);
            \draw[dashed] (0.5*\T,0) --++ (0,2.4*\A/pi);
            \tick{\T/2,0}{90} node[left=1,below=-2,scale=0.8] {\strut$\pi$}; %{\strut\contour{white}{$\pi$}};
            \tick{\T,0}{90} node[right=2,below=-2,scale=0.8] {\strut$2\pi$}; %{\strut\contour{white}{$2\pi$}};
            \tick{0,2*\A/pi}{0} node[left=-1,scale=0.8] {$\dfrac{2}{\pi}=\xi$};
            \node[above right=-2,mydarkblue,scale=0.9] at (0.3*\T,\A) {$y=\sin x$};
        \end{tikzpicture}
        \caption{Fläche unter einer Sinusfunktion auf dem Intervall $[0,\pi]$. Die Fläche beträgt $2 = \xi(b-a)$.}
    \end{subfigure}
    % \begin{subfigure}[t]{0.45\textwidth}
    %     \centering
    %     % SINE^2, COSINE^2
    %     \begin{tikzpicture}[scale=2.0]
    %         \def\T{0.60*\xmax} % period
    %         \def\A{0.88*\ymax} % amplitude
    %         \fill[myblue!10,samples=\Nr,smooth,variable=\x,domain=0:\T]
    %         plot(\x,{\A*sin(360/(\T)*\x)^2});
    %         \fill[myred!50,opacity=0.2,samples=\Nr,smooth,variable=\x,domain=0:\T]
    %         (0,0) -- plot(\x,{\A*cos(360/(\T)*\x)^2}) |- cycle;
    %         \draw[->,thick] (0,-0.15*\ymax) -- (0,\ymax+0.1) node[left] {$y$};
    %         \draw[->,thick] (-0.15*\ymax,0) -- (\xmax+0.1,0) node[right=1,below] {$x$};
    %         \draw[xline,samples=\Nr,smooth,variable=\x,domain=0:0.94*\xmax]
    %         plot(\x,{\A*sin(360/(\T)*\x)^2});
    %         \draw[xline,myred,samples=\Nr,smooth,variable=\x,domain=0:0.94*\xmax]
    %         plot(\x,{\A*cos(360/(\T)*\x)^2});
    %         \draw[dashed] (0,\A/2) --++ (0.96*\xmax,0); %1.05*\T
    %         \draw[dashed] (\T,0) --++ (0,1.1*\A);
    %         \tick{\T/2,0}{90} node[left=1,below=-2,scale=0.8] {\strut$\pi$};
    %         \tick{\T,0}{90} node[right=0,below=-2,scale=0.8] {\strut$2\pi$};
    %         \tick{1.5*\T,0}{90} node[right=0,below=-2,scale=0.8] {\strut$3\pi$};
    %         \tick{0,\A/2}{0} node[left=-1,scale=0.8] {$\dfrac{1}{2}$};
    %         \node[above right=0,mydarkblue,scale=0.9] at (0.23*\T,\A) {$y=\sin^2x$};
    %         \node[above right=0,mydarkred,scale=0.9] at (0.99*\T,\A) {$y=\cos^2x$};
    %     \end{tikzpicture}
    % \end{subfigure}
    \caption{Integrale einiger Funktionen und Anwendung des Mittelwertsatzes der Integralrechnung.}
\end{figure}

\begin{definition}{[Stammfunktion]}
    Sei $f: I \to \R$ eine Funktion. Eine Funktion $F: I \to \R$ heisst \textbf{Stammfunktion} von $f$, falls $F$ auf $I$ differenzierbar ist und
    \begin{align}
        F'(x)=f(x) \text{ für alle } x \in I.
    \end{align}

    \begin{remark}{[Viele Stammfunktionen]}
        Sei $F$ eine Stammfunktion von $f$. Dann ist auch $F+c$ eine Stammfunktion von $f$ für jede Konstante $c\in\R$.
    \end{remark}
\end{definition}

\begin{definition}{[Mittelwertsatz der Integralrechnung]}
    Sei $f: [a,b] \to \R$ stetig. Dann existiert ein $\xi \in (a,b)$, so dass
    \begin{align}
        \colorbox{pink}{\mathstrut$\displaystyle\int_{a}^{b}f(x)dx=f(\xi)(b-a)$}.
    \end{align}
\end{definition}

\begin{lemma}{[Hauptsatz der Differential- und Integralrechnung]}
    Sei $f: [a,b] \to \R$ eine stetige Funktion. Dann ist die Funktion $F: [a,b] \to \R$ definiert durch
    \begin{align}
        \colorbox{pink}{$\displaystyle F_a(x)=\int_{a}^{x}f(t)dt$}
    \end{align}
    eine Stammfunktion von $f$. Es gilt also
    \begin{align}
        \colorbox{pink}{$\displaystyle F_a'(x)=f(x) \text{ für alle } x \in [a,b].$}
    \end{align}
    $F_a$ ist unabhänging von der Wahl des unteren Integrationslimits $a$.

    \begin{proof}
        Sei $x \in [a,b]$. Dann ist die Funktion $F$ auf $[a,x]$ stetig und auf $(a,x)$ differenzierbar. Nach dem Hauptsatz der Differential- und Integralrechnung gilt
        \begin{align*}
            F'(x) & =\lim_{h \to 0}\frac{F(x+h)-F(x)}{h}                             \\
                  & =\lim_{h \to 0}\frac{\int_{a}^{x+h}f(t)dt-\int_{a}^{x}f(t)dt}{h} \\
                  & =\lim_{h \to 0}\frac{\int_{x}^{x+h}f(t)dt}{h}                    \\
                  \inttext{Nach dem Mittelwertsatz der Integralrechnung gibt es ein $\theta \in (x,x+h)$ so dass}
                  & =\lim_{h \to 0}\frac{f(\theta)h}{h}\\
                  &=\lim_{h \to 0}\frac{f(x+\phi h)h}{h}=f(x). 
        \end{align*}
        wobei $\phi \in (0,1)$.
    \end{proof}
\end{lemma}

\begin{definition}{[Unbestimmtes Integral]}
    Sei $f: I \to \R$ eine Funktion. Dann ist die Menge aller Stammfunktionen von $f$ definiert als
    \begin{align}
        \colorbox{pink}{\mathstrut$\displaystyle\int f(x)dx = F(x)+c$}.
    \end{align}
    Das ``$+c$'' (die \textbf{Integrationskonstante}) ist nötig um die ganze Menge der Stammfunktionen zu beschreiben.
\end{definition}

\begin{definition}{[Integrale Berechnen]}
    Sei $F$ eine Stammfunktion von $f$. Dann gilt
    \begin{align}
        \colorbox{pink}{\mathstrut$\displaystyle\int_{a}^{b}f(x)dx=F(b)-F(a)$}.
    \end{align}
    \textbf{Notation:}
    Das Integral von $f$ über das Intervall $[a,b]$ wird auch als
    \begin{align}
        \colorbox{pink}{\mathstrut$\displaystyle\int_{a}^{b}f(x)dx\triangleq F(x)\Big\vert_a^b\triangleq \left[F(x)\right]_a^b$}
    \end{align}
    geschrieben.

    \textbf{Beispiel:}
    \begin{enumerate}
        \item $\displaystyle\int_{0}^{1}x^2dx=\left[\frac{x^3}{3}\right]_0^1=\frac{1}{3}$
        \item Allgemein gilt
              $\displaystyle\int_{a}^{b}x^ndx=\left[\frac{x^{n+1}}{n+1}\right]_a^b=\frac{b^{n+1}-a^{n+1}}{n+1}$
    \end{enumerate}
\end{definition}

\begin{definition}{[Partielle Integration]}
    Seien $u(x)$ und $v(x)$ differenzierbare Funktionen. Dann gilt
    \begin{align}
        \colorbox{pink}{\mathstrut$\displaystyle\int u(x)v'(x)dx=u(x)v(x)-\displaystyle\int u'(x)v(x)dx$}.
    \end{align}
    \textbf{Herleitung:} Mit der Produktregel gilt
    \begin{align*}
        \ddx(u(x)v(x))  =u(x)v'(x)+u'(x)v(x) \Follows u(x)v'(x)=\ddx(u(x)v(x))-u'(x)v(x) \\
    \end{align*}
    und somit
    \begin{align*}
        \displaystyle\int u(x)v'(x)dx=\displaystyle\int \ddx(u(x)v(x))dx-\displaystyle\int u'(x)v(x)dx=u(x)v(x)-\displaystyle\int u'(x)v(x)dx.
    \end{align*}

    \textbf{Tricks:}
    \begin{enumerate}
        \item ``Nach dem Integral $I$ auflösen''.
        \item 2x partielle Integration und dann nach $I$ auflösen.
        \item Mit $1$ multiplizieren.
        \item Wähle $u(x)$ oder $v(x)$ so, dass nach (mehrmaligem) Ableiten die Funktion
              verschwindet.
    \end{enumerate}

    \textbf{Beispiel:}
    \begin{align*}
        \displaystyle\int x\sin x\dx & =-x\cos x+\displaystyle\int \cos x\dx=-x\cos x+\sin x+c
    \end{align*}
\end{definition}

\begin{definition}{[Substitution]}
    Sei $f: I \to \R$ eine Funktion und $g: J \to I$ eine differenzierbare Funktion. Dann gilt
    \begin{align}
        \colorbox{pink}{\mathstrut$\displaystyle\int f(g(x))g'(x)\dx=\displaystyle\int f(u)\du$}
    \end{align}
    wobei $u=g(x)$.

    \textbf{Herleitung:}
    \begin{align*}
        \frac{\dd{u}}{\dd{x}}=g'(x) \Follows \dd{u}=g'(x)\dx \Follows \dx=\frac{1}{g'(x)}\du \\
        \int f(g(x))g'(x)\dx=\int f(u)g'(x)\frac{1}{g'(x)}\du=\int f(u)\du
    \end{align*}

    Bei bestimmten Integralen muss auch das Intervall angepasst werden.
    \begin{align}
        \colorbox{pink}{\mathstrut$\displaystyle\int_{a}^{b}f(g(x))g'(x)\dx=\displaystyle\int_{g(a)}^{g(b)}f(u)\du$}
    \end{align}

    \textbf{Bemerkung:} Manchmal ist es nötig, allgemeiner $h(u)=g(x)$ zu setzen und dann $h'(u)\du=g'(x)\dx$ zu verwenden.

    \textbf{Beispiel:}
    \begin{align*}
        \displaystyle\int x\cos(x^2)dx & =\frac{1}{2}\displaystyle\int \cos(u)du=\frac{1}{2}\sin(u)+c=\frac{1}{2}\sin(x^2)+c
    \end{align*}
\end{definition}

\textbf{Allgemeie Integrationstricks:}
\begin{enumerate}
    \item Symmetrien ausnutzen. \textbf{Beispiel:} $\displaystyle\int_{-a}^{a}f(x)dx=0$ für ungerade Funktionen $f$.
    \item Linearität und Additivität ausnutzen.
    \item Partialbruchzerlegung und Polynomdivision verwenden um Brüche zu vereinfachen. \\[5pt]\textbf{Beispiel:} $\displaystyle\int \frac{1}{x^2-1}dx=\displaystyle\int \frac{1}{2}\left(\frac{1}{x-1}-\frac{1}{x+1}\right)dx$.
    \item Trigonometrische und Hyperbolische Identitäten verwenden.
\end{enumerate}

\end{document}