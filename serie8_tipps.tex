\documentclass[12pt]{article}

\usepackage[margin=1in]{geometry}
\usepackage[utf8]{inputenc}
\usepackage[ngerman]{babel}
\usepackage{parskip}

\usepackage{amsmath}
\usepackage{amssymb}
\usepackage{amsfonts}
\usepackage{enumitem}
\usepackage{mathtools}
\usepackage{mathrsfs}
\usepackage{setspace}
\usepackage{xargs}
\usepackage{xcolor}
\usepackage{witharrows}
\usepackage{commath}
\usepackage{physics}
\usepackage{tikz}
\usepackage[hidelinks]{hyperref}
\usepackage{graphicx}
\usepackage{float}
\usepackage[centerlast,small,sc]{caption}
\usepackage{subcaption} % causes weird error with \setlength{\captionmargin}{20pt}

\newcommand{\dx}{\mathrm{d}x}
\newcommand{\ddx}{\frac{\mathrm{d}}{\mathrm{d}x}}
\newcommand{\dt}{\mathrm{d}t}
\newcommand{\du}{\mathrm{d}u}
%\newcommand{\dv}{\mathrm{d}v}
\newcommand{\dy}{\mathrm{d}y}
\newcommand{\dz}{\mathrm{d}z}

\newcommand{\Rn}{\mathbb{R}^n}
\newcommand{\Rm}{\mathbb{R}^m}
\newcommand{\Rk}{\mathbb{R}^k}
\newcommand{\und}{\text{ und }}
\newcommand{\oder}{\text{ oder }}
\newcommand{\bydef}{\underset{def.}{=}}
\newcommand{\BH}{\underset{\textrm{B-H}}{=}}

\newcommand{\Follows}{\Longrightarrow\ }
\newcommand{\sameas}{\Longleftrightarrow}
\newcommandx{\Laplace}[2][1=f(t), 2=s]{\mathscr{L}\{#1\}(#2)}
\newcommandx{\LaplaceInv}[2][1=F(s), 2=t]{\mathscr{L}^{-1}\{#1\}(#2)}
\DeclareMathOperator{\arccosh}{Arcosh}
\DeclareMathOperator{\arcsinh}{Arsinh}
\DeclareMathOperator{\arctanh}{Artanh}
\DeclareMathOperator{\arcsech}{arcsech}
\DeclareMathOperator{\arccsch}{arcCsch}
\DeclareMathOperator{\arccoth}{arcCoth} 

\def\doubleunderline#1{\underline{\underline{#1}}}
\def\ez{\begin{flushright}\underline{ez.}\end{flushright}}
%\[
%   \Laplace[\cos(x)]=\int_{t=0}^{\infty}f(t)e^{-st}dt
%\]

\newcommand{\R}{\mathbb{R}} % gives blackboard R
\newcommand{\Z}{\mathbb{Z}}
\newcommand{\N}{\mathbb{N}}
\newcommand{\Q}{\mathbb{Q}}
\newcommand{\C}{\mathbb{C}}

\newcommand{\inttext}{\shortintertext}

\newenvironment{definition}[2][Definition]{\begin{trivlist}
        \item[\hskip \labelsep {\bfseries #1}\hskip \labelsep {\bfseries #2.}]}{\flushright{$\square$}\end{trivlist}}
\newenvironment{lemma}[2][Theorem]{\begin{trivlist}
        \item[\hskip \labelsep {\bfseries #1}\hskip \labelsep {\bfseries #2.}]}{\flushright{$\square$}\end{trivlist}}
\newenvironment{exercise}[2][Exercise]{\begin{trivlist}
        \item[\hskip \labelsep {\bfseries #1}\hskip \labelsep {\bfseries #2.}]}{\end{trivlist}}
\newenvironment{problem}[2][\textcolor{blue}{Tipps \& Tricks zu}]{\begin{trivlist}
        \item[\hskip \labelsep {\bfseries #1}\hskip \labelsep {\bfseries \textcolor{blue}{#2}.}]}{\end{trivlist}}
\newenvironment{question}[2][\textcolor{red}{Aufgabe}]{\begin{trivlist}
        \item[\hskip \labelsep {\bfseries \textcolor{red}{#1}}\hskip \labelsep {\bfseries \textcolor{red}{#2}.}]}{\end{trivlist}}
\newenvironment{remark}[2][Bemerkung]{\begin{trivlist}
        \item[\hskip \labelsep {\bfseries #1}\hskip \labelsep {\bfseries #2.}]}{\end{trivlist}}

\begin{document}
\title{Problem Set 8, Tips}
\author{Vikram Damani\\
        Analysis I}

\maketitle
Aufgaben in \textcolor{red}{rot} markiert, Tipps \& Tricks in \textcolor{blue}{blau}.

\begin{problem}{1}

\end{problem}
 (Definitionen und Sätze aus den Tipps \& Tricks für Serie 5:)
\begin{definition}{[Extremalstellen]}
        Eine Funktion $f(x)$ hat eine Extremalstelle an der Stelle $x_0$, falls $f'(x_0)=0$.
\end{definition}

\begin{definition}{[Höhere Ableitungen]}
        Sei $f:\mathcal{D}(f)\to\R$ diff'bar. Dann ist
        \begin{align}
                \frac{d}{dx}f=f':\mathcal{D}(f)\to\R
        \end{align}
        die erste Ableitung. Falls $f'$ diff'bar, ist die zweite Ableitung:
        \begin{align}
                \frac{d^2}{dx^2}f=(f')':\mathcal{D}(f')\to\R
        \end{align}

        Allgemein, ist die $n$-te Ableitung von $f(x)$ ist definiert als:
        \begin{align*}
                f^{(n)}(x)=\dfrac{d^n}{dx^n}f(x)=\underbrace{\dfrac{d}{dx}(\dfrac{d}{dx}(\ldots\dfrac{d}{dx}}_{\text{n mal}}f(x))),\quad x\in\mathcal{D}(f^{(n)})
        \end{align*}
        wobei $f^{(0)}=f$.
\end{definition}
\begin{definition}{[Maxima und Minima mit höheren Ableitungen]}
        Sei $f(x)$ eine Funktion, die in $x_0$ mindestens zweimal differenzierbar ist. Falls $f'(x_0)=0$ und $f''(x_0)>0$, dann hat $f(x)$ in $x_0$ ein lokales Minimum. Falls $f'(x_0)=0$ und $f''(x_0)<0$, dann hat $f(x)$ in
        $x_0$ ein lokales Maximum.

        \textbf{Achtung:} Am Rand des Definitionsbereiches sind Ableitungen nicht definiert. Es kann dort dennoch lokale Maximal- oder Minimalstellen geben!
\end{definition}

\begin{definition}{[Konkavität und Konvexität]}
        Eine Funktion ist in eimem Intervall $I$ \textit{konvex}, falls sie oberhalb ihrer Tangente liegt. Anders gesagt, eine Funktion ist in einem Intervall $I$ konvex, falls für alle $x_1,x_2\in I$, die Gerade durch $(x_1,f(x_1))$ und $(x_2,f(x_2))$ oberhalb der Funktion liegt.

        \textit{Wir nennen eine Funktion konvex, falls sie im ganzen Definitionsbereich konvex ist.}

        Die Funktion $f(x)=x^2$ ist ein Beispiel für eine konvexe Funktion.

        \begin{figure}[ht]
                \begin{subfigure}[t]{0.4\textwidth}
                        \centering
                        \begin{tikzpicture}
                                \draw[->] (-3,0) -- (3,0) node[right] {$x$};
                                \draw[->] (0,-1) -- (0,5) node[above] {$f(x)$};
                                \draw[domain=-3.25:3.25,smooth,variable=\x,blue] plot ({\x},{0.5*(\x)^2}) node[right] {$f(x)=x^2$};

                                \draw[domain=-2:3,smooth,variable=\x,red] plot ({\x},{1/2*\x+3});
                                \filldraw[red] (-2, 2) circle (2pt);
                                \filldraw[red] (3, 4.5) circle (2pt);
                                \draw[domain=-1:2.5,smooth,variable=\x,red] plot ({\x},{(3.125-0.5)/3.5*\x+1.25});
                                \filldraw[red] (-1, 0.5) circle (2pt);
                                \filldraw[red] (2.5, 3.125) circle (2pt);
                        \end{tikzpicture}
                        \caption{Ein Beispiel für eine konvexe Funktion.}
                \end{subfigure}
                \hfill
                \begin{subfigure}[t]{0.4\textwidth}
                        \centering
                        \begin{tikzpicture}
                                \draw[->] (-3,0) -- (3,0) node[right] {$x$};
                                \draw[->] (0,-5) -- (0,1) node[above] {$f(x)$};
                                \draw[domain=-3.25:3.25,smooth,variable=\x,blue] plot ({\x},{-0.5*(\x)^2}) node[right] {$f(x)=-x^2$};

                                \draw[domain=-3:2,smooth,variable=\x,red] plot ({\x},{1/2*\x-3});
                                \filldraw[red] (2, -2) circle (2pt);
                                \filldraw[red] (-3, -4.5) circle (2pt);
                                \draw[domain=-1:2.5,smooth,variable=\x,red] plot ({\x},{-(3.125-0.5)/3.5*\x-1.25});
                                \filldraw[red] (-1, -0.5) circle (2pt);
                                \filldraw[red] (2.5, -3.125) circle (2pt);
                        \end{tikzpicture}
                        \caption{Ein Beispiel für eine konkave Funktion.}
                \end{subfigure}
                \caption{Beispiele.}
        \end{figure}

        Eine Funktion ist in einem Intervall $I$ \textit{konkav}, falls sie unterhalb
        ihrer Tangente liegt, oder äquivalent, falls für alle $x_1,x_2\in I$, die
        Gerade durch $(x_1,f(x_1))$ und $(x_2,f(x_2))$ unterhalb der Funktion liegt.

        \textit{Wir nennen eine Funktion konkav, falls sie im ganzen Definitionsbereich konkav ist.}

        Die Funktion $f(x)=-x^2$ ist ein Beispiel für eine konkave Funktion. Die
        Funktion $f(x)=-x+1$ ist ein Beispiel für eine konvexe Funktion.

        Wir können auch die zweite Ableitung verwenden, um Konvexität und Konkavität zu
        bestimmen.
        \begin{align*}
                f''(x)>0\Rightarrow f(x)\text{ ist \textit{konvex} in }I \\
                f''(x)<0\Rightarrow f(x)\text{ ist \textit{konkav} in }I
        \end{align*}
        Eine Funktion ist in einem Intervall $I$ konvex, falls $f''(x)> 0$ für alle $x\in I$. Eine Funktion ist in einem Intervall $I$ konkav, falls $f''(x)< 0$ für alle $x\in I$.

        \textbf{Etwas mathematischer:}
        Eine Funktion $f(x)$ ist in einem Intervall $I$ konvex, falls für alle $x_1,x_2\in I$ und $0\leq\lambda\leq 1$ gilt:
        \begin{align}
                f(\lambda x_1+(1-\lambda)x_2)\leq\lambda f(x_1)+(1-\lambda)f(x_2)
        \end{align}
        Eine Funktion $f(x)$ ist in einem Intervall $I$ konkav, falls für alle $x_1,x_2\in I$ und $0\leq\lambda\leq 1$ gilt:
        \begin{align}
                f(\lambda x_1+(1-\lambda)x_2)\geq\lambda f(x_1)+(1-\lambda)f(x_2)
        \end{align}
        Die funktionen $l(\lambda)=\lambda x_1+(1-\lambda)x_2$ sind lineare Interpolationen (also Geraden) durch $x_1$ und $x_2$.
\end{definition}

\begin{definition}{[Wendepunkte]}
        Eine Funktion $f(x)$ hat einen Wendepunkt an der Stelle $x_0$, falls $f''(x_0)=0$ und $f''(x)$ das Vorzeichen wechselt.

        \begin{remark}{[Wendepunkte und Sattelpunkte]}
                Bei einem Sattelpunkt gilt zusatzlich $f'(x_0)=0$.
        \end{remark}
\end{definition}

\textbf{\textcolor{blue}{Theorie zu 2 \& 3.}}

\begin{definition}{[Ebene Kurven]}
        Ebene Kurven sind ``eindimensionale'' Teilmengen des $\R^2$.

        Eine Kurve $C$ ist eine Menge von Punkten $(x,y)$, die auf verschiedene Weisen
        beschrieben werden können:
        \begin{enumerate}
                \item Parametrisierung: $C=\{(x(t),y(t))\mid t\in I\subseteq\R\}$, wobei $x(t)$ und
                      $y(t)$ Funktionen sind, die $t$ auf $x$ und $y$ abbilden.
                \item Implizite Darstellung: $C=\{(x,y)\mid F(x,y)=0\}$, wobei $F(x,y)$ eine Funktion
                      von $x$ und $y$ ist.
                \item Explizite Darstellung: $C=\{(x,y)\subset\R^2\}$. $y$ lässt sich dann anhand
                      (mehrerer) Funktionen von $x$ ausdrücken.

                      Zum Beispiel:
                      \begin{align*}
                              C=\{(x,y)\mid y=\pm\sqrt{1-x^2}\}, \quad \text{ein Kreis mit Radius 1.}
                      \end{align*}
                      \begin{figure}[hbt!]
                              \centering
                              \begin{subfigure}[][][]{0.45\textwidth}
                                      \centering
                                      \begin{tikzpicture}
                                              \draw[->] (-1.5,0) -- (1.5,0) node[right] {$x$};
                                              \draw[->] (0,-0.5) -- (0,1.5) node[above] {$y$};
                                              \draw[thick, blue] (-1,0) arc (180:0:1) node[right, yshift=0.5cm] {$y=\sqrt{1-x^2}$};
                                      \end{tikzpicture}
                              \end{subfigure}
                              \begin{subfigure}[][][]{0.45\textwidth}
                                      \centering
                                      \begin{tikzpicture}
                                              \draw[->] (-1.5,0) -- (1.5,0) node[right] {$x$};
                                              \draw[->] (0,-1.5) -- (0,0.5) node[above] {$y$};
                                              \draw[thick, blue] (-1,0) arc (-180:0:1) node[right, yshift=0.5cm] {$y=-\sqrt{1-x^2}$};
                                      \end{tikzpicture}
                              \end{subfigure}
                      \end{figure}
        \end{enumerate}
        \begin{remark}{[Von Parametrisierung zu impliziten Darstellung]}
                Um von einer Parametrisierung zu einer impliziten Darstellung zu kommen, muss man in den Parametrisierungen $x(t)$ und $y(t)$ $t$ eliminieren. Dazu kann man z.B. die Umkehrfunktion $t(x)$ bestimmen und in $y(t)$ einsetzen.
        \end{remark}
\end{definition}



\end{document}