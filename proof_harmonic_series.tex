\documentclass[12pt]{article}

\usepackage[margin=1in]{geometry}

\usepackage{amsmath}
\usepackage{amssymb}
\usepackage{amsfonts}
\usepackage{enumitem}
\usepackage{mathtools}
\usepackage{mathrsfs}
\usepackage{setspace}
\usepackage{xargs}
\usepackage{witharrows}
\usepackage{amsthm}

\newcommand{\Follows}{\Longrightarrow\ }
\newcommand{\sameas}{\Longleftrightarrow}
\newcommandx{\Laplace}[2][1=f(t), 2=s]{\mathscr{L}\{#1\}(#2)}

\def\doubleunderline#1{\underline{\underline{#1}}}
\def\ez{\begin{flushright}\underline{ez.}\end{flushright}}
%\[
%   \Laplace[\cos(x)]=\int_{t=0}^{\infty}f(t)e^{-st}dt
%\]

\newcommand{\R}{\mathbb{R}} % gives blackboard R
\newcommand{\Z}{\mathbb{Z}}
\newcommand{\N}{\mathbb{N}}
\newcommand{\Q}{\mathbb{Q}}
\newcommand{\C}{\mathbb{C}}

\newcommand{\inttext}{\shortintertext}

\newenvironment{theorem}[2][Theorem]{\begin{trivlist}
        \item[\hskip \labelsep {\bfseries #1}\hskip \labelsep {\bfseries #2.}]}{\end{trivlist}}
\newenvironment{lemma}[2][Lemma]{\begin{trivlist}
        \item[\hskip \labelsep {\bfseries #1}\hskip \labelsep {\bfseries #2.}]}{\end{trivlist}}
\newenvironment{exercise}[2][Exercise]{\begin{trivlist}
        \item[\hskip \labelsep {\bfseries #1}\hskip \labelsep {\bfseries #2.}]}{\end{trivlist}}
\newenvironment{problem}[2][Problem]{\begin{trivlist}
        \item[\hskip \labelsep {\bfseries #1}\hskip \labelsep {\bfseries #2.}]}{\end{trivlist}}
\newenvironment{question}[2][Question]{\begin{trivlist}
        \item[\hskip \labelsep {\bfseries #1}\hskip \labelsep {\bfseries #2.}]}{\end{trivlist}}
\newenvironment{corollary}[2][Corollary]{\begin{trivlist}
        \item[\hskip \labelsep {\bfseries #1}\hskip \labelsep {\bfseries #2.}]}{\end{trivlist}}

\begin{document}
\title{Proof of Divergence Harmonic Series}
\author{Vikram Damani\\
        Analysis I}

\maketitle

\begin{question}{[Divergence of Harmonic Series]}
        Prove that the harmonic series diverges.
        \begin{proof}
                \begin{spreadlines}{0.8em}
                        (Proof by contradiction) Suppose that the harmonic series converges. Let $S$ be the sum of the harmonic series, i.e.
                        \begin{align*}
                                S=\sum_{n=1}^{\infty}\frac{1}{n}=1+\frac{1}{2}+\frac{1}{3}+\cdots.
                        \end{align*}

                        We may group the terms of the harmonic series into groups of 3 as follows:
                        \begin{align*}
                                S & =1+(\frac{1}{2}+\frac{1}{3}+\frac{1}{4})+(\frac{1}{5}+\frac{1}{6}+\frac{1}{7})+\cdots
                        \end{align*}

                        For all $x\in\R,x>1$, we have that
                        $\frac{1}{x-1}+\frac{1}{x}+\frac{1}{x+1}>\frac{3}{x}$. This can be shown by
                        expanding the left hand side and simplifying as follows:
                        \begin{align*}
                                \frac{1}{x-1}+\frac{1}{x}+\frac{1}{x+1} & =\frac{(x^2+x)+(x^2-1)+(x^2-x)}{x(x-1)(x+1)} \\
                                                                        & =\frac{3x^2-1}{x(x^2-1)}                     \\
                                                                        & =\frac{2x^2+(x^2-1)}{x(x^2-1)}               \\
                                                                        & =\frac{2x}{x^2-1}+\frac{1}{x}                \\
                                                                        & >\frac{2x}{x^2}+\frac{1}{x}                  \\
                                                                        & =\frac{3}{x}.
                        \end{align*}
                        Therefore, choosing $x=3,6,9,\cdots$ we have that
                        \begin{align*}
                                \frac{1}{2}+\frac{1}{3}+\frac{1}{4}  & >\frac{3}{3}=\frac{1}{1} \\
                                \frac{1}{5}+\frac{1}{6}+\frac{1}{7}  & >\frac{3}{6}=\frac{1}{2} \\
                                \frac{1}{8}+\frac{1}{9}+\frac{1}{10} & >\frac{3}{9}=\frac{1}{3} \\
                                \vdots
                        \end{align*}
                        Proceeding in this manner, we have that
                        \begin{align*}
                                S & =1+(\frac{1}{2}+\frac{1}{3}+\frac{1}{4})+(\frac{1}{5}+\frac{1}{6}+\frac{1}{7})+\cdots \\
                                  & >1+\frac{1}{1}+\frac{1}{2}+\frac{1}{3}+\cdots                                         \\
                                  & = 1 + 1 + \frac{1}{2}+\frac{1}{3}+\cdots                                              \\
                                  & = 1 + S.                                                                              \\
                                \therefore S > 1 + S.
                        \end{align*}
                        Which is clearly impossible for any finite $S$. Therefore, the harmonic series diverges.
                \end{spreadlines}
        \end{proof}
\end{question}

\end{document}