\documentclass[12pt]{article}

%--------Packages--------
\usepackage{tikz}
\usepackage{tikz-cd}
\usepackage{relsize}
\usepackage{tikz-3dplot}
\usetikzlibrary{matrix,positioning,fit}
\usetikzlibrary{calc, arrows, automata, positioning}
\usetikzlibrary{shapes}
\usetikzlibrary{patterns,hobby}
\usetikzlibrary{positioning}
\tikzset{>=latex} % for LaTeX arrow head
\colorlet{myred}{red!85!black}
\colorlet{myblue}{blue!80!black}
\colorlet{mydarkred}{myred!80!black}
\colorlet{mydarkblue}{myblue!60!black}
\tikzstyle{xline}=[myblue,thick]
\def\tick#1#2{\draw[thick] (#1) ++ (#2:0.09) --++ (#2-180:0.18)}
\tikzstyle{myarr}=[myblue!50,-{Latex[length=3,width=2]}]
\def\Nr{100}
\tikzset{fontscale/.style = {font=\relsize{#1}}}
\usepackage{esvect} %graphics for volume elem
\usepackage{booktabs}
\usepackage{tabularx}
\usepackage{multirow}
\usepackage{longtable}
\usepackage{makecell}
\usepackage{amsmath,amsfonts,amsthm,amssymb,mathrsfs,bbm,mathtools,nicefrac,witharrows,commath,cancel}
\usepackage{xargs}
\usepackage{xcolor}
\usepackage{enumitem}
\usepackage[hidelinks]{hyperref}
\usepackage[noabbrev,capitalize]{cleveref}
\usepackage[italic]{derivative}
\usepackage{xparse}
\usepackage{upgreek}
\usepackage{setspace}
\usepackage[most]{tcolorbox}
\usepackage[parfill]{parskip}
\usepackage{centernot}
\usepackage{needspace}
\usepackage{varwidth}
%\usepackage{physics}

%\renewcommand{\vec}[1]{\overrightarrow{#1}}


%--------Graphics/Images--------
\usepackage{graphicx}
\usepackage{float}
\usepackage[centerlast,small,sc]{caption}
\usepackage{subcaption} % causes weird error with \setlength{\captionmargin}{20pt}

\usepackage{pgfplots}
\pgfplotsset{compat=1.6}
\usepgfplotslibrary{fillbetween}
\usetikzlibrary{patterns}
\usepackage[outline]{contour} % glow around text
\contourlength{1.0pt}


%--------Theorem Environments--------
\newtheorem{thm}{Theorem}
\newtheorem{cor}[thm]{Corollary}
\newtheorem{lem}[thm]{Lemma}
\newtheorem{fct}[thm]{Fact}
\newtheorem{ntn}[thm]{Notation}

\theoremstyle{definition}
\newtheorem{defn}[thm]{Definition}

%\numberwithin{equation}{chapter}

\newcounter{exs}

\Crefname{thm}{Theorem}{Theorems}

\tcbuselibrary{theorems}

\newtcbtheorem
  [use counter*=thm,crefname={Example}{Examples},Crefname={Example}{Examples}]%
  {ex}
  {Example}
  {%
    before skip=10pt,after skip=10pt,
    left=0.2cm,right=0.2cm,top=0cm,
    toptitle=0.2cm,bottomtitle=0cm,
    breakable,
    toprule at break=0.2cm,
    sharp corners,
    colback=blue!10,
    coltitle=black,
    colframe=blue!10,
    fonttitle=\bfseries,
    parbox=false,
    halign=justify, % use `flush left` if document is raggedright and not justified
  }% options
  {ex}% prefix

%--------Remarks-------------
\newtcbtheorem
  [use counter*=thm,crefname={Remark}{Remarks},Crefname={Remark}{Remarks}]%
  {rmk}
  {Remark}
  {%
    before skip=10pt,after skip=10pt,
    left=0.2cm,right=0.2cm,top=0cm,
    toptitle=0.2cm,bottomtitle=0cm,
    breakable,
    toprule at break=0.2cm,
    sharp corners,
    colback=green!10,
    coltitle=black,
    colframe=green!10,
    fonttitle=\bfseries,
    parbox=false,
    halign=justify,
  }% options
  {rmk}% prefix

%--------Exercises-------------
\newtcbtheorem
  [use counter*=exs,crefname={Aufgabe}{Problems},Crefname={Aufgabe}{Problems}]%
  {exc}
  {Aufgabe}
  {%
    before skip=10pt,after skip=10pt,
    left=0.2cm,right=0.2cm,top=0cm,
    toptitle=0.2cm,bottomtitle=0cm,
    breakable,
    toprule at break=0.2cm,
    sharp corners,
    colback=purple!10,
    coltitle=black,
    colframe=purple!10,
    fonttitle=\bfseries,
    parbox=false,
    halign=justify,
  }% options
  {exercise}% prefix

%--------Readings-------------
\newtcolorbox{readings}{%
    before skip=10pt,after skip=10pt,
    left=0.2cm,right=0.2cm,top=0cm,
    toptitle=0.2cm,bottomtitle=0cm,
    breakable,
    toprule at break=0.2cm,
    sharp corners,
    colback=green!30,
    coltitle=black,
    colframe=green!30,
    fonttitle=\bfseries,
    title={Readings},
    parbox=false,
    halign=justify,
}
\newtcolorbox{oreadings}{%
    before skip=10pt,after skip=10pt,
    left=0.2cm,right=0.2cm,top=0cm,
    toptitle=0.2cm,bottomtitle=0cm,
    breakable,
    toprule at break=0.2cm,
    sharp corners,
    colback=green!20,
    coltitle=black,
    colframe=green!20,
    fonttitle=\bfseries,
    title={Optional Readings},
    parbox=false,
    halign=justify,
}

%--------important theorems-------------
\newtcolorbox{thmb}{%
    before skip=10pt,after skip=10pt,
    left=0.2cm,right=0.2cm, top=0cm,
    toptitle=0cm,bottomtitle=0cm,
    breakable,
    toprule at break=0.2cm,
    sharp corners,
    colback=gray!10,
    coltitle=black,
    colframe=gray!10,
    fonttitle=\bfseries,
    title={},
    parbox=false,
    halign=justify,
}

%-------QED-at-end-of-env---

% Define a macro for changing the QED symbol at the
% end of environments. This command allows for the
% use of \qedhere to insert the QED into, e.g., 
% equations or lists. 
\newcommand{\setEnvironmentQed}[2]{
  % #1: Environment name
  % #2: QED Symbol. Must be OK in text or math mode. 
  %     Use \ensuremath, if math is desired.
  \AtBeginEnvironment{#1}{%
    \pushQED{\qed}\renewcommand{\qedsymbol}{#2}%
  }
  \AtEndEnvironment{#1}{\popQED}
}

\setEnvironmentQed{defn}{\ensuremath{\blacksquare}}
%\setEnvironmentQed{thm}{\ensuremath{\blacksquare}}

%--------Colors-------------
\definecolor{blue}{RGB}{02,106,253}
\definecolor{red}{RGB}{245,51,30}
\definecolor{green}{RGB}{96,189,69}
\definecolor{purple}{RGB}{200,0,240}
\definecolor{nice_purple}{RGB}{128, 0, 128}
\definecolor{antiquefuchsia}{rgb}{0.57, 0.36, 0.51}
\definecolor{awesome}{rgb}{1.0, 0.13, 0.32}
\definecolor{carrotorange}{rgb}{0.93, 0.57, 0.13}
\def\b{\textcolor{blue}}
\def\r{\textcolor{red}}
\def\g{\textcolor{green}}
\def\p{\textcolor{purple}}
\def\np{\textcolor{nice_purple}}
\def\af{\textcolor{antiquefuchsia}}
\def\aw{\textcolor{awesome}}
\def\co{\textcolor{carrotorange}}

%--------Margin Tags-------------
%\usepackage{marginfix}
\let\marginnote\relax
\usepackage{marginnote}
\NewDocumentCommand{\margintag}{O{0\baselineskip}m}{%
  %\checkoddpage%
  %\ifoddpage%
    {\marginnote{\footnotesize #2}[#1]}%
  %\else%
   % {\reversemarginpar\marginnote{\footnotesize #2}[#1]}%
    % {\marginnote{\footnotesize #2}[#1]}
  %\fi
  }%
% \NewDocumentCommand{\margintagt}{O{0\baselineskip}m}{\marginpar{\vspace{#1}\footnotesize #2}}
\NewDocumentCommand{\safefootnote}{om}{\footnotemark\margintag[#1]{\textsuperscript{\tiny\arabic{footnote}} \normalfont#2}}

%--------Margin Boxes-------------
\NewDocumentEnvironment{marginbox}{O{0\baselineskip}m}{\begin{marginfigure}[#1]{\textbf{#2}}\quad}{\end{marginfigure}}

%--------Equation numbers in text-------------
\makeatletter
\NewDocumentCommand{\embeq}{m}{%
  \leavevmode\hfill\refstepcounter{equation}\textup{\tagform@{\theequation}}\label{#1}%
}
\makeatother

%--------Equation numbers in algorithms-------------
\makeatletter
\NewDocumentCommand{\algeq}{m}{%
  \leavevmode\Comment*[r]{\refstepcounter{equation}\textup{\tagform@{\theequation}}\label{#1}}%
}
\makeatother

\usepackage{etoolbox}
\makeatletter
% Remove right hand margin in algorithm
\patchcmd{\@algocf@start}% <cmd>
  {-1.5em}% <search>
  {0pt}% <replace>
  {}{}% <success><failure>
\makeatother

%--------Allow page breaks in align-------------
\allowdisplaybreaks

%--------Enumerations-------------
\setlist[enumerate]{noitemsep, topsep=-6pt, leftmargin=16pt}
\setlist[itemize]{noitemsep, topsep=-6pt}

%--------Figures-------------
\usepackage{import}
\usepackage{xifthen}
\usepackage{pdfpages}
\usepackage{transparent}

\NewDocumentCommand{\incfig}{om}{%
    \IfValueTF{#1}{%
        \def\svgwidth{#1}%
    }{%
        \def\svgwidth{\columnwidth}%
    }%
    \centering\import{./figures/}{#2.pdf_tex}%
}
\newcommand{\incplt}[1]{%
  \begin{center}
    \import{./plots/output/}{#1.pgf}
  \end{center}
}

%--------Styling part-------------
\usepackage{titlesec}
\titleclass{\part}{top} % make part like a chapter
\titleformat{\part}
[display]
{\centering\normalfont}
{\vspace{3pt}\Large\smallcaps{\partname} \thepart}
{0pt}
{\vspace{1pc}\Huge\normalfont\textit}
%
\titlespacing*{\part}{0pt}{0pt}{20pt}
%

%--------Exercises-------------
\NewDocumentEnvironment{exercise}{mm}{\begin{exc}{#1}{#2}}{\par\textit{\hyperref[solution:#2]{$\triangleright$ Solution}}
\end{exc}}


\newtheorem{nexc}{}
\crefname{nexc}{Aufgabe}{Problems}
\Crefname{nexc}{Aufgabe}{Problems}

\NewDocumentCommand{\excheading}{}{\needspace{6\baselineskip}\section*{Problems}}

\NewDocumentEnvironment{nexercise}{mm}{%
  % Reserve enough space for exactly two lines (heading + one line).
  \needspace{2\baselineskip}%
  \begin{nexc}%
  \hyperref[solution:#2]{\b{\textbf{#1.}}}\label{exercise:#2}%
  % Possibly force them into the same paragraph:
  \par\nobreak
  % or \nolinebreak, if you prefer to keep them in one paragraph.
}{%
  \end{nexc}%
}

\NewDocumentCommand{\exerciseref}{mo}{\margintag{\normalfont\textbf{\Cref{exercise:#1} \IfValueT{#2}{({#2})}{}}}}
\newcommand*\circled[1]{\tikz[baseline=(char.base)]{
            \node[shape=circle,draw,inner sep=1pt] (char) {#1};}}
\NewDocumentCommand{\exerciserefmark}{mo}{\hyperref[exercise:#1]{\circled{\normalfont\textbf{?}}}\exerciseref{#1}[#2]}

%--------Solutions-------------
\NewDocumentEnvironment{tips}{m}{\paragraph{\normalfont{\textbf{Tipps zu \cref{exercise:#1}.}}}\label{solution:#1}}{}

\newcommand{\blankpage}{\newpage\hbox{}\thispagestyle{empty}\newpage}
\newcommand{\emptyparagraph}{\paragraph{}\noindent}

\newcommand{\course}{\ifthenelse{\boolean{manuscript}}{manuscript}{course}\xspace}

% Comments
\newcommand{\ak}[1]{{\bf[AK: #1]}}

%--------Basic Math--------
\NewDocumentCommand{\floor}{m}{\left\lfloor #1 \right\rfloor}
\NewDocumentCommand{\ceil}{m}{\left\lceil #1 \right\rceil}
\NewDocumentCommand{\ip}{m}{\left\langle #1 \right\rangle}

%\newcommand*{\abs}[1]{\left| #1 \right|}
\newcommand*{\card}[1]{\left| #1 \right|}
%\NewDocumentCommand{\norm}{sm}{\IfBooleanTF{#1}{\|#2\|}{\left\| #2 \right\|}}

\newcommand*{\const}{\mathrm{const}}

\newcommand*{\defeq}{\overset{.}{=}}
\newcommand*{\eqdef}{\overset{.}{=}}

\DeclareMathOperator*{\argmax}{arg\,max}
\DeclareMathOperator*{\argmin}{arg\,min}


\DeclarePairedDelimiter\parentheses{(}{)}
\DeclarePairedDelimiter\brackets{[}{]}
\DeclarePairedDelimiter\braces{\{}{\}}


%--------Sets--------
\newcommand{\R}{\mathbb{R}}
\newcommand{\Rzero}{\mathbb{R}_{\geq 0}}
\newcommand{\Nat}{\mathbb{N}}
\newcommand{\NatZ}{\mathbb{N}_0}


%--------Symbols--------
%\renewcommand{\vec}[1]{\mathbold{#1}}
%\newcommand{\mat}[1]{\mathbold{#1}}
\newcommand{\rvec}[1]{\mathbf{#1}}
%\newcommand{\set}[1]{#1}
\newcommand{\spa}[1]{\mathcal{#1}}

\newcommand{\mean}[1]{\overline{#1}}
\newcommand{\compl}[1]{\overline{#1}}
\newcommand{\old}[1]{#1^{\mathrm{old}}}
\newcommand{\opt}[1]{#1^\star}

\newcommand{\altpi}{\Pi} % \vec{\uppi}



\NewDocumentCommand{\fnv}{oo}{v\IfValueT{#2}{_{#2}}\IfValueT{#1}{^{#1}}}
\RenewDocumentCommand{\v}{somo}{\IfBooleanTF{#1}{\fnv[\star][#4]\parentheses{#3}}{\fnv[#2][#4]\parentheses{#3}}}
\NewDocumentCommand{\fnq}{oo}{q\IfValueT{#2}{_{#2}}\IfValueT{#1}{^{#1}}}
\NewDocumentCommand{\q}{sommo}{\IfBooleanTF{#1}{\fnq[\star][#5]\parentheses{#3,#4}}{\fnq[#2][#5]\parentheses{#3,#4}}}
\NewDocumentCommand{\fnV}{oo}{V\IfValueT{#2}{_{#2}}\IfValueT{#1}{^{#1}}}
\NewDocumentCommand{\V}{somo}{\IfBooleanTF{#1}{\fnV[\star][#4]\parentheses{#3}}{\fnV[#2][#4]\parentheses{#3}}}
\NewDocumentCommand{\fnQ}{oo}{Q\IfValueT{#2}{_{#2}}\IfValueT{#1}{^{#1}}}
\NewDocumentCommand{\Q}{sommo}{\IfBooleanTF{#1}{\fnQ[\star][#5]\parentheses{#3,#4}}{\fnQ[#2][#5]\parentheses{#3,#4}}}
\NewDocumentCommand{\fna}{oo}{a\IfValueT{#2}{_{#2}}\IfValueT{#1}{^{#1}}}
\RenewDocumentCommand{\a}{sommo}{\IfBooleanTF{#1}{\fna[\star][#5]\parentheses{#3,#4}}{\fna[#2][#5]\parentheses{#3,#4}}}
\NewDocumentCommand{\fnA}{oo}{A\IfValueT{#2}{_{#2}}\IfValueT{#1}{^{#1}}}
\NewDocumentCommand{\fnAhat}{oo}{\hat{A}\IfValueT{#2}{_{#2}}\IfValueT{#1}{^{#1}}}
\NewDocumentCommand{\A}{sommo}{\IfBooleanTF{#1}{\fnA[\star][#5]\parentheses{#3,#4}}{\fnA[#2][#5]\parentheses{#3,#4}}}
\NewDocumentCommand{\fnj}{o}{J\IfValueT{#1}{_{#1}}}
\RenewDocumentCommand{\j}{mo}{\fnj[#2]\parentheses{#1}}
\NewDocumentCommand{\fnJ}{o}{\widehat{J}\IfValueT{#1}{_{#1}}}
\NewDocumentCommand{\J}{mo}{\fnJ[#2]\parentheses{#1}}

\NewDocumentCommand{\pset}{m}{\mathcal{P}\parentheses*{#1}}

\NewDocumentCommand{\pf}{mm}{{#1}_\sharp #2}

\NewDocumentCommand{\grad}{e_}{\boldsymbol{\nabla}\IfValueT{#1}{_{\!\!#1}\,}}
\NewDocumentCommand{\jac}{}{\mD}
\NewDocumentCommand{\hes}{}{\mH}
\NewDocumentCommand{\dive}{}{\grad\cdot}
\NewDocumentCommand{\lapl}{}{\Delta}

\NewDocumentCommand{\BigO}{m}{O\parentheses*{#1}}
\NewDocumentCommand{\BigOTil}{m}{\widetilde{O}\parentheses*{#1}}

\NewDocumentCommand{\transpose}{m}{#1^\top}
\NewDocumentCommand{\inv}{m}{#1^{-1}}
\RenewDocumentCommand{\det}{m}{\mathrm{det}\parentheses*{#1}}
\NewDocumentCommand{\tr}{m}{\mathrm{tr}\parentheses*{#1}}
\NewDocumentCommand{\diag}{om}{\mathrm{diag}\IfValueT{#1}{_{#1}}{}\braces{#2}}
\NewDocumentCommand{\msqrt}{m}{#1^{\nicefrac{1}{2}}}
\NewDocumentCommand{\vecop}{m}{\mathrm{vec}\brackets{#1}}

%--------Common vectors/matrices/sets--------
\newcommand{\vzero}{\vec{0}}
\newcommand{\vone}{\vec{1}}
\newcommand{\va}{\vec{a}}
\newcommand{\vap}{\vec{a'}}
\newcommand{\vas}{\vec{\opt{a}}}
\newcommand{\vb}{\vec{b}}
\newcommand{\vc}{\vec{c}}
\newcommand{\vd}{\vec{d}}
\newcommand{\ve}{\vec{e}}
\newcommand{\vf}{\vec{f}}
\newcommand{\vfhat}{\vec{\hat{f}}}
\newcommand{\vg}{\vec{g}}
\newcommand{\vh}{\vec{h}}
\newcommand{\vk}{\vec{k}}
\newcommand{\vm}{\vec{m}}
\newcommand{\vp}{\vec{p}}
\newcommand{\vq}{\vec{q}}
\newcommand{\vr}{\vec{r}}
\newcommand{\vs}{\vec{s}}
\newcommand{\vt}{\vec{t}}
\newcommand{\vu}{\vec{u}}
\newcommand{\vve}{\vec{v}}
\newcommand{\vvp}{\vec{v'}}
\newcommand{\vvs}{\vec{\opt{v}}}
\newcommand{\vw}{\vec{w}}
\newcommand{\vwhat}{\vec{\hat{w}}}
\newcommand{\vx}{\vec{x}}
\newcommand{\vxp}{\vec{x'}}
\newcommand{\vxs}{\vec{\opt{x}}}
\newcommand{\vy}{\vec{y}}
\newcommand{\vyp}{\vec{y'}}
\newcommand{\vz}{\vec{z}}
\newcommand{\valpha}{\vec{\alpha}}
\newcommand{\valphahat}{\vec{\hat{\alpha}}}
\newcommand{\vdelta}{\vec{\delta}}
\newcommand{\vDelta}{\vec{\Delta}}
\newcommand{\vepsilon}{\vec{\epsilon}}
\newcommand{\vvarepsilon}{\vec{\varepsilon}}
\newcommand{\veta}{\vec{\eta}}
\newcommand{\vlambda}{\vec{\lambda}}
\newcommand{\vmu}{\vec{\mu}}
\newcommand{\vmuhat}{\vec{\hat{\mu}}}
\newcommand{\vmup}{\vec{\mu'}}
\newcommand{\vnu}{\vec{\nu}}
\newcommand{\vomega}{\vec{\omega}}
\newcommand{\vphi}{\vec{\phi}}
\newcommand{\vpi}{\vec{\pi}}
\newcommand{\vpsi}{\vec{\psi}}
\newcommand{\vvarphi}{\vec{\varphi}}
\newcommand{\vvarphihat}{\vec{\hat{\varphi}}}
\newcommand{\vtheta}{\vec{\theta}}
\newcommand{\vthetahat}{\vec{\hat{\theta}}}
\newcommand{\vxi}{\vec{\xi}}
\newcommand{\mzero}{\mat{0}}
\newcommand{\mA}{\mat{A}}
\newcommand{\mB}{\mat{B}}
\newcommand{\mBs}{\mat{\opt{B}}}
\newcommand{\mC}{\mat{C}}
\newcommand{\mD}{\mat{D}}
\newcommand{\mF}{\mat{F}}
\newcommand{\mH}{\mat{H}}
\newcommand{\mI}{\mat{I}}
\newcommand{\mK}{\mat{K}}
\newcommand{\mL}{\mat{L}}
\newcommand{\mCalL}{\mat{\mathcal{L}}}
\newcommand{\mM}{\mat{M}}
\newcommand{\mP}{\mat{P}}
\newcommand{\mQ}{\mat{Q}}
\newcommand{\mS}{\mat{S}}
\newcommand{\mT}{\mat{T}}
\newcommand{\mU}{\mat{U}}
\newcommand{\mV}{\mat{V}}
\newcommand{\mW}{\mat{W}}
\newcommand{\mX}{\mat{X}}
\newcommand{\mLambda}{\mat{\Lambda}}
\newcommand{\mPhi}{\mat{\Phi}}
\newcommand{\mPi}{\mat{\Pi}}
\newcommand{\mSigma}{\mat{\Sigma}}
\newcommand{\mSigmap}{\mat{\Sigma'}}
\newcommand{\rG}{\rvec{G}}
\newcommand{\rQ}{\rvec{Q}}
\newcommand{\rU}{\rvec{U}}
\newcommand{\rV}{\rvec{V}}
\newcommand{\rW}{\rvec{W}}
\newcommand{\rX}{\rvec{X}}
\newcommand{\rXp}{\rvec{X'}}
\newcommand{\rY}{\rvec{Y}}
\newcommand{\rZ}{\rvec{Z}}
\newcommand{\sA}{\set{A}}
\newcommand{\sB}{\set{B}}
\newcommand{\sC}{\set{C}}
\newcommand{\sD}{\set{D}}
\newcommand{\sI}{\set{I}}
\newcommand{\sM}{\set{M}}
\newcommand{\sS}{\set{S}}
\newcommand{\sU}{\set{U}}
\newcommand{\sX}{\set{X}}
\newcommand{\sY}{\set{Y}}
\newcommand{\sZ}{\set{Z}}
\newcommand{\spA}{\spa{A}}
\newcommand{\spB}{\spa{B}}
\newcommand{\spC}{\spa{C}}
\newcommand{\spD}{\spa{D}}
\newcommand{\spF}{\spa{F}}
\newcommand{\spH}{\spa{H}}
\newcommand{\spL}{\spa{L}}
\newcommand{\spM}{\spa{M}}
\newcommand{\spO}{\spa{O}}
\newcommand{\spP}{\spa{P}}
\newcommand{\spQ}{\spa{Q}}
\newcommand{\spT}{\spa{T}}
\newcommand{\spW}{\spa{W}}
\newcommand{\spX}{\spa{X}}
\newcommand{\spY}{\spa{Y}}
\newcommand{\spZ}{\spa{Z}}
\newcommand{\fs}{\opt{f}}
\newcommand{\ps}{\opt{p}}
\newcommand{\qs}{\opt{q}}
\newcommand{\xs}{\opt{x}}
\newcommand{\ys}{\opt{y}}
\newcommand{\Bs}{\opt{B}}
\newcommand{\Qs}{\opt{Q}}
\newcommand{\sSs}{\opt{\sS}}
\newcommand{\hQs}{\opt{\hat{Q}}}
\newcommand{\Vs}{\opt{V}}
\newcommand{\pis}{\opt{\pi}}

\newcommand{\vF}{\rvec{F}}
\newcommand{\vS}{\rvec{S}}
\newcommand{\vT}{\rvec{T}}


\renewcommand\qedsymbol{$\blacksquare$}


\newcommand{\dx}{\mathrm{d}x}
\newcommand{\ddx}{\frac{\mathrm{d}}{\mathrm{d}x}}
\newcommand{\dt}{\mathrm{d}t}
\newcommand{\du}{\mathrm{d}u}
\newcommand{\dve}{\mathrm{d}v}
\newcommand{\dw}{\mathrm{d}w}
\newcommand{\dy}{\mathrm{d}y}
\newcommand{\dz}{\mathrm{d}z}
\newcommand{\dF}{\mathrm{d}F}
\newcommand{\dV}{\mathrm{d}V}
\newcommand{\dr}{\mathrm{d}r}
\newcommand{\dtheta}{\mathrm{d}\theta}
\newcommand{\drho}{\mathrm{d}\rho}
\newcommand{\dphi}{\mathrm{d}\varphi}

\newcommand{\Rn}{\mathbb{R}^n}
\newcommand{\Rm}{\mathbb{R}^m}
\newcommand{\Rk}{\mathbb{R}^k}
\newcommand{\und}{\text{ und }}
\newcommand{\oder}{\text{ oder }}
\newcommand{\bydef}{\underset{def.}{=}}
\newcommand{\BH}{\underset{\textrm{B-H}}{=}}

\newcommand{\Follows}{\Longrightarrow\ }
\newcommand{\sameas}{\Longleftrightarrow}
\newcommandx{\Laplace}[2][1=f(t), 2=s]{\mathscr{L}\{#1\}(#2)}
\newcommandx{\LaplaceInv}[2][1=F(s), 2=t]{\mathscr{L}^{-1}\{#1\}(#2)}
\DeclareMathOperator{\arccosh}{Arcosh}
\DeclareMathOperator{\arcsinh}{Arsinh}
\DeclareMathOperator{\arctanh}{Artanh}
\DeclareMathOperator{\arcsech}{arcsech}
\DeclareMathOperator{\arccsch}{arcCsch}
\DeclareMathOperator{\arccoth}{arcCoth} 

\def\doubleunderline#1{\underline{\underline{#1}}}
\def\ez{\begin{flushright}\underline{ez.}\end{flushright}}
%\[
%   \Laplace[\cos(x)]=\int_{t=0}^{\infty}f(t)e^{-st}dt
%\]

\newcommand{\Z}{\mathbb{Z}}
\newcommand{\N}{\mathbb{N}}

\newcommand{\C}{\mathbb{C}}

\newcommand{\inttext}{\shortintertext}

% The following example defines \colorboxed as wrapper around amsmath's \boxed to set the frame color. It uses package xcolor for the color support to save the current color . before changing the color for the frame. Inside the box, the previous saved color is restored. This avoids a white background of \fcolorbox, since there is no "transparent" color.

% The macro also supports an optional argument for specifying the color model.

% Definition of \boxed in amsmath.sty:
% \newcommand{\boxed}[1]{\fbox{\m@th$\displaystyle#1$}}


% Syntax: \colorboxed[<color model>]{<color specification>}{<math formula>}
\newcommand*{\colorboxed}{}
\def\colorboxed#1#{%
  \colorboxedAux{#1}%
}
\newcommand*{\colorboxedAux}[3]{%
  % #1: optional argument for color model
  % #2: color specification
  % #3: formula
  \begingroup
    \colorlet{cb@saved}{.}%
    \color#1{#2}%
    \boxed{%
      \color{cb@saved}%
      #3%
    }%
  \endgroup
}

%
\usepackage[utf8]{inputenc}
\usepackage[ngerman]{babel}
\usepackage[a4paper, top=1in, bottom=1.3in, rmargin=1.5in, marginparwidth=80pt]{geometry}


\begin{document}
\title{\vspace*{-2.5em}Problem Set 1, Summary \& Tips}
\author{Vikram R. Damani\\
        Analysis~II}

\maketitle

\section{Theorie}

\begin{defn}{\b{[Funktionen in zwei Variablen]}}
        Eine Funktion $f: D \to \R, \quad D \subseteq \R^2$ mit defnsberiech $D(f)=D$ und Wertebereich $W(f)=\{z\in\R\,|\,(x,y)\in D(f): f(x,y)=z\}$ ist eine Abbildung, die jedem $(x, y) \in D$ genau ein $f(x, y) \in \R$ zuordnet.
        %\safefootnote{what hasd asdfqwse asg aerg werwg  we weg qerg}

        \textbf{Beispiel:} $f(x,y)=ax+by+c$ ist eine affin lineare Funktion in zwei variablen.
\end{defn}

\begin{defn}{\b{[Graph]}} Der Graph von $f$ is die Menge
        \begin{align}
                \Gamma(f)=\{(x,y,z)\in\R^3\,|\,(x,y)\in D(f):z=f(x,y)\}
        \end{align}

        \textbf{Beispiel:} Darstellung des Graphen $\Gamma(f)$ von $f(x,y)=x \exp(-x^2-y^2)$
        \begin{figure}[htbp!]
                \centering
                \begin{tikzpicture}
                        \begin{axis}[
                                        xlabel=$x$, ylabel=$y$,
                                        small,
                                ]
                                \addplot3 [
                                        surf,
                                        domain=-2:2,
                                        domain y=-1.3:1.3,
                                ] {exp(-x^2-y^2)*x};
                        \end{axis}
                \end{tikzpicture}
        \end{figure}
\end{defn}

\begin{defn}{\b{[Niveaulinien]}}
        Für ein fixes $C\in\R$ ist
        \begin{align}
                N_{f,C}=\{(x,y)\in\R^2\,|\,f(x,y)=C\}
        \end{align}
        die Niveaulinie von $f$ zum Niveau $C$.\safefootnote{This is a neat little Geogebra applet: \url{https://www.geogebra.org/m/nuR3n88b}}

        \textbf{Beispiel:} Höhenlinien auf einer Karte.
\end{defn}

Analog zu Ableitungen in Funktionen einer Variable lassen sich hier (partielle)
Ableitungen (d.h Ableitungen in jeweils x-Koordinaten- oder
y-Koordinaten-Richtung) definieren.

\begin{thmb}{\emph{\np{[Partielle Ableitungen].}}}
        Die Partielle Ableitung einer Funktion mit mehreren Argumenten ist die Ableitung dieser Funktion nach \emph{einer} Variable. Die \r{Partielle Ableitung von $f(x,y)$ nach $x$ an der Stelle $(x_0,y_0)$} ist gegeben durch
        \begin{align}
                \pdv{f}{x}(x_0,y_0)=\lim_{\r{\Delta x\to0}}\frac{f(\r{x_0+\Delta x},\b{y_0})-f(\r{x_0},\b{y_0
                        })}{\r{\Delta x}}
        \end{align}
        Hierbei wird die $\b y$-Koordinate konstant gehalten.

        Die Partielle Ableitung in \r{$y$} wird analog berechnet
        \begin{align}
                \pdv{f}{y}(x_0,y_0)=\lim_{\r{\Delta y\to0}}\frac{f(x_0,\r{y_0+\Delta y})-f(x_0,\r{y_0})}{\r{\Delta y}}
        \end{align}
\end{thmb}

Die Partiellen Ableitungen $\pdv{f}{x}$ und $\pdv{f}{y}$ sind wieder Funktionen
in $x$ und $y$.

\begin{rmk}{Berechnung}{Berechnung}
        Jeder, der in einer Dimension ableiten kann, kann auch Funktionen in mehreren Variablen partiell ableiten. Eine partielle Ableitung nach einer Variablen ist per Definition genau analog zu einer Ableitung in einer Dimension, wobei alle anderen Variablen konstant gehalten werden.
\end{rmk}

\begin{rmk}{Notation}{Notation}
        Je nach Autor wird eine partielle Ableitung auf verschiedene Weise geschrieben. Hier eine kurze Liste
        \begin{align*}
                \pdv{f}{x} = \partial_x f = D_x f = f_x
        \end{align*}
\end{rmk}\vspace*{1em}

\begin{defn}{\b{[Gradient]}}
        Der Gradient  von $f$ ist der Vektor
        \begin{align}
                grad(f)=\nabla f = \begin{pmatrix}
                                           f_x \\
                                           f_y
                                   \end{pmatrix}
        \end{align}
\end{defn}

Da die partielle Ableitung einer Funktion mit mehreren Argumenten wiederum eine
Funktion in diesen Argumenten ist, kann man diese auch auf Diff'barkeit
untersuchen. Die zweite Ableitung von $f$ nach $x$ wird wie folgt gebildet
\begin{align}
        (f_{x})_x\defeq f_{xx}=\pdv{f_x}{x}=\frac{\partial}{\partial x}\left(\pdv{f}{x}\right)
\end{align}

\begin{defn}{\b{[Höhere Ableitungen]}}
        Allgemeiner erhält man höhere Ableitungen von $f$ nach $x_i$ und $x_j$ wie folgt
        \begin{align}
                \frac{\partial^2 f}{\partial x_i\partial x_j}=\frac{\partial}{\partial x_i}\left(\pdv{f}{x_j}\right)
        \end{align}

        Das ergibt vier mögliche Kombinationen für Funktionen in zwei Variablen:
        \begin{enumerate}[topsep=0pt,itemsep=0.15em]
                \item $f_{xx}$
                \item $(f_{x})_y$
                \item $f_{yy}$
                \item $(f_y)_x$
        \end{enumerate}
\end{defn}

\begin{defn}{\b{[Stetigkeit]}}
        Eine Funktion in zwei Variablen ist in einer Umgebung von $(x_0,y_0)$ stetig, falls für jede Folge $\{(x_n,y_n)\}_{n=0}^{\infty}$ gilt
        \begin{align}
                x_n\to x y_n\to y \implies f(x_n,y_n)\to f(x,y)
        \end{align}
\end{defn}

\begin{thmb}{\emph{\np{[Satz von Schwartz].}}}
        Sei $f:A\to\R$ und $(x_0,y_0)\in A\subseteq\R^2$. Wenn $f_{xy}$ und $f_{yx}$ in einer Umgebung um $(x_0,y_0)$ stetig sind, dann gilt
        \begin{align}
                f_{xy}(x_0,y_0)=f_{yx}(x_0,y_0)
        \end{align}
\end{thmb}

\begin{rmk}{Reihenfolge der Ableitungen}{Reihenfolge}
        Eine unmittelbare Folge vom Satz von Schwartz ist, dass meistens die Reihenfolge der Ableitungen egal ist!
\end{rmk}

\begin{figure}[htbp!]
        \centering
        \begin{tikzpicture}[
                        %> = stealth, % arrow head style
                        %shorten > = 1pt, % don't touch arrow head to node
                        auto, node distance = 3cm, semithick ]% distance between nodes% line style

                \node[] (y1) at (-2,-2) {$\r{\dfrac{\partial f}{\partial x}}\overset{?}{=}\psi$};
                \node[] (y2) at (2,-2) {$\r{\dfrac{\partial f}{\partial y}}\overset{?}{=}\varphi$};
                \node[] (x1) at (0,0) {$\r f$};
                \node[] (z1) at (0,-4) {$\dfrac{\partial^2 f}{\partial x\partial y}$};

                \path[->,red] (x1) edge  node[above left] {\r{(?)}} (y1);
                \path[->,red] (x1) edge  node[above right] {\r{(?)}} (y2);
                \path[->] (y1) edge  node[above left] {} (z1);
                \path[->] (y2) edge  node[above right] {} (z1);

        \end{tikzpicture}
\end{figure}

Anhand bekannten partiellen Ableitungen lässt sich die ursprüngliche Funktion
unter folgender Bedingung rekonstruieren

\begin{thmb}{\emph{\np{[Integrabilitätsbedingung (IB)].}}}
        \begin{align}
                \varphi_y\equiv\psi_x
        \end{align}
\end{thmb}\vspace*{1em}

\begin{fct}
        Erfüllen zwei stetig diff'bare funktionen die Integrabilitätsbedingung $\varphi_y\equiv\psi_x$ in einem achsenparallelen Rechteck, dann gibt es eine Funktion $f$ mit
        \begin{align}
                f_y=\varphi\text{ und }f_x=\psi
        \end{align}
\end{fct}

\begin{defn}{\b{[Lineare Ersatzfunktion]}}
        Die lineare Ersatzfunktion an der Stelle $(x_0,y_0)$ einer Funktion $f:\R^2\to\R$ in zwei Variablen ist gegeben durch
        \begin{align}
                t_f(x,y)=\underbrace{f_x(x_0,y_0)(x-x_0)}_{\text{Spannt $t_f$ in $x$-Richtung auf}}+\underbrace{f_y(x_0,y_0)(y-y_0)}_{\text{Spannt $t_f$ in $y$-Richtung auf}}+\underbrace{f(x_0,y_0)}_{\text{Stützpunkt}}
        \end{align}
\end{defn}

Wie bei Funktionen in einer Variablen können wir auch für Funktionen in höheren
Dimensionen ein lokales Koordinatensystem in $(x_0,y_0f(x_0,y_0))$ definieren,
in dem die lineare Ersatzfunktion wie folgt neu definiert ist\vspace*{0.3em}

\begin{thmb}{\np{\textit{[Totales Differential].}}}
        \begin{align}
                df = f_x(x_0,y_0)\,dx+f_y(x_0,y_0)\,dy
        \end{align}
\end{thmb}\vspace*{1em}

\begin{fct}
        Sei $\Delta f=f(x+\Delta x,y+\Delta y)-f(x_0,y_0)$. Für kleine $\Delta x = dx$ und $\Delta y = dy$ ist die Approximation $\Delta f \approx df$ gut.
\end{fct}

Wir können also den genauen absoluten Fehler durch Linearisierung
näherungs\-weise bestimmen, wenn der Messfehler der Variablen klein ist.

\section{Tipps}
\setcounter{nexc}{1}

\begin{nexercise}{Aufgabe 2}{1}
        Wir betrachten die folgende Liste von Funktionen

        \begin{varwidth}{\textwidth}
                \begin{enumerate}[label=(\alph*)]
                        \item $f(x,y)=\sin(x-y)(-x^2+y^2)$
                        \item $g(x,y)=(x-y)e^{-x^2+y^2}$
                        \item $h(x,y)=\cos(x)+\cos(y)$
                        \item $l(x,y)=(x^2-y^2)xy.$
                \end{enumerate}
        \end{varwidth}
\end{nexercise}

\begin{tips}{1}
        Hier ist es wichtig, sowohl die Symmetrien der Funktionen als auch die Periodizität in Richtung der $x$- und $y$-Achsen zu betrachten. Die Funktion $l$ lässt sich faktorisieren, was Berechnung der Niveaulinen evtl.\ vereinfacht.
\end{tips}

\end{document}

\begin{defn}{Funktionen in mehreren Variablen}
        Eine Funktion $f: D \to \R$ mit $D \subseteq \R^n$ ist eine Abbildung, die jedem $x \in D$ genau ein $f(x) \in \R$ zuordnet.
\end{defn}

\footnote{Plot contour.png - Wikimedia Commons. (n.d.). Retrieved February 16, 2025, from https://commons.wikimedia.org/wiki/File:Plot_contour.png}