\documentclass[12pt]{article}

%--------Packages--------
\usepackage{tikz}
\usepackage{tikz-cd}
\usepackage{relsize}
\usepackage{tikz-3dplot}
\usetikzlibrary{matrix,positioning,fit}
\usetikzlibrary{calc, arrows, automata, positioning}
\usetikzlibrary{shapes}
\usetikzlibrary{patterns,hobby}
\usetikzlibrary{positioning}
\tikzset{>=latex} % for LaTeX arrow head
\colorlet{myred}{red!85!black}
\colorlet{myblue}{blue!80!black}
\colorlet{mydarkred}{myred!80!black}
\colorlet{mydarkblue}{myblue!60!black}
\tikzstyle{xline}=[myblue,thick]
\def\tick#1#2{\draw[thick] (#1) ++ (#2:0.09) --++ (#2-180:0.18)}
\tikzstyle{myarr}=[myblue!50,-{Latex[length=3,width=2]}]
\def\Nr{100}
\tikzset{fontscale/.style = {font=\relsize{#1}}}
\usepackage{esvect} %graphics for volume elem
\usepackage{booktabs}
\usepackage{tabularx}
\usepackage{multirow}
\usepackage{longtable}
\usepackage{makecell}
\usepackage{amsmath,amsfonts,amsthm,amssymb,mathrsfs,bbm,mathtools,nicefrac,witharrows,commath,cancel}
\usepackage{xargs}
\usepackage{xcolor}
\usepackage{enumitem}
\usepackage[hidelinks]{hyperref}
\usepackage[noabbrev,capitalize]{cleveref}
\usepackage[italic]{derivative}
\usepackage{xparse}
\usepackage{upgreek}
\usepackage{setspace}
\usepackage[most]{tcolorbox}
\usepackage[parfill]{parskip}
\usepackage{centernot}
\usepackage{needspace}
\usepackage{varwidth}
%\usepackage{physics}

%\renewcommand{\vec}[1]{\overrightarrow{#1}}


%--------Graphics/Images--------
\usepackage{graphicx}
\usepackage{float}
\usepackage[centerlast,small,sc]{caption}
\usepackage{subcaption} % causes weird error with \setlength{\captionmargin}{20pt}

\usepackage{pgfplots}
\pgfplotsset{compat=1.6}
\usepgfplotslibrary{fillbetween}
\usetikzlibrary{patterns}
\usepackage[outline]{contour} % glow around text
\contourlength{1.0pt}


%--------Theorem Environments--------
\newtheorem{thm}{Theorem}
\newtheorem{cor}[thm]{Corollary}
\newtheorem{lem}[thm]{Lemma}
\newtheorem{fct}[thm]{Fact}
\newtheorem{ntn}[thm]{Notation}

\theoremstyle{definition}
\newtheorem{defn}[thm]{Definition}

%\numberwithin{equation}{chapter}

\newcounter{exs}

\Crefname{thm}{Theorem}{Theorems}

\tcbuselibrary{theorems}

\newtcbtheorem
  [use counter*=thm,crefname={Example}{Examples},Crefname={Example}{Examples}]%
  {ex}
  {Example}
  {%
    before skip=10pt,after skip=10pt,
    left=0.2cm,right=0.2cm,top=0cm,
    toptitle=0.2cm,bottomtitle=0cm,
    breakable,
    toprule at break=0.2cm,
    sharp corners,
    colback=blue!10,
    coltitle=black,
    colframe=blue!10,
    fonttitle=\bfseries,
    parbox=false,
    halign=justify, % use `flush left` if document is raggedright and not justified
  }% options
  {ex}% prefix

%--------Remarks-------------
\newtcbtheorem
  [use counter*=thm,crefname={Remark}{Remarks},Crefname={Remark}{Remarks}]%
  {rmk}
  {Remark}
  {%
    before skip=10pt,after skip=10pt,
    left=0.2cm,right=0.2cm,top=0cm,
    toptitle=0.2cm,bottomtitle=0cm,
    breakable,
    toprule at break=0.2cm,
    sharp corners,
    colback=green!10,
    coltitle=black,
    colframe=green!10,
    fonttitle=\bfseries,
    parbox=false,
    halign=justify,
  }% options
  {rmk}% prefix

%--------Exercises-------------
\newtcbtheorem
  [use counter*=exs,crefname={Aufgabe}{Problems},Crefname={Aufgabe}{Problems}]%
  {exc}
  {Aufgabe}
  {%
    before skip=10pt,after skip=10pt,
    left=0.2cm,right=0.2cm,top=0cm,
    toptitle=0.2cm,bottomtitle=0cm,
    breakable,
    toprule at break=0.2cm,
    sharp corners,
    colback=purple!10,
    coltitle=black,
    colframe=purple!10,
    fonttitle=\bfseries,
    parbox=false,
    halign=justify,
  }% options
  {exercise}% prefix

%--------Readings-------------
\newtcolorbox{readings}{%
    before skip=10pt,after skip=10pt,
    left=0.2cm,right=0.2cm,top=0cm,
    toptitle=0.2cm,bottomtitle=0cm,
    breakable,
    toprule at break=0.2cm,
    sharp corners,
    colback=green!30,
    coltitle=black,
    colframe=green!30,
    fonttitle=\bfseries,
    title={Readings},
    parbox=false,
    halign=justify,
}
\newtcolorbox{oreadings}{%
    before skip=10pt,after skip=10pt,
    left=0.2cm,right=0.2cm,top=0cm,
    toptitle=0.2cm,bottomtitle=0cm,
    breakable,
    toprule at break=0.2cm,
    sharp corners,
    colback=green!20,
    coltitle=black,
    colframe=green!20,
    fonttitle=\bfseries,
    title={Optional Readings},
    parbox=false,
    halign=justify,
}

%--------important theorems-------------
\newtcolorbox{thmb}{%
    before skip=10pt,after skip=10pt,
    left=0.2cm,right=0.2cm, top=0cm,
    toptitle=0cm,bottomtitle=0cm,
    breakable,
    toprule at break=0.2cm,
    sharp corners,
    colback=gray!10,
    coltitle=black,
    colframe=gray!10,
    fonttitle=\bfseries,
    title={},
    parbox=false,
    halign=justify,
}

%-------QED-at-end-of-env---

% Define a macro for changing the QED symbol at the
% end of environments. This command allows for the
% use of \qedhere to insert the QED into, e.g., 
% equations or lists. 
\newcommand{\setEnvironmentQed}[2]{
  % #1: Environment name
  % #2: QED Symbol. Must be OK in text or math mode. 
  %     Use \ensuremath, if math is desired.
  \AtBeginEnvironment{#1}{%
    \pushQED{\qed}\renewcommand{\qedsymbol}{#2}%
  }
  \AtEndEnvironment{#1}{\popQED}
}

\setEnvironmentQed{defn}{\ensuremath{\blacksquare}}
%\setEnvironmentQed{thm}{\ensuremath{\blacksquare}}

%--------Colors-------------
\definecolor{blue}{RGB}{02,106,253}
\definecolor{red}{RGB}{245,51,30}
\definecolor{green}{RGB}{96,189,69}
\definecolor{purple}{RGB}{200,0,240}
\definecolor{nice_purple}{RGB}{128, 0, 128}
\definecolor{antiquefuchsia}{rgb}{0.57, 0.36, 0.51}
\definecolor{awesome}{rgb}{1.0, 0.13, 0.32}
\definecolor{carrotorange}{rgb}{0.93, 0.57, 0.13}
\def\b{\textcolor{blue}}
\def\r{\textcolor{red}}
\def\g{\textcolor{green}}
\def\p{\textcolor{purple}}
\def\np{\textcolor{nice_purple}}
\def\af{\textcolor{antiquefuchsia}}
\def\aw{\textcolor{awesome}}
\def\co{\textcolor{carrotorange}}

%--------Margin Tags-------------
%\usepackage{marginfix}
\let\marginnote\relax
\usepackage{marginnote}
\NewDocumentCommand{\margintag}{O{0\baselineskip}m}{%
  %\checkoddpage%
  %\ifoddpage%
    {\marginnote{\footnotesize #2}[#1]}%
  %\else%
   % {\reversemarginpar\marginnote{\footnotesize #2}[#1]}%
    % {\marginnote{\footnotesize #2}[#1]}
  %\fi
  }%
% \NewDocumentCommand{\margintagt}{O{0\baselineskip}m}{\marginpar{\vspace{#1}\footnotesize #2}}
\NewDocumentCommand{\safefootnote}{om}{\footnotemark\margintag[#1]{\textsuperscript{\tiny\arabic{footnote}} \normalfont#2}}

%--------Margin Boxes-------------
\NewDocumentEnvironment{marginbox}{O{0\baselineskip}m}{\begin{marginfigure}[#1]{\textbf{#2}}\quad}{\end{marginfigure}}

%--------Equation numbers in text-------------
\makeatletter
\NewDocumentCommand{\embeq}{m}{%
  \leavevmode\hfill\refstepcounter{equation}\textup{\tagform@{\theequation}}\label{#1}%
}
\makeatother

%--------Equation numbers in algorithms-------------
\makeatletter
\NewDocumentCommand{\algeq}{m}{%
  \leavevmode\Comment*[r]{\refstepcounter{equation}\textup{\tagform@{\theequation}}\label{#1}}%
}
\makeatother

\usepackage{etoolbox}
\makeatletter
% Remove right hand margin in algorithm
\patchcmd{\@algocf@start}% <cmd>
  {-1.5em}% <search>
  {0pt}% <replace>
  {}{}% <success><failure>
\makeatother

%--------Allow page breaks in align-------------
\allowdisplaybreaks

%--------Enumerations-------------
\setlist[enumerate]{noitemsep, topsep=-6pt, leftmargin=16pt}
\setlist[itemize]{noitemsep, topsep=-6pt}

%--------Figures-------------
\usepackage{import}
\usepackage{xifthen}
\usepackage{pdfpages}
\usepackage{transparent}

\NewDocumentCommand{\incfig}{om}{%
    \IfValueTF{#1}{%
        \def\svgwidth{#1}%
    }{%
        \def\svgwidth{\columnwidth}%
    }%
    \centering\import{./figures/}{#2.pdf_tex}%
}
\newcommand{\incplt}[1]{%
  \begin{center}
    \import{./plots/output/}{#1.pgf}
  \end{center}
}

%--------Styling part-------------
\usepackage{titlesec}
\titleclass{\part}{top} % make part like a chapter
\titleformat{\part}
[display]
{\centering\normalfont}
{\vspace{3pt}\Large\smallcaps{\partname} \thepart}
{0pt}
{\vspace{1pc}\Huge\normalfont\textit}
%
\titlespacing*{\part}{0pt}{0pt}{20pt}
%

%--------Exercises-------------
\NewDocumentEnvironment{exercise}{mm}{\begin{exc}{#1}{#2}}{\par\textit{\hyperref[solution:#2]{$\triangleright$ Solution}}
\end{exc}}


\newtheorem{nexc}{}
\crefname{nexc}{Aufgabe}{Problems}
\Crefname{nexc}{Aufgabe}{Problems}

\NewDocumentCommand{\excheading}{}{\needspace{6\baselineskip}\section*{Problems}}

\NewDocumentEnvironment{nexercise}{mm}{%
  % Reserve enough space for exactly two lines (heading + one line).
  \needspace{2\baselineskip}%
  \begin{nexc}%
  \hyperref[solution:#2]{\b{\textbf{#1.}}}\label{exercise:#2}%
  % Possibly force them into the same paragraph:
  \par\nobreak
  % or \nolinebreak, if you prefer to keep them in one paragraph.
}{%
  \end{nexc}%
}

\NewDocumentCommand{\exerciseref}{mo}{\margintag{\normalfont\textbf{\Cref{exercise:#1} \IfValueT{#2}{({#2})}{}}}}
\newcommand*\circled[1]{\tikz[baseline=(char.base)]{
            \node[shape=circle,draw,inner sep=1pt] (char) {#1};}}
\NewDocumentCommand{\exerciserefmark}{mo}{\hyperref[exercise:#1]{\circled{\normalfont\textbf{?}}}\exerciseref{#1}[#2]}

%--------Solutions-------------
\NewDocumentEnvironment{tips}{m}{\paragraph{\normalfont{\textbf{Tipps zu \cref{exercise:#1}.}}}\label{solution:#1}}{}

\newcommand{\blankpage}{\newpage\hbox{}\thispagestyle{empty}\newpage}
\newcommand{\emptyparagraph}{\paragraph{}\noindent}

\newcommand{\course}{\ifthenelse{\boolean{manuscript}}{manuscript}{course}\xspace}

% Comments
\newcommand{\ak}[1]{{\bf[AK: #1]}}

%--------Basic Math--------
\NewDocumentCommand{\floor}{m}{\left\lfloor #1 \right\rfloor}
\NewDocumentCommand{\ceil}{m}{\left\lceil #1 \right\rceil}
\NewDocumentCommand{\ip}{m}{\left\langle #1 \right\rangle}

%\newcommand*{\abs}[1]{\left| #1 \right|}
\newcommand*{\card}[1]{\left| #1 \right|}
%\NewDocumentCommand{\norm}{sm}{\IfBooleanTF{#1}{\|#2\|}{\left\| #2 \right\|}}

\newcommand*{\const}{\mathrm{const}}

\newcommand*{\defeq}{\overset{.}{=}}
\newcommand*{\eqdef}{\overset{.}{=}}

\DeclareMathOperator*{\argmax}{arg\,max}
\DeclareMathOperator*{\argmin}{arg\,min}


\DeclarePairedDelimiter\parentheses{(}{)}
\DeclarePairedDelimiter\brackets{[}{]}
\DeclarePairedDelimiter\braces{\{}{\}}


%--------Sets--------
\newcommand{\R}{\mathbb{R}}
\newcommand{\Rzero}{\mathbb{R}_{\geq 0}}
\newcommand{\Nat}{\mathbb{N}}
\newcommand{\NatZ}{\mathbb{N}_0}


%--------Symbols--------
%\renewcommand{\vec}[1]{\mathbold{#1}}
%\newcommand{\mat}[1]{\mathbold{#1}}
\newcommand{\rvec}[1]{\mathbf{#1}}
%\newcommand{\set}[1]{#1}
\newcommand{\spa}[1]{\mathcal{#1}}

\newcommand{\mean}[1]{\overline{#1}}
\newcommand{\compl}[1]{\overline{#1}}
\newcommand{\old}[1]{#1^{\mathrm{old}}}
\newcommand{\opt}[1]{#1^\star}

\newcommand{\altpi}{\Pi} % \vec{\uppi}



\NewDocumentCommand{\fnv}{oo}{v\IfValueT{#2}{_{#2}}\IfValueT{#1}{^{#1}}}
\RenewDocumentCommand{\v}{somo}{\IfBooleanTF{#1}{\fnv[\star][#4]\parentheses{#3}}{\fnv[#2][#4]\parentheses{#3}}}
\NewDocumentCommand{\fnq}{oo}{q\IfValueT{#2}{_{#2}}\IfValueT{#1}{^{#1}}}
\NewDocumentCommand{\q}{sommo}{\IfBooleanTF{#1}{\fnq[\star][#5]\parentheses{#3,#4}}{\fnq[#2][#5]\parentheses{#3,#4}}}
\NewDocumentCommand{\fnV}{oo}{V\IfValueT{#2}{_{#2}}\IfValueT{#1}{^{#1}}}
\NewDocumentCommand{\V}{somo}{\IfBooleanTF{#1}{\fnV[\star][#4]\parentheses{#3}}{\fnV[#2][#4]\parentheses{#3}}}
\NewDocumentCommand{\fnQ}{oo}{Q\IfValueT{#2}{_{#2}}\IfValueT{#1}{^{#1}}}
\NewDocumentCommand{\Q}{sommo}{\IfBooleanTF{#1}{\fnQ[\star][#5]\parentheses{#3,#4}}{\fnQ[#2][#5]\parentheses{#3,#4}}}
\NewDocumentCommand{\fna}{oo}{a\IfValueT{#2}{_{#2}}\IfValueT{#1}{^{#1}}}
\RenewDocumentCommand{\a}{sommo}{\IfBooleanTF{#1}{\fna[\star][#5]\parentheses{#3,#4}}{\fna[#2][#5]\parentheses{#3,#4}}}
\NewDocumentCommand{\fnA}{oo}{A\IfValueT{#2}{_{#2}}\IfValueT{#1}{^{#1}}}
\NewDocumentCommand{\fnAhat}{oo}{\hat{A}\IfValueT{#2}{_{#2}}\IfValueT{#1}{^{#1}}}
\NewDocumentCommand{\A}{sommo}{\IfBooleanTF{#1}{\fnA[\star][#5]\parentheses{#3,#4}}{\fnA[#2][#5]\parentheses{#3,#4}}}
\NewDocumentCommand{\fnj}{o}{J\IfValueT{#1}{_{#1}}}
\RenewDocumentCommand{\j}{mo}{\fnj[#2]\parentheses{#1}}
\NewDocumentCommand{\fnJ}{o}{\widehat{J}\IfValueT{#1}{_{#1}}}
\NewDocumentCommand{\J}{mo}{\fnJ[#2]\parentheses{#1}}

\NewDocumentCommand{\pset}{m}{\mathcal{P}\parentheses*{#1}}

\NewDocumentCommand{\pf}{mm}{{#1}_\sharp #2}

\NewDocumentCommand{\grad}{e_}{\boldsymbol{\nabla}\IfValueT{#1}{_{\!\!#1}\,}}
\NewDocumentCommand{\jac}{}{\mD}
\NewDocumentCommand{\hes}{}{\mH}
\NewDocumentCommand{\dive}{}{\grad\cdot}
\NewDocumentCommand{\lapl}{}{\Delta}

\NewDocumentCommand{\BigO}{m}{O\parentheses*{#1}}
\NewDocumentCommand{\BigOTil}{m}{\widetilde{O}\parentheses*{#1}}

\NewDocumentCommand{\transpose}{m}{#1^\top}
\NewDocumentCommand{\inv}{m}{#1^{-1}}
\RenewDocumentCommand{\det}{m}{\mathrm{det}\parentheses*{#1}}
\NewDocumentCommand{\tr}{m}{\mathrm{tr}\parentheses*{#1}}
\NewDocumentCommand{\diag}{om}{\mathrm{diag}\IfValueT{#1}{_{#1}}{}\braces{#2}}
\NewDocumentCommand{\msqrt}{m}{#1^{\nicefrac{1}{2}}}
\NewDocumentCommand{\vecop}{m}{\mathrm{vec}\brackets{#1}}

%--------Common vectors/matrices/sets--------
\newcommand{\vzero}{\vec{0}}
\newcommand{\vone}{\vec{1}}
\newcommand{\va}{\vec{a}}
\newcommand{\vap}{\vec{a'}}
\newcommand{\vas}{\vec{\opt{a}}}
\newcommand{\vb}{\vec{b}}
\newcommand{\vc}{\vec{c}}
\newcommand{\vd}{\vec{d}}
\newcommand{\ve}{\vec{e}}
\newcommand{\vf}{\vec{f}}
\newcommand{\vfhat}{\vec{\hat{f}}}
\newcommand{\vg}{\vec{g}}
\newcommand{\vh}{\vec{h}}
\newcommand{\vk}{\vec{k}}
\newcommand{\vm}{\vec{m}}
\newcommand{\vp}{\vec{p}}
\newcommand{\vq}{\vec{q}}
\newcommand{\vr}{\vec{r}}
\newcommand{\vs}{\vec{s}}
\newcommand{\vt}{\vec{t}}
\newcommand{\vu}{\vec{u}}
\newcommand{\vve}{\vec{v}}
\newcommand{\vvp}{\vec{v'}}
\newcommand{\vvs}{\vec{\opt{v}}}
\newcommand{\vw}{\vec{w}}
\newcommand{\vwhat}{\vec{\hat{w}}}
\newcommand{\vx}{\vec{x}}
\newcommand{\vxp}{\vec{x'}}
\newcommand{\vxs}{\vec{\opt{x}}}
\newcommand{\vy}{\vec{y}}
\newcommand{\vyp}{\vec{y'}}
\newcommand{\vz}{\vec{z}}
\newcommand{\valpha}{\vec{\alpha}}
\newcommand{\valphahat}{\vec{\hat{\alpha}}}
\newcommand{\vdelta}{\vec{\delta}}
\newcommand{\vDelta}{\vec{\Delta}}
\newcommand{\vepsilon}{\vec{\epsilon}}
\newcommand{\vvarepsilon}{\vec{\varepsilon}}
\newcommand{\veta}{\vec{\eta}}
\newcommand{\vlambda}{\vec{\lambda}}
\newcommand{\vmu}{\vec{\mu}}
\newcommand{\vmuhat}{\vec{\hat{\mu}}}
\newcommand{\vmup}{\vec{\mu'}}
\newcommand{\vnu}{\vec{\nu}}
\newcommand{\vomega}{\vec{\omega}}
\newcommand{\vphi}{\vec{\phi}}
\newcommand{\vpi}{\vec{\pi}}
\newcommand{\vpsi}{\vec{\psi}}
\newcommand{\vvarphi}{\vec{\varphi}}
\newcommand{\vvarphihat}{\vec{\hat{\varphi}}}
\newcommand{\vtheta}{\vec{\theta}}
\newcommand{\vthetahat}{\vec{\hat{\theta}}}
\newcommand{\vxi}{\vec{\xi}}
\newcommand{\mzero}{\mat{0}}
\newcommand{\mA}{\mat{A}}
\newcommand{\mB}{\mat{B}}
\newcommand{\mBs}{\mat{\opt{B}}}
\newcommand{\mC}{\mat{C}}
\newcommand{\mD}{\mat{D}}
\newcommand{\mF}{\mat{F}}
\newcommand{\mH}{\mat{H}}
\newcommand{\mI}{\mat{I}}
\newcommand{\mK}{\mat{K}}
\newcommand{\mL}{\mat{L}}
\newcommand{\mCalL}{\mat{\mathcal{L}}}
\newcommand{\mM}{\mat{M}}
\newcommand{\mP}{\mat{P}}
\newcommand{\mQ}{\mat{Q}}
\newcommand{\mS}{\mat{S}}
\newcommand{\mT}{\mat{T}}
\newcommand{\mU}{\mat{U}}
\newcommand{\mV}{\mat{V}}
\newcommand{\mW}{\mat{W}}
\newcommand{\mX}{\mat{X}}
\newcommand{\mLambda}{\mat{\Lambda}}
\newcommand{\mPhi}{\mat{\Phi}}
\newcommand{\mPi}{\mat{\Pi}}
\newcommand{\mSigma}{\mat{\Sigma}}
\newcommand{\mSigmap}{\mat{\Sigma'}}
\newcommand{\rG}{\rvec{G}}
\newcommand{\rQ}{\rvec{Q}}
\newcommand{\rU}{\rvec{U}}
\newcommand{\rV}{\rvec{V}}
\newcommand{\rW}{\rvec{W}}
\newcommand{\rX}{\rvec{X}}
\newcommand{\rXp}{\rvec{X'}}
\newcommand{\rY}{\rvec{Y}}
\newcommand{\rZ}{\rvec{Z}}
\newcommand{\sA}{\set{A}}
\newcommand{\sB}{\set{B}}
\newcommand{\sC}{\set{C}}
\newcommand{\sD}{\set{D}}
\newcommand{\sI}{\set{I}}
\newcommand{\sM}{\set{M}}
\newcommand{\sS}{\set{S}}
\newcommand{\sU}{\set{U}}
\newcommand{\sX}{\set{X}}
\newcommand{\sY}{\set{Y}}
\newcommand{\sZ}{\set{Z}}
\newcommand{\spA}{\spa{A}}
\newcommand{\spB}{\spa{B}}
\newcommand{\spC}{\spa{C}}
\newcommand{\spD}{\spa{D}}
\newcommand{\spF}{\spa{F}}
\newcommand{\spH}{\spa{H}}
\newcommand{\spL}{\spa{L}}
\newcommand{\spM}{\spa{M}}
\newcommand{\spO}{\spa{O}}
\newcommand{\spP}{\spa{P}}
\newcommand{\spQ}{\spa{Q}}
\newcommand{\spT}{\spa{T}}
\newcommand{\spW}{\spa{W}}
\newcommand{\spX}{\spa{X}}
\newcommand{\spY}{\spa{Y}}
\newcommand{\spZ}{\spa{Z}}
\newcommand{\fs}{\opt{f}}
\newcommand{\ps}{\opt{p}}
\newcommand{\qs}{\opt{q}}
\newcommand{\xs}{\opt{x}}
\newcommand{\ys}{\opt{y}}
\newcommand{\Bs}{\opt{B}}
\newcommand{\Qs}{\opt{Q}}
\newcommand{\sSs}{\opt{\sS}}
\newcommand{\hQs}{\opt{\hat{Q}}}
\newcommand{\Vs}{\opt{V}}
\newcommand{\pis}{\opt{\pi}}

\newcommand{\vF}{\rvec{F}}
\newcommand{\vS}{\rvec{S}}
\newcommand{\vT}{\rvec{T}}


\renewcommand\qedsymbol{$\blacksquare$}


\newcommand{\dx}{\mathrm{d}x}
\newcommand{\ddx}{\frac{\mathrm{d}}{\mathrm{d}x}}
\newcommand{\dt}{\mathrm{d}t}
\newcommand{\du}{\mathrm{d}u}
\newcommand{\dve}{\mathrm{d}v}
\newcommand{\dw}{\mathrm{d}w}
\newcommand{\dy}{\mathrm{d}y}
\newcommand{\dz}{\mathrm{d}z}
\newcommand{\dF}{\mathrm{d}F}
\newcommand{\dV}{\mathrm{d}V}
\newcommand{\dr}{\mathrm{d}r}
\newcommand{\dtheta}{\mathrm{d}\theta}
\newcommand{\drho}{\mathrm{d}\rho}
\newcommand{\dphi}{\mathrm{d}\varphi}

\newcommand{\Rn}{\mathbb{R}^n}
\newcommand{\Rm}{\mathbb{R}^m}
\newcommand{\Rk}{\mathbb{R}^k}
\newcommand{\und}{\text{ und }}
\newcommand{\oder}{\text{ oder }}
\newcommand{\bydef}{\underset{def.}{=}}
\newcommand{\BH}{\underset{\textrm{B-H}}{=}}

\newcommand{\Follows}{\Longrightarrow\ }
\newcommand{\sameas}{\Longleftrightarrow}
\newcommandx{\Laplace}[2][1=f(t), 2=s]{\mathscr{L}\{#1\}(#2)}
\newcommandx{\LaplaceInv}[2][1=F(s), 2=t]{\mathscr{L}^{-1}\{#1\}(#2)}
\DeclareMathOperator{\arccosh}{Arcosh}
\DeclareMathOperator{\arcsinh}{Arsinh}
\DeclareMathOperator{\arctanh}{Artanh}
\DeclareMathOperator{\arcsech}{arcsech}
\DeclareMathOperator{\arccsch}{arcCsch}
\DeclareMathOperator{\arccoth}{arcCoth} 

\def\doubleunderline#1{\underline{\underline{#1}}}
\def\ez{\begin{flushright}\underline{ez.}\end{flushright}}
%\[
%   \Laplace[\cos(x)]=\int_{t=0}^{\infty}f(t)e^{-st}dt
%\]

\newcommand{\Z}{\mathbb{Z}}
\newcommand{\N}{\mathbb{N}}

\newcommand{\C}{\mathbb{C}}

\newcommand{\inttext}{\shortintertext}

% The following example defines \colorboxed as wrapper around amsmath's \boxed to set the frame color. It uses package xcolor for the color support to save the current color . before changing the color for the frame. Inside the box, the previous saved color is restored. This avoids a white background of \fcolorbox, since there is no "transparent" color.

% The macro also supports an optional argument for specifying the color model.

% Definition of \boxed in amsmath.sty:
% \newcommand{\boxed}[1]{\fbox{\m@th$\displaystyle#1$}}


% Syntax: \colorboxed[<color model>]{<color specification>}{<math formula>}
\newcommand*{\colorboxed}{}
\def\colorboxed#1#{%
  \colorboxedAux{#1}%
}
\newcommand*{\colorboxedAux}[3]{%
  % #1: optional argument for color model
  % #2: color specification
  % #3: formula
  \begingroup
    \colorlet{cb@saved}{.}%
    \color#1{#2}%
    \boxed{%
      \color{cb@saved}%
      #3%
    }%
  \endgroup
}

%
\usepackage[utf8]{inputenc}
\usepackage[ngerman]{babel}
\usepackage[a4paper, top=1in, bottom=1.3in, rmargin=1.5in, left=1.4in, marginparwidth=80pt]{geometry}
\usepackage{etoc}
%\usepackage[tight]{minitoc}


\begin{document}

\title{\vspace*{-2.5em}Problem Set 2, Summary \& Tips}
\author{Vikram R. Damani\\
    Analysis~II}

\maketitle
%\dosectoc
%\sectoc

%\tableofcontents % //FIXME: make small toc

\section{Tipps}

\begin{nexercise}{Aufgabe 1}{1}\addcontentsline{toc}{subsection}{A1}
    Sei $T$ ein Tetraeder mit Eckpunkten $(0,0,0), (1,0,0), (0,2,0), (0,0,3)$ und $f(x,y,z)=x+y$. Berechnen Sie das Integral
    \begin{align}
        \iiint_T f(x,y,z)\dV
    \end{align}
\end{nexercise}

\begin{tips}{1} Es hilft $T$ zu skizzieren um die Integrationsgrenzen intuitiver zu bestimmen können.

    \emph{Recall: Die inneren Integrationsgrenzen sind Funktionen der äusseren Variablen!}
\end{tips}\vspace*{1em}

\begin{figure}[htbp!]
    \centering
    \begin{tikzpicture}[scale=1.2]
        \begin{axis}[
                xlabel=$x$, ylabel=$y$,
                small,
                grid=major,
            ]
            \addplot3 [
                surf,
                domain=-3:3,
                domain y=-3:3,
            ] {exp(-(x^2+y^2)/8)-exp(-2)};
        \end{axis}
    \end{tikzpicture}
    \caption{Der Graph von $f(x,y)=e^{\frac{-(x^2+y^2)}{8}}-e^{-2}$ dargestellt auf dem Gebiet $B=[-2,2]\times[-2,2]\subset\R^2$.}
\end{figure}

\begin{nexercise}{Aufgabe 2}{2}
    \begin{enumerate}[label=(\alph*)]
        \item Die Oberfläche einer Insel sei gegeben durch
              \begin{align}
                  f(x,y)=e^{\frac{-(x^2+y^2)}{8}}-e^{-2}.
              \end{align}
              Berechnen Sie das Volumen der Insel, welches über der Wasseroberfläche (Höhe $z=0$) liegt.

        \item Sei $K$ der endliche Körper, der im ersten Oktanten (d.h. $x,y,z>0$) durch den
              Zylinder $y^2+z^2=9$ und die Ebene $x=y$ begrenzt wird. Erstellen Sie eine
              Skizze dieser Situation und berechnen Sie das Volumen von $K$.
    \end{enumerate}
\end{nexercise}

\begin{tips}{2}
    \begin{enumerate}
        \item[(b)] Das Volumendes Zylinders ist gegeben durch das (Dreifach-)integral $\iiint_K \dV$. Die Grenzen des gegebenen Zylinders werden implizit durch den Kreis und explizit durch die Linie $y=x$ und $x,y,z\geq0$ beschrieben. Richtige Wahl der \p{Integrationsreihenfolge} vereinfacht die Parametrisierung der Grenzen.
    \end{enumerate}
\end{tips}\vspace*{1em}

\begin{nexercise}{Aufgabe 3}{3}
    Ein gerader Kreiszylinder mit Radius $R$, $(x^2+y^2\leq R^2)$, und Höhe $H$, $(0\leq{}z\leq{}H)$, habe eine Dichte von $\rho(x,y,z)=1+x^2+y^2+z$. Berechnen Sie die Masse und das Trägheitsmoment bei Rotation um die $z$-Achse.
\end{nexercise}

\begin{tips}{3}
    Die Masse ist gegeben durch $\iiint_V \rho \dV$. Da wir über einen Zylinder integrieren, vereinfacht eine geeignete Wahl des Koordinatensystems die Berechnung. Achtung: Der Integrand muss entsprechend angepasst werden.

    Das Trägheitsmoment ergibt sich aus dem quadrierten Abstand zur Drehachse mal der Dichte über den ganzen Körper integriert: $\iiint_V r^2 \rho\,\dV$.
\end{tips}

\section{Theorie}
%\localtableofcontents

\begin{thmb}{\np{\emph{[Flächenelement in Polarkoordinaten]}}}
    Das Flächenelement $\dF$ lässt sich in Polarkoordinaten $(\rho,\varphi)$ schreiben als
    \begin{align}
        \begin{aligned}
            \dF & =\dx\,\dy=\dy\,\dx                                                  \\
                & \colorboxed{antiquefuchsia}{=\rho\,\drho\,\dphi=\rho\,\dphi\,\drho}
        \end{aligned}
    \end{align}
\end{thmb}\vspace*{1em}

\begin{thmb}{\np{\emph{[Volumenintegrale]}}}
    Sei $\mathbf{B}\subseteq\R^3$ und
    \begin{align}
         & \begin{aligned}
               f\colon\quad \mathbf{B} & \longrightarrow\R        \\
               \vec{r}                 & \longmapsto f(\vec{r}\,)
           \end{aligned} \\
    \end{align}
    eine Funktion in 3 Variablen.

    Das Volumenintegral $V$ der Funktion $f$ über das Gebiet $\mathbf{B}$ ist
    gegeben durch
    \begin{align}
        V & =\iiint_B \dV
    \end{align}
    In kartesischen Koordinaten gilt $\dV=\dx\,\dy\,\dz$
\end{thmb}

\begin{rmk}{}{}
    Wie bei Flächenintegralen liegt die Schwierigkeit hier eher beim Bestimmen der Integrationsgrenzen statt beim eigentlichen Integrieren.
\end{rmk}

\subsection{Koordinatentransformation}

Sei eine Koordinatentransformation gegeben
\begin{align}
    \begin{tabular}{ccc}
        $x=x(u,v)$ &  & $y=y(u,v)$
    \end{tabular}
\end{align}

Wie verhält sich der Flächeninhalt von unserem infinitesimalen
Flächen-/Volumenelement unter dieser Transformation?
%Also
% \begin{align*}
%     \dF=\dx\,\dy=\;???
% \end{align*}

\begin{figure}[!htb]
  \centering

  \makeatletter
  \define@key{x sphericalkeys}{radius}{\def\myradius{#1}}
  \define@key{x sphericalkeys}{theta}{\def\mytheta{#1}}
  \define@key{x sphericalkeys}{phi}{\def\myphi{#1}}
  \tikzdeclarecoordinatesystem{x spherical}{% %%%rotation around x
    \setkeys{x sphericalkeys}{#1}%
    \pgfpointxyz{\myradius*cos(\mytheta)}{\myradius*sin(\mytheta)*cos(\myphi)}{\myradius*sin(\mytheta)*sin(\myphi)}}

  %along y axis
  \define@key{y sphericalkeys}{radius}{\def\myradius{#1}}
  \define@key{y sphericalkeys}{theta}{\def\mytheta{#1}}
  \define@key{y sphericalkeys}{phi}{\def\myphi{#1}}
  \tikzdeclarecoordinatesystem{y spherical}{% %%%rotation around x
    \setkeys{y sphericalkeys}{#1}%
    \pgfpointxyz{\myradius*sin(\mytheta)*cos(\myphi)}{\myradius*cos(\mytheta)}{\myradius*sin(\mytheta)*sin(\myphi)}}

  %along z axis
  \define@key{z sphericalkeys}{radius}{\def\myradius{#1}}
  \define@key{z sphericalkeys}{theta}{\def\mytheta{#1}}
  \define@key{z sphericalkeys}{phi}{\def\myphi{#1}}
  \tikzdeclarecoordinatesystem{z spherical}{% %%%rotation around x
    \setkeys{z sphericalkeys}{#1}%
    \pgfpointxyz{\myradius*sin(\mytheta)*cos(\myphi)}{\myradius*sin(\mytheta)*sin(\myphi)}{\myradius*cos(\mytheta)}}

  \makeatother

  \begin{subfigure}[t]{0.5\textwidth}
    \begin{tikzpicture}[scale=1.2]
      \draw[fill=yellow] (0.2,0.2)coordinate(aa)  rectangle (2,3)coordinate(bb);
      \draw[-latex] (0,0) -- (2.5,0) node[above](xx){$\vv{x}$};
      \draw[-latex] (0,0) coordinate(oo) -- (0,3.3) node[right](yy){$\vv{y}$};

      \draw[dashed] (oo-|aa)node[below]{$X_1$} --(aa);
      \draw[dashed] (oo|-aa)node[left]{$Y_1$} --(aa);
      \draw[dashed] (oo-|bb)node[below](X2){$X_2$} --(aa-|bb);
      \draw[dashed] (oo|-bb)node[left](Y2){$Y_2$} --(aa|-bb);
      \node[fill=gray] (P) at (1.2,1.3){+};
      \node[above=0em of P] {$\mathrm{d}x$};
      \node[right=0em of P] {$\mathrm{d}y$};
      \draw[dashed] (P.center) --(P.center|-oo)node[below]{$x$};
      \draw[dashed] (P.center) --(oo|-P.center)node[left]{$y$};

      \node[fit=(xx) (Y2) (X2)](cadre){};
      \node[below=0em of cadre]{$\dF= \mathrm{d} x \cdot \mathrm{d} y$};
    \end{tikzpicture}
    \subcaption{$\dF$ in kartesischen Korrdinaten}
  \end{subfigure}
  \hfill
  \begin{subfigure}[t]{0.45\textwidth}
    \begin{tikzpicture}[scale=1.25]
      \draw [dashed] (0,0) coordinate(oo) -- (15:1.5cm)coordinate(aa) --(15:3cm)coordinate(bb);
      \draw [dashed] (0,0) coordinate(oo) -- (75:1.5cm)coordinate(aa1) --(75:3cm)coordinate(bb1);
      \draw[fill=yellow] (aa) arc (15:75:1.5cm) -- (bb1) arc (75:15:3cm) --(aa);
      \draw[-latex] (0,0) -- (3.5,0) node[above]{$\vv{x}$};
      \draw[-latex] (0,0) coordinate(oo) -- (0,3) node[right]{$\vv{y}$};

      \draw [-latex] (2.5,0) arc(0:15:2.5cm);
      \path (0,0) -- (60:.9cm)node[above right=0em]  {$\varphi_2$} ;
      \draw [-latex] (1,0) arc (0:75:1cm);
      \path (0,0) -- (5:2.5cm)node[right]  {$\varphi_1$} ;
      \draw[dashed] (1.5,0) node[below]{$R_1$}arc (0:15:1.5cm);
      \draw[dashed] (3,0) node[below](R2){$R_2$}arc (0:15:3cm);

      \draw[fill=gray] (35:2cm) coordinate(aa) arc (35:43:2cm)coordinate(bb) --(43:2.5) arc (43:35:2.5) -- (35:2cm) ;
      \draw[dashed] (0,0) -- (aa);
      \draw[dashed] (0,0) --(bb);
      \node (P) at (39:2.25){+};
      \node[above right=0.125em of P]{$r\cdot\mathrm{d}\varphi$};
      \draw[dashed] (0,0) --(P.center)node[right]{$P$};
      \node[above left=0em of P]{$\mathrm{d}r$};
      \draw[latex-latex] (46:2cm) -- (46:2.5cm);
      \draw[latex-latex] (35:2.7cm) arc (35:43:2.7cm);
      \draw[dashed] (2.25,0) node[below]{$r$}arc (0:39:2.25cm);
      \draw [-latex] (1.7,0) arc (0:38:1.7cm);
      \path (0,0) -- (6:1.7cm)node[right]  {$\varphi$} ;

      \node[fit=(xx) (yy) (R2)](cadre){};
      \node[below=0em of cadre]{$\mathrm{d} F=r\cdot \mathrm{d} \varphi \cdot \mathrm{d} r$};
    \end{tikzpicture}
    \subcaption{$\dF$ in Polarkoordinaten}
  \end{subfigure}
  \begin{subfigure}[t]{0.45\textwidth}
    \begin{tikzpicture}[scale=2.7]
      \begin{scope}[canvas is zx plane at y=0]
        %\draw (0,0) circle (1cm);
        \draw (0,0)coordinate(O) -- (1,0) (0,0) -- (0,1);
        \coordinate (Z0) at (0:0.5);
        \draw[fill=green!30,opacity=0.3] (0,0) -- (10:1)coordinate(A1) arc (10:110:1) coordinate(A2)-- (0,0);
        \foreach \aa in {10,15,20,...,110}{
            \coordinate (A\aa) at (\aa:1);
          }
      \end{scope}

      \begin{scope}[canvas is zx plane at y=0.9]
        \draw[fill=green!30,opacity=0.3] (0,0) -- (10:1)coordinate(B1) arc (10:110:1) coordinate(B2)-- (0,0);
        \foreach \aa in {10,15,20,...,110}{
            \coordinate (B\aa) at (\aa:1);
          }

      \end{scope}

      \begin{scope}[canvas is zx plane at y=0.4]
        \foreach \aa in {30,32,34,...,42}{
            \coordinate (C\aa) at (\aa:1);
          }
          \coordinate (out1) at (30:1.5);
          \coordinate (out2) at (42:1.5);

        \draw[dashed](0,0)-- (0:1.5);
        \coordinate (Z4) at (0:0.5);
        \draw[dashed](0,0) -- (C30) coordinate[pos=2];% (ff) -- (ff);
        \draw[dashed,color=red] (C30) -- (out1); %??
        \draw[dashed](0,0) -- (C42) coordinate[pos=2];% (ff) -- (ff);
        \draw[dashed,color=red] (C42) -- (out2); %??
        \draw[-latex] (0:1.5) arc (0:30:1.5) node[pos=0.6,below]{$\varphi$};
        \draw[latex-latex] (30:1.6) arc (30:40:1.6)node[pos=0.5,below]{$\mathrm{d}\,\varphi$};
      \end{scope}

      \begin{scope}[canvas is zx plane at y=0.65]
        \foreach \aa in {30,32,34,...,42}{
            \coordinate (D\aa) at (\aa:1);
          }
        \draw[dashed](0,0)coordinate(Z6) -- (D30);
        \draw[dashed](0,0) -- (D42);
        \coordinate (Z6) at (0:0.5);
      \end{scope}

      \draw[-latex] (0,0,0) -- (1.1,0,0) node[above](yy){$\vv{y_0}$};
      \draw[-latex] (0,0,0) -- (0,1.1,0) node[above](zz){$\vv{z_0}$};
      \draw[-latex] (0,0,0) -- (0,0,1.1) node[above left=-0.1em](xx){$\vv{x_0}$};

      \foreach \aa in {10,15,20,...,105}{
          \pgfmathsetmacro{\bb}{\aa+5}
          \fill[fill=black!30,opacity=0.3] (A\aa) -- (A\bb) -- (B\bb) -- (B\aa) -- cycle;
        }

      \foreach \aa in {30,32,34,...,40}{
          \pgfmathsetmacro{\bb}{\aa+2}
          \fill[fill=blue,opacity=0.3] (C\aa) -- (C\bb) -- (D\bb) -- (D\aa) -- cycle;
        }

      \draw[-latex] (Z0) -- (Z4) node[left,pos=0.5]{$z$};
      \draw[latex-latex] (Z4) -- (Z6) node[left,pos=0.5]{$\mathrm{d}\,z$};

      \node[fit=(xx) (yy) (zz)](cadre){};
      \node[below=0.5em of cadre]{ $\mathrm{d}F=r\cdot \mathrm{d}\varphi\cdot \mathrm{d} z $};
    \end{tikzpicture}
    \subcaption{$\dF$ in Zylinderkoordinaten}
  \end{subfigure}\hfill
  \begin{subfigure}[t]{0.45\textwidth}

    \tdplotsetmaincoords{60}{110}

    \pgfmathsetmacro{\rvec}{.8}
    \pgfmathsetmacro{\thetavec}{30}
    \pgfmathsetmacro{\phivec}{55}
    \pgfmathsetmacro{\dphi}{12}
    \pgfmathsetmacro{\dtheta}{12}
    \pgfmathsetmacro{\drvec}{0.15}
    \pgfmathsetmacro{\Rvec}{\rvec+\drvec}
    \pgfmathsetmacro{\Thetavec}{\thetavec+\dtheta}
    \pgfmathsetmacro{\Phivec}{\phivec+\dphi}

    \begin{tikzpicture}[scale=3.66,tdplot_main_coords]

      %-----------------------
      \coordinate (O) at (0,0,0);

      \tdplotsetcoord{P}{\rvec}{\thetavec}{\phivec}
      \tdplotsetcoord{P3}{\rvec}{\Thetavec}{\phivec}

      \tdplotsetcoord{Q}{\rvec}{\thetavec}{\Phivec}

      \tdplotsetcoord{Q3}{\rvec}{\Thetavec}{\Phivec}

      \tdplotsetthetaplanecoords{0}
      \draw[red,fill=green!30,,tdplot_rotated_coords] (0,0) -- (\rvec,0,0) arc (0:90:\rvec) -- (0,0);

      \tdplotsetthetaplanecoords{90}
      \draw[red,fill=green!30,,tdplot_rotated_coords] (0,0) -- (\rvec,0,0) arc (0:90:\rvec) -- (0,0);
      \draw (0,0) -- (30:\rvec) coordinate[pos=1.2](ff);
      \draw[latex-] (30:\rvec) -- (ff)--++(0,0.1)node[above]{$R$};

      \draw[] (\rvec,0,0) arc (0:90:\rvec);
      \foreach \aa in {0,2,4,...,90}{
          \coordinate (A\aa) at (\aa:\rvec);
        }

      %if you want to convince yourself that this works:
      \draw[blue,fill=green!30,opacity=0.5] plot[variable=\x,domain=90:0]
      (z spherical cs: radius = \rvec, phi = 0, theta= \x)
      -- plot[variable=\x,domain=0:90]
      (z spherical cs: radius = \rvec, phi = \x, theta= 0)
      -- plot[variable=\x,domain=0:90]
      (z spherical cs: radius = \rvec, phi = 90, theta= \x)
      -- plot[variable=\x,domain=90:0]
      (z spherical cs: radius = \rvec, phi = \x, theta= 90);

      %draw figure contents
      %--------------------
      %draw the main coordinate system axes
      \draw[thick,->] (0,0,0) -- (0.9,0,0) node[above left](xx){$\vv{x}$};
      \draw[thick,->] (0,0,0) -- (0,0.9,0) node[above](yy){$\vv{y}$};
      \draw[thick,->] (0,0,0) -- (0,0,0.9) node[right](zz){$\vv{z}$};

      %draw a line from origin to point (P) 
      \draw[,color=red] (O) -- (P)

      ;
      \draw[,color=red] (O) -- (P3);
      \draw[color=red] (O) -- (Q3);

      \draw[dashed, color=red] (O) -- (Pxy);
      \draw[dashed, color=red] (P) -- (Pxy);
      %
      \draw[dashed, color=red] (O) -- (P3xy);
      \draw[dashed, color=red] (P3) -- (P3xy);

      %draw a line from origin to point (Q) 
      \draw[,color=red] (O) -- (Q);
      \draw[,color=red] (O) -- (Q3);

      \draw[dashed, color=red] (O) -- (Qxy);
      \draw[dashed, color=red] (Q) -- (Qxy);
      %
      \draw[dashed, color=red] (O) -- (Q3xy);
      \draw[dashed, color=red] (Q3) -- (Q3xy);

      \pgfmathsetmacro{\Rproj}{\Rvec*sin(\Thetavec)}

      \draw[latex-latex] (P3xy) -- (Q3xy)node[below]{$\mathrm{d}\theta$};

      \tdplotdrawarc[-latex]{(O)}{0.3}{0}{\phivec}{anchor=north}{$\theta$}

      \tdplotsetthetaplanecoords{\phivec}

      \tdplotdrawarc[tdplot_rotated_coords,-latex]{(0,0,0)}{0.5}{0}{\thetavec}{anchor=south west}{$\varphi$}
      \tdplotdrawarc[tdplot_rotated_coords,latex-latex]{(0,0,0)}{0.55}{\thetavec}{\Thetavec}{anchor=south west}{$\mathrm{d}\phi$}

      \draw[dashed,tdplot_rotated_coords] (\rvec,0,0) arc (0:90:\rvec);

      \draw[dashed] (\rvec,0,0) arc (0:90:\rvec);

      \tdplotsetthetaplanecoords{\Phivec}

      \draw[dashed,tdplot_rotated_coords] (\rvec,0,0) arc (0:90:\rvec);
      \begin{scope}[tdplot_main_coords]
        \draw[blue,fill=blue!30 ] plot[variable=\x,domain=\thetavec:\Thetavec]
        (z spherical cs: radius = \rvec, phi = \Phivec, theta= \x)
        -- plot[variable=\x,domain=\Phivec:\phivec]
        (z spherical cs: radius = \rvec, phi = \x, theta= \Thetavec)
        -- plot[variable=\x,domain=\Thetavec:\thetavec]
        (z spherical cs: radius = \rvec, phi = \phivec, theta= \x)
        -- plot[variable=\x,domain=\phivec:\Phivec]
        (z spherical cs: radius = \rvec, phi = \x, theta= \thetavec);
        %
      \end{scope}
      \node[fit=(xx) (yy) (zz)](cadre){};
      \node[below=0em of cadre]{$\mathrm{d} F=R\cdot  \sin\varphi \cdot \mathrm{d} \theta \cdot R\cdot \mathrm{d} \varphi$};
    \end{tikzpicture}
    \subcaption{$\dF$ in Kugelkoordinaten}
  \end{subfigure}

  \caption{Flächenelement $\mathrm{d}F$}
  \label{fig:elementdesurface}
\end{figure} % credit to https://tex.stackexchange.com/questions/622658/how-can-i-draw-an-infinitesimal-portion-of-a-surface

\begin{thmb}{\np{\emph{[Jacobi-Determinante (2D)]}}}
    Das Flächenelement $\dF$ unter einer Koordinatentransformation $\vec{r}(u,v)$ ist gegeben durch
    \begin{align}
        \colorboxed{antiquefuchsia}{\dF=\abs{\det{\mathbf{J}}}\du\,\dve}
    \end{align}
    mit der Jacobi Matrix
    \begin{align}
        \colorboxed{antiquefuchsia}{\mathbf{J}=\begin{pmatrix}
                                                       x_u & x_v \\
                                                       y_u & y_v
                                                   \end{pmatrix}}
    \end{align}
    Für das Gebietsintegral gilt dann
    \begin{align}
        V=\iint_B f(x,y)\,\dF = \iint_{\widetilde{B}} \widetilde{f}(u,v)\abs{\det{\mathbf{J}}} \,\du\,\dve
    \end{align}
    mit $\widetilde{f}(u,v)=f(x(u,v),y(u,v))$ und $\widetilde{B}=\vec{r}\,(B)$.
\end{thmb}


\begin{figure}[!htb]
    \makeatletter
    \define@key{x sphericalkeys}{radius}{\def\myradius{#1}}
    \define@key{x sphericalkeys}{theta}{\def\mytheta{#1}}
    \define@key{x sphericalkeys}{phi}{\def\myphi{#1}}
    \tikzdeclarecoordinatesystem{x spherical}{% %%%rotation around x
        \setkeys{x sphericalkeys}{#1}%
        \pgfpointxyz{\myradius*cos(\mytheta)}{\myradius*sin(\mytheta)*cos(\myphi)}{\myradius*sin(\mytheta)*sin(\myphi)}}

    %along y axis
    \define@key{y sphericalkeys}{radius}{\def\myradius{#1}}
    \define@key{y sphericalkeys}{theta}{\def\mytheta{#1}}
    \define@key{y sphericalkeys}{phi}{\def\myphi{#1}}
    \tikzdeclarecoordinatesystem{y spherical}{% %%%rotation around x
        \setkeys{y sphericalkeys}{#1}%
        \pgfpointxyz{\myradius*sin(\mytheta)*cos(\myphi)}{\myradius*cos(\mytheta)}{\myradius*sin(\mytheta)*sin(\myphi)}}

    %along z axis
    \define@key{z sphericalkeys}{radius}{\def\myradius{#1}}
    \define@key{z sphericalkeys}{theta}{\def\mytheta{#1}}
    \define@key{z sphericalkeys}{phi}{\def\myphi{#1}}
    \tikzdeclarecoordinatesystem{z spherical}{% %%%rotation around x
        \setkeys{z sphericalkeys}{#1}%
        \pgfpointxyz{\myradius*sin(\mytheta)*cos(\myphi)}{\myradius*sin(\mytheta)*sin(\myphi)}{\myradius*cos(\mytheta)}}

    \makeatother

    \tdplotsetmaincoords{60}{110}

    \pgfmathsetmacro{\rvec}{.7}
    \pgfmathsetmacro{\thetavec}{30}
    \pgfmathsetmacro{\phivec}{50}
    \pgfmathsetmacro{\dphi}{12}
    \pgfmathsetmacro{\dtheta}{12}
    \pgfmathsetmacro{\drvec}{0.2}
    \pgfmathsetmacro{\Rvec}{\rvec+\drvec}
    \pgfmathsetmacro{\Thetavec}{\thetavec+\dtheta}
    \pgfmathsetmacro{\Phivec}{\phivec+\dphi}

    \centering
    \begin{subfigure}[t]{0.45\textwidth}
        \begin{tikzpicture}[scale=0.5]
            \coordinate(O) at (0,0,0);
            \coordinate(A1) at (4,4,4);
            \draw[fill=red!80,opacity=0.5] ($(A1)+(0.4,0.4,0) $)coordinate(aa)  --($(A1)+(0.4,-0.4,0)$)coordinate(bb) --($(A1)+(-0.4,-0.4,0)$)coordinate(cc) --($(A1)+(-0.4,0.4,0)$)coordinate(dd)-- cycle;
            \draw[fill=red!80,opacity=0.5] ($(A1)+(0.4,0.4,0.8) $)coordinate(aa1) --($(A1)+(0.4,-0.4,0.8)$)coordinate(bb1) --($(A1)+(-0.4,-0.4,0.8)$)coordinate(cc1)--($(A1)+(-0.4,0.4,0.8)$)coordinate(dd1)-- cycle;
            \foreach \aa/\bb in {cc/dd,dd/aa,aa/bb,bb/cc}{
                    \draw[fill=red!80,opacity=0.5] (\aa) --(\aa1) --(\bb1) --(\bb) ;
                }

            \coordinate(A) at (6,5.8,4.8);
            \draw[fill=green!50,opacity=0.5] ($(A)+(2,2,0) $)coordinate(aa)  --($(A)+(2,-2,0)$)coordinate(bb) --($(A)+(-2,-2,0)$)coordinate(cc) --($(A)+(-2,2,0)$)coordinate(dd)-- cycle;
            \draw[fill=green!80,opacity=0.5] ($(A)+(2,2,4) $)coordinate(aa1) --($(A)+(2,-2,4)$)coordinate(bb1) --($(A)+(-2,-2,4)$)coordinate(cc1)--($(A)+(-2,2,4)$)coordinate(dd1)-- cycle;
            \foreach \aa/\bb in {cc/dd,dd/aa,aa/bb,bb/cc}{
                    \draw[fill=green!50,opacity=0.5] (\aa) --(\aa1) --(\bb1) --(\bb) ;
                }
            %\draw[dashed]  (0,4,0)node[above left]{$y$}  --  (0,4,4)coordinate(aa) --(0,0,4);
            %\draw[dashed]  (4,0,0) --(4,0,4) coordinate(bb)--(0,0,4)node[left]{$z$} ;
            %\draw[dashed]  (4,0,0)node[above right]{$x$} --(4,4,0) coordinate(cc)--(0,4,0);
            %\draw[dashed] (aa) --(A1);
            %\draw[dashed] (bb) --(A1);
            %\draw[dashed] (cc) --(A1);
            \node[above =1em of A1]{$\mathrm{d}y$};
            \node[left =1.8em of A1]{$\mathrm{d}z$};
            \node[below right =0.5em of A1]{$\mathrm{d}x$};

            \draw[-latex](0,0,0) -- (5,0,0) node[right](xx){$\vv{y_0}$};
            \draw[-latex](0,0,0) -- (0,5,0) node[right](yy){$\vv{z_0}$};
            \draw[-latex](0,0,0) -- (0,0,3) node[right](zz){$\vv{x_0}$};

            \node[fit=(xx) (yy) (zz)](cadre){};
            \node[below=0em of cadre]{ $\mathrm{d}V= \mathrm{d}x\cdot \mathrm{d} y \cdot\mathrm{d}z$};
        \end{tikzpicture}
        \subcaption{$\dV$ in kartesischen Koordinaten}
    \end{subfigure}
    \hfill
    \begin{subfigure}[t]{0.45\textwidth}
        \begin{tikzpicture}[scale=2.6]

            \begin{scope}[canvas is zx plane at y=0]
                %\draw (0,0) circle (1cm);
                \draw (0,0)coordinate(O) -- (1,0) (0,0) -- (0,1);
                \coordinate (Z0) at (0:0.1);
                \draw[fill=green!30,opacity=0.3] (0,0) -- (10:1)coordinate(A1) arc (10:110:1) coordinate(A2)-- (0,0);
                \foreach \aa in {10,15,20,...,110}{
                        \coordinate (A\aa) at (\aa:1);
                    }
            \end{scope}

            \begin{scope}[canvas is zx plane at y=0.9]
                %\draw (0,0) circle (1cm);
                \draw[fill=green!30,opacity=0.3] (0,0) -- (10:1)coordinate(B1) arc (10:110:1) coordinate(B2)-- (0,0);
                \foreach \aa in {10,15,20,...,110}{
                        \coordinate (B\aa) at (\aa:1);
                    }

            \end{scope}

            \begin{scope}[canvas is zx plane at y=0.4]
                %\draw (0,0) circle (1cm);
                \draw[fill=red!30,opacity=0.3] (30:0.7) -- (30:0.5)coordinate(A1) arc (30:50:0.5) -- (50:0.7) arc (50:30:0.7) -- cycle;
                \foreach \aa in {30,32,34,...,50}{
                        \coordinate (C\aa) at (\aa:0.7);
                        \coordinate (E\aa) at (\aa:0.5);
                    }
                \draw[dashed](0,0)-- (0:1);
                \coordinate (Z4) at (0:0.1);
                \draw[dashed](0,0) -- (C30) coordinate[pos=2] (ff) -- (ff);
                \draw[dashed](0,0) -- (C50) coordinate[pos=2] (ff) -- (ff);
                \draw[-latex] (0:0.8) arc (0:30:0.8)node[pos=0.4,below]{$\varphi$};
                \draw[latex-latex] (30:1) arc (30:50:1)node[pos=0.55,below=0.15em]{$\mathrm{d}\,\varphi$};
            \end{scope}

            \begin{scope}[canvas is zx plane at y=0.6]
                %\draw (0,0) circle (1cm);
                \draw[fill=red!30,opacity=0.3] (30:0.7) -- (30:0.5)coordinate(A1) arc (30:50:0.5) -- (50:0.7) arc (50:30:0.7) -- cycle;
                \foreach \aa in {30,32,34,...,50}{
                        \coordinate (D\aa) at (\aa:0.7);
                        \coordinate (F\aa) at (\aa:0.5);
                    }
                \draw[dashed](0,0)coordinate(Z6) -- (D30) coordinate[pos=2] (ff) -- (ff);
                \draw[dashed](0,0) -- (D50) coordinate[pos=2] (ff) -- (ff);
                \coordinate (Z6) at (0:0.1);
            \end{scope}

            \draw[-latex] (0,0,0) -- (1.1,0,0) node[above](yy){$\vv{y_0}$};
            \draw[-latex] (0,0,0) -- (0,1.1,0) node[above](zz){$\vv{z_0}$};
            \draw[-latex] (0,0,0) -- (0,0,1.1) node[above left](xx){$\vv{x_0}$};

            \foreach \aa in {10,15,20,...,105}{
                    \pgfmathsetmacro{\bb}{\aa+5}
                    \fill[fill=green!30,opacity=0.3] (A\aa) -- (A\bb) -- (B\bb) -- (B\aa) -- cycle;
                }

            \foreach \aa in {30,32,34,...,48}{
                    \pgfmathsetmacro{\bb}{\aa+2}
                    \fill[fill=red!30,opacity=0.3] (C\aa) -- (C\bb) -- (D\bb) -- (D\aa) -- cycle;
                    \fill[fill=red!30,opacity=0.3] (E\aa) -- (E\bb) -- (F\bb) -- (F\aa) -- cycle;
                }
            \draw[fill=red!30,opacity=0.3] (C30) -- (E30) -- (F30) -- (D30) -- cycle;

            \draw[fill=red!30,opacity=0.3] (C50) -- (E50) -- (F50) -- (D50) -- cycle;

            \draw[-latex] (Z0) -- (Z4) node[left,pos=0.5]{$z$};
            \draw[latex-latex] (Z4) -- (Z6) node[left,pos=0.5]{$\mathrm{d}\,z$};
            \draw [dashed] (D50) --++(0,0.15)coordinate(aa);
            \draw [dashed] (F50) --++(0,0.15)coordinate(bb);
            \draw[latex-latex] (aa) -- (bb) node[above,pos=0.5,sloped]{$\mathrm{d}\,r$};

            \node[fit=(xx) (yy) (zz)](cadre){};
            \node[below=0.5em of cadre]{ $\mathrm{d}V=r\cdot \mathrm{d}\varphi\cdot \mathrm{d} r \cdot\mathrm{d}z$};
        \end{tikzpicture}
        \subcaption{$\dV$ in Polarkoordinaten}
    \end{subfigure}
    \hspace{1em}
    \begin{subfigure}[t]{0.45\textwidth}
        \begin{tikzpicture}[scale=0.9]
            %\clip[draw] (-1.7,-2) rectangle (3.5,3.5);
            \begin{scope}[scale=4.2,tdplot_main_coords]

                %-----------------------
                \coordinate (O) at (0,0,0);

                \tdplotsetcoord{P}{\rvec}{\thetavec}{\phivec}
                \tdplotsetcoord{P1}{\Rvec}{\thetavec}{\phivec}
                \tdplotsetcoord{P2}{\Rvec}{\Thetavec}{\phivec}
                \tdplotsetcoord{P3}{\rvec}{\Thetavec}{\phivec}

                \tdplotsetcoord{Q}{\rvec}{\thetavec}{\Phivec}
                \tdplotsetcoord{Q1}{\Rvec}{\thetavec}{\Phivec}
                \tdplotsetcoord{Q2}{\Rvec}{\Thetavec}{\Phivec}
                \tdplotsetcoord{Q3}{\rvec}{\Thetavec}{\Phivec}

                \draw[thick,fill=green!30] (P) -- (P1) -- (Q1) -- (Q)--cycle;
                \draw[thick,fill=green!30] (P3) -- (P2) -- (Q2) -- (Q3)--cycle;
                %draw figure contents
                %--------------------
                %draw the main coordinate system axes
                \draw[thick,->] (0,0,0) -- (0.9,0,0) node[above left](xx){$\vv{x_0}$};
                \draw[thick,->] (0,0,0) -- (0,0.9,0) node[above](yy){$\vv{y_0}$};
                \draw[thick,->] (0,0,0) -- (0,0,1) node[right](zz){$\vv{z_0}$};

                %draw a line from origin to point (P) 
                \draw[,color=red] (O) -- (P);
                \draw[,color=red] (O) -- (P2);
                \draw[color=red] (O) -- (P3);

                \draw[dashed, color=red] (O) -- (Pxy);
                \draw[dashed, color=red] (P) -- (Pxy);
                %
                \draw[dashed, color=red] (O) -- (P2xy);
                \draw[dashed, color=red] (P2) -- (P2xy);

                %draw a line from origin to point (Q) 
                \draw[,color=red] (O) -- (Q);
                \draw[,color=red] (O) -- (Q2);
                \draw[color=red] (O) -- (Q3);

                %\draw[,color=red] (P) -- (P1) --(P2) --(P3)--(P);
                %draw projection on xy plane, and a connecting line
                \draw[dashed, color=red] (O) -- (Qxy);
                \draw[dashed, color=red] (Q) -- (Qxy);
                %
                \draw[dashed, color=red] (O) -- (Q2xy);
                \draw[dashed, color=red] (Q2) -- (Q2xy);

                \pgfmathsetmacro{\Rproj}{\Rvec*sin(\Thetavec)}

                \draw[fill=gray!50] (Pxy) -- (Qxy) -- (Q2xy) -- (P2xy)--cycle;

                \tdplotdrawarc[-latex]{(O)}{0.25}{0}{\phivec}{anchor=north}{$\theta$}
                \tdplotdrawarc[latex-latex]{(O)}{0.65}{\phivec}{\Phivec}{anchor=north}{$\mathrm{d}\,\theta$}

                \tdplotsetthetaplanecoords{\phivec}

                \tdplotdrawarc[tdplot_rotated_coords,-latex]{(0,0,0)}{0.5}{0}{\thetavec}{anchor=south}{$\varphi$}

                \tdplotsetthetaplanecoords{\Phivec}

                \tdplotdrawarc[tdplot_rotated_coords,latex-latex]{(0,0,0)}{1}{\thetavec}{\Thetavec}{above,rotate=-55}{$\mathrm{d}\,\varphi$}

                \tdplotsetthetaplanecoords{\phivec}

                \draw[dashed,tdplot_rotated_coords] (\rvec,0,0) arc (0:90:\rvec);
                \draw[dashed,tdplot_rotated_coords] (\Rvec,0,0) arc (0:90:\Rvec);

                \draw[dashed] (\rvec,0,0) arc (0:90:\rvec);
                \draw[dashed] (\Rvec,0,0) arc (0:90:\Rvec);

                \tdplotsetthetaplanecoords{\Phivec}

                \draw[dashed,tdplot_rotated_coords] (\rvec,0,0) arc (0:90:\rvec);
                \draw[dashed,tdplot_rotated_coords] (\Rvec,0,0) arc (0:90:\Rvec);

                \begin{scope}[tdplot_main_coords]
                    \draw[blue,fill=red!30,opacity=0.3] plot[variable=\x,domain=\thetavec:\Thetavec]
                    (z spherical cs: radius = \rvec, phi = \Phivec, theta= \x)
                    -- plot[variable=\x,domain=\Phivec:\phivec]
                    (z spherical cs: radius = \rvec, phi = \x, theta= \Thetavec)
                    -- plot[variable=\x,domain=\Thetavec:\thetavec]
                    (z spherical cs: radius = \rvec, phi = \phivec, theta= \x)
                    -- plot[variable=\x,domain=\phivec:\Phivec]
                    (z spherical cs: radius = \rvec, phi = \x, theta= \thetavec);
                    %
                    \draw[blue,fill=red!30] plot[variable=\x,domain=\thetavec:\Thetavec]
                    (z spherical cs: radius = \Rvec, phi = \Phivec, theta= \x)
                    -- plot[variable=\x,domain=\Phivec:\phivec]
                    (z spherical cs: radius = \Rvec, phi = \x, theta= \Thetavec)
                    -- plot[variable=\x,domain=\Thetavec:\thetavec]
                    (z spherical cs: radius = \Rvec, phi = \phivec, theta= \x)
                    -- plot[variable=\x,domain=\phivec:\Phivec]
                    (z spherical cs: radius = \Rvec, phi = \x, theta= \thetavec);

                \end{scope}

                \tdplotsetthetaplanecoords{\phivec}

                %\tdplotdrawarc[tdplot_rotated_coords]{(0,0,0)}{0.5}{0}{\thetavec}{anchor=south west}{$\theta$}

                \draw[dashed,tdplot_rotated_coords] (\rvec,0,0) arc (0:90:\rvec);
                \draw[dashed,tdplot_rotated_coords] (\Rvec,0,0) arc (0:90:\Rvec);
                \draw[tdplot_rotated_coords,black,fill=blue!30,opacity=0.3] (P) -- (P1) arc (\thetavec:\Thetavec:\Rvec) -- (P3)  arc (\Thetavec:\thetavec:\rvec);

                \draw[dashed] (\rvec,0,0) arc (0:90:\rvec);
                \draw[dashed] (\Rvec,0,0) arc (0:90:\Rvec);

                \draw[latex-latex] (60:\rvec) -- (60:\Rvec) node[below right=-0.4em]{$\mathrm{d}\,r$};
                \draw[latex-latex] (75:0) --node[sloped,above,pos=0.8]{$r$} (75:\rvec) ;

                \tdplotsetthetaplanecoords{\Phivec}

                \draw[dashed,tdplot_rotated_coords] (\rvec,0,0) arc (0:90:\rvec);
                \draw[dashed,tdplot_rotated_coords] (\Rvec,0,0) arc (0:90:\Rvec);
                \draw[tdplot_rotated_coords,black,fill=blue!30,opacity=0.3] (Q) -- (Q1) arc (\thetavec:\Thetavec:\Rvec) -- (Q3)  arc (\Thetavec:\thetavec:\rvec);
            \end{scope}

            \node[fit=(xx) (yy) (zz)](cadre){};
            \node[below=1em of cadre]{$\mathrm{d}V=r\cdot  \sin \varphi \cdot\mathrm{d}\theta\cdot  r \cdot\mathrm{d}\phi \cdot\mathrm{d}r$.};
        \end{tikzpicture}
        \subcaption{$\dV$ in Kugelkoordinaten}
    \end{subfigure}

    \caption{Volumenelement}
    \label{fig:elementdevolume}
\end{figure}

\begin{thmb}{\np{\emph{[Jacobi-Determinante (3D)]}}}
    Das Volumenelement $\dV$ unter einer Koordinatentransformation $\vec{r}(u,v,w)$ ist gegeben durch
    \begin{align}
        \colorboxed{antiquefuchsia}{\dF=\abs{(\vr_u\times\vr_v)\cdot\vr_w}\,\du\,\dve\,\dw=\abs{\det{\mathbf{J}}}\du\,\dve\,\dw}
    \end{align}
    mit der Jacobi Matrix
    \begin{align}
        \colorboxed{antiquefuchsia}{\mathbf{J}=\begin{pmatrix}
                                                       x_u & x_v & x_w \\
                                                       y_u & y_v & y_w \\
                                                       z_u & z_v & z_w
                                                   \end{pmatrix}}
    \end{align}
    Für das Gebietsintegral gilt dann
    \begin{align}
        V=\iiint_B f(x,y,z)\,\dV = \iiint_{\widetilde{B}} \widetilde{f}(u,v,w)\abs{\det{\mathbf{J}}} \,\du\,\dve\,\dw
    \end{align}
    mit $\widetilde{f}(u,v)=f(x(u,v,w),y(u,v,w),z(u,v,w))$ und $\widetilde{B}=\vr\,(B)$.
\end{thmb}

\begin{thm}{\r{[Volumenelement in Kugelkoordinaten]}}
    Das Volumenelement in Kugelkoordinaten ist gegeben durch
    \begin{align}
        \dV=r^2\sin(\theta)\,\dr\,\dphi\,\dtheta
    \end{align}
\end{thm}

\end{document}

\begin{defn}{Funktionen in mehreren Variablen}
    Eine Funktion $f: D \to \R$ mit $D \subseteq \R^n$ ist eine Abbildung, die jedem $x \in D$ genau ein $f(x) \in \R$ zuordnet.
\end{defn}

\begin{tikzpicture}

    \begin{axis}[
            %axis background/.style={fill=green!10},
            %3d box=complete*,
            grid=major,
            %colorbar % show key
        ]
        \addplot3[surf] {{sin(deg(x)) * y*(1-y)}};
    \end{axis}

\end{tikzpicture}