\documentclass[12pt]{article}

\input{settings/settings_sparse.tex}
\newcommand{\blankpage}{\newpage\hbox{}\thispagestyle{empty}\newpage}
\newcommand{\emptyparagraph}{\paragraph{}\noindent}

\newcommand{\course}{\ifthenelse{\boolean{manuscript}}{manuscript}{course}\xspace}

% Comments
\newcommand{\ak}[1]{{\bf[AK: #1]}}

%--------Basic Math--------
\NewDocumentCommand{\floor}{m}{\left\lfloor #1 \right\rfloor}
\NewDocumentCommand{\ceil}{m}{\left\lceil #1 \right\rceil}
\NewDocumentCommand{\ip}{m}{\left\langle #1 \right\rangle}

%\newcommand*{\abs}[1]{\left| #1 \right|}
\newcommand*{\card}[1]{\left| #1 \right|}
%\NewDocumentCommand{\norm}{sm}{\IfBooleanTF{#1}{\|#2\|}{\left\| #2 \right\|}}

\newcommand*{\const}{\mathrm{const}}

\newcommand*{\defeq}{\overset{.}{=}}
\newcommand*{\eqdef}{\overset{.}{=}}

\DeclareMathOperator*{\argmax}{arg\,max}
\DeclareMathOperator*{\argmin}{arg\,min}


\DeclarePairedDelimiter\parentheses{(}{)}
\DeclarePairedDelimiter\brackets{[}{]}
\DeclarePairedDelimiter\braces{\{}{\}}


%--------Sets--------
\newcommand{\R}{\mathbb{R}}
\newcommand{\Rzero}{\mathbb{R}_{\geq 0}}
\newcommand{\Nat}{\mathbb{N}}
\newcommand{\NatZ}{\mathbb{N}_0}


%--------Symbols--------
%\renewcommand{\vec}[1]{\mathbold{#1}}
%\newcommand{\mat}[1]{\mathbold{#1}}
\newcommand{\rvec}[1]{\mathbf{#1}}
%\newcommand{\set}[1]{#1}
\newcommand{\spa}[1]{\mathcal{#1}}

\newcommand{\mean}[1]{\overline{#1}}
\newcommand{\compl}[1]{\overline{#1}}
\newcommand{\old}[1]{#1^{\mathrm{old}}}
\newcommand{\opt}[1]{#1^\star}

\newcommand{\altpi}{\Pi} % \vec{\uppi}



\NewDocumentCommand{\fnv}{oo}{v\IfValueT{#2}{_{#2}}\IfValueT{#1}{^{#1}}}
\RenewDocumentCommand{\v}{somo}{\IfBooleanTF{#1}{\fnv[\star][#4]\parentheses{#3}}{\fnv[#2][#4]\parentheses{#3}}}
\NewDocumentCommand{\fnq}{oo}{q\IfValueT{#2}{_{#2}}\IfValueT{#1}{^{#1}}}
\NewDocumentCommand{\q}{sommo}{\IfBooleanTF{#1}{\fnq[\star][#5]\parentheses{#3,#4}}{\fnq[#2][#5]\parentheses{#3,#4}}}
\NewDocumentCommand{\fnV}{oo}{V\IfValueT{#2}{_{#2}}\IfValueT{#1}{^{#1}}}
\NewDocumentCommand{\V}{somo}{\IfBooleanTF{#1}{\fnV[\star][#4]\parentheses{#3}}{\fnV[#2][#4]\parentheses{#3}}}
\NewDocumentCommand{\fnQ}{oo}{Q\IfValueT{#2}{_{#2}}\IfValueT{#1}{^{#1}}}
\NewDocumentCommand{\Q}{sommo}{\IfBooleanTF{#1}{\fnQ[\star][#5]\parentheses{#3,#4}}{\fnQ[#2][#5]\parentheses{#3,#4}}}
\NewDocumentCommand{\fna}{oo}{a\IfValueT{#2}{_{#2}}\IfValueT{#1}{^{#1}}}
\RenewDocumentCommand{\a}{sommo}{\IfBooleanTF{#1}{\fna[\star][#5]\parentheses{#3,#4}}{\fna[#2][#5]\parentheses{#3,#4}}}
\NewDocumentCommand{\fnA}{oo}{A\IfValueT{#2}{_{#2}}\IfValueT{#1}{^{#1}}}
\NewDocumentCommand{\fnAhat}{oo}{\hat{A}\IfValueT{#2}{_{#2}}\IfValueT{#1}{^{#1}}}
\NewDocumentCommand{\A}{sommo}{\IfBooleanTF{#1}{\fnA[\star][#5]\parentheses{#3,#4}}{\fnA[#2][#5]\parentheses{#3,#4}}}
\NewDocumentCommand{\fnj}{o}{J\IfValueT{#1}{_{#1}}}
\RenewDocumentCommand{\j}{mo}{\fnj[#2]\parentheses{#1}}
\NewDocumentCommand{\fnJ}{o}{\widehat{J}\IfValueT{#1}{_{#1}}}
\NewDocumentCommand{\J}{mo}{\fnJ[#2]\parentheses{#1}}

\NewDocumentCommand{\pset}{m}{\mathcal{P}\parentheses*{#1}}

\NewDocumentCommand{\pf}{mm}{{#1}_\sharp #2}

\NewDocumentCommand{\grad}{e_}{\boldsymbol{\nabla}\IfValueT{#1}{_{\!\!#1}\,}}
\NewDocumentCommand{\jac}{}{\mD}
\NewDocumentCommand{\hes}{}{\mH}
\NewDocumentCommand{\dive}{}{\grad\cdot}
\NewDocumentCommand{\lapl}{}{\Delta}

\NewDocumentCommand{\BigO}{m}{O\parentheses*{#1}}
\NewDocumentCommand{\BigOTil}{m}{\widetilde{O}\parentheses*{#1}}

\NewDocumentCommand{\transpose}{m}{#1^\top}
\NewDocumentCommand{\inv}{m}{#1^{-1}}
\RenewDocumentCommand{\det}{m}{\mathrm{det}\parentheses*{#1}}
\NewDocumentCommand{\tr}{m}{\mathrm{tr}\parentheses*{#1}}
\NewDocumentCommand{\diag}{om}{\mathrm{diag}\IfValueT{#1}{_{#1}}{}\braces{#2}}
\NewDocumentCommand{\msqrt}{m}{#1^{\nicefrac{1}{2}}}
\NewDocumentCommand{\vecop}{m}{\mathrm{vec}\brackets{#1}}

%--------Common vectors/matrices/sets--------
\newcommand{\vzero}{\vec{0}}
\newcommand{\vone}{\vec{1}}
\newcommand{\va}{\vec{a}}
\newcommand{\vap}{\vec{a'}}
\newcommand{\vas}{\vec{\opt{a}}}
\newcommand{\vb}{\vec{b}}
\newcommand{\vc}{\vec{c}}
\newcommand{\vd}{\vec{d}}
\newcommand{\ve}{\vec{e}}
\newcommand{\vf}{\vec{f}}
\newcommand{\vfhat}{\vec{\hat{f}}}
\newcommand{\vg}{\vec{g}}
\newcommand{\vh}{\vec{h}}
\newcommand{\vk}{\vec{k}}
\newcommand{\vm}{\vec{m}}
\newcommand{\vp}{\vec{p}}
\newcommand{\vq}{\vec{q}}
\newcommand{\vr}{\vec{r}}
\newcommand{\vs}{\vec{s}}
\newcommand{\vt}{\vec{t}}
\newcommand{\vu}{\vec{u}}
\newcommand{\vve}{\vec{v}}
\newcommand{\vvp}{\vec{v'}}
\newcommand{\vvs}{\vec{\opt{v}}}
\newcommand{\vw}{\vec{w}}
\newcommand{\vwhat}{\vec{\hat{w}}}
\newcommand{\vx}{\vec{x}}
\newcommand{\vxp}{\vec{x'}}
\newcommand{\vxs}{\vec{\opt{x}}}
\newcommand{\vy}{\vec{y}}
\newcommand{\vyp}{\vec{y'}}
\newcommand{\vz}{\vec{z}}
\newcommand{\valpha}{\vec{\alpha}}
\newcommand{\valphahat}{\vec{\hat{\alpha}}}
\newcommand{\vdelta}{\vec{\delta}}
\newcommand{\vDelta}{\vec{\Delta}}
\newcommand{\vepsilon}{\vec{\epsilon}}
\newcommand{\vvarepsilon}{\vec{\varepsilon}}
\newcommand{\veta}{\vec{\eta}}
\newcommand{\vlambda}{\vec{\lambda}}
\newcommand{\vmu}{\vec{\mu}}
\newcommand{\vmuhat}{\vec{\hat{\mu}}}
\newcommand{\vmup}{\vec{\mu'}}
\newcommand{\vnu}{\vec{\nu}}
\newcommand{\vomega}{\vec{\omega}}
\newcommand{\vphi}{\vec{\phi}}
\newcommand{\vpi}{\vec{\pi}}
\newcommand{\vpsi}{\vec{\psi}}
\newcommand{\vvarphi}{\vec{\varphi}}
\newcommand{\vvarphihat}{\vec{\hat{\varphi}}}
\newcommand{\vtheta}{\vec{\theta}}
\newcommand{\vthetahat}{\vec{\hat{\theta}}}
\newcommand{\vxi}{\vec{\xi}}
\newcommand{\mzero}{\mat{0}}
\newcommand{\mA}{\mat{A}}
\newcommand{\mB}{\mat{B}}
\newcommand{\mBs}{\mat{\opt{B}}}
\newcommand{\mC}{\mat{C}}
\newcommand{\mD}{\mat{D}}
\newcommand{\mF}{\mat{F}}
\newcommand{\mH}{\mat{H}}
\newcommand{\mI}{\mat{I}}
\newcommand{\mK}{\mat{K}}
\newcommand{\mL}{\mat{L}}
\newcommand{\mCalL}{\mat{\mathcal{L}}}
\newcommand{\mM}{\mat{M}}
\newcommand{\mP}{\mat{P}}
\newcommand{\mQ}{\mat{Q}}
\newcommand{\mS}{\mat{S}}
\newcommand{\mT}{\mat{T}}
\newcommand{\mU}{\mat{U}}
\newcommand{\mV}{\mat{V}}
\newcommand{\mW}{\mat{W}}
\newcommand{\mX}{\mat{X}}
\newcommand{\mLambda}{\mat{\Lambda}}
\newcommand{\mPhi}{\mat{\Phi}}
\newcommand{\mPi}{\mat{\Pi}}
\newcommand{\mSigma}{\mat{\Sigma}}
\newcommand{\mSigmap}{\mat{\Sigma'}}
\newcommand{\rG}{\rvec{G}}
\newcommand{\rQ}{\rvec{Q}}
\newcommand{\rU}{\rvec{U}}
\newcommand{\rV}{\rvec{V}}
\newcommand{\rW}{\rvec{W}}
\newcommand{\rX}{\rvec{X}}
\newcommand{\rXp}{\rvec{X'}}
\newcommand{\rY}{\rvec{Y}}
\newcommand{\rZ}{\rvec{Z}}
\newcommand{\sA}{\set{A}}
\newcommand{\sB}{\set{B}}
\newcommand{\sC}{\set{C}}
\newcommand{\sD}{\set{D}}
\newcommand{\sI}{\set{I}}
\newcommand{\sM}{\set{M}}
\newcommand{\sS}{\set{S}}
\newcommand{\sU}{\set{U}}
\newcommand{\sX}{\set{X}}
\newcommand{\sY}{\set{Y}}
\newcommand{\sZ}{\set{Z}}
\newcommand{\spA}{\spa{A}}
\newcommand{\spB}{\spa{B}}
\newcommand{\spC}{\spa{C}}
\newcommand{\spD}{\spa{D}}
\newcommand{\spF}{\spa{F}}
\newcommand{\spH}{\spa{H}}
\newcommand{\spL}{\spa{L}}
\newcommand{\spM}{\spa{M}}
\newcommand{\spO}{\spa{O}}
\newcommand{\spP}{\spa{P}}
\newcommand{\spQ}{\spa{Q}}
\newcommand{\spT}{\spa{T}}
\newcommand{\spW}{\spa{W}}
\newcommand{\spX}{\spa{X}}
\newcommand{\spY}{\spa{Y}}
\newcommand{\spZ}{\spa{Z}}
\newcommand{\fs}{\opt{f}}
\newcommand{\ps}{\opt{p}}
\newcommand{\qs}{\opt{q}}
\newcommand{\xs}{\opt{x}}
\newcommand{\ys}{\opt{y}}
\newcommand{\Bs}{\opt{B}}
\newcommand{\Qs}{\opt{Q}}
\newcommand{\sSs}{\opt{\sS}}
\newcommand{\hQs}{\opt{\hat{Q}}}
\newcommand{\Vs}{\opt{V}}
\newcommand{\pis}{\opt{\pi}}

\newcommand{\vF}{\rvec{F}}
\newcommand{\vS}{\rvec{S}}
\newcommand{\vT}{\rvec{T}}


\renewcommand\qedsymbol{$\blacksquare$}


\newcommand{\dx}{\mathrm{d}x}
\newcommand{\ddx}{\frac{\mathrm{d}}{\mathrm{d}x}}
\newcommand{\dt}{\mathrm{d}t}
\newcommand{\du}{\mathrm{d}u}
\newcommand{\dve}{\mathrm{d}v}
\newcommand{\dw}{\mathrm{d}w}
\newcommand{\dy}{\mathrm{d}y}
\newcommand{\dz}{\mathrm{d}z}
\newcommand{\dF}{\mathrm{d}F}
\newcommand{\dV}{\mathrm{d}V}
\newcommand{\dr}{\mathrm{d}r}
\newcommand{\dtheta}{\mathrm{d}\theta}
\newcommand{\drho}{\mathrm{d}\rho}
\newcommand{\dphi}{\mathrm{d}\varphi}

\newcommand{\Rn}{\mathbb{R}^n}
\newcommand{\Rm}{\mathbb{R}^m}
\newcommand{\Rk}{\mathbb{R}^k}
\newcommand{\und}{\text{ und }}
\newcommand{\oder}{\text{ oder }}
\newcommand{\bydef}{\underset{def.}{=}}
\newcommand{\BH}{\underset{\textrm{B-H}}{=}}

\newcommand{\Follows}{\Longrightarrow\ }
\newcommand{\sameas}{\Longleftrightarrow}
\newcommandx{\Laplace}[2][1=f(t), 2=s]{\mathscr{L}\{#1\}(#2)}
\newcommandx{\LaplaceInv}[2][1=F(s), 2=t]{\mathscr{L}^{-1}\{#1\}(#2)}
\DeclareMathOperator{\arccosh}{Arcosh}
\DeclareMathOperator{\arcsinh}{Arsinh}
\DeclareMathOperator{\arctanh}{Artanh}
\DeclareMathOperator{\arcsech}{arcsech}
\DeclareMathOperator{\arccsch}{arcCsch}
\DeclareMathOperator{\arccoth}{arcCoth} 

\def\doubleunderline#1{\underline{\underline{#1}}}
\def\ez{\begin{flushright}\underline{ez.}\end{flushright}}
%\[
%   \Laplace[\cos(x)]=\int_{t=0}^{\infty}f(t)e^{-st}dt
%\]

\newcommand{\Z}{\mathbb{Z}}
\newcommand{\N}{\mathbb{N}}

\newcommand{\C}{\mathbb{C}}

\newcommand{\inttext}{\shortintertext}

% The following example defines \colorboxed as wrapper around amsmath's \boxed to set the frame color. It uses package xcolor for the color support to save the current color . before changing the color for the frame. Inside the box, the previous saved color is restored. This avoids a white background of \fcolorbox, since there is no "transparent" color.

% The macro also supports an optional argument for specifying the color model.

% Definition of \boxed in amsmath.sty:
% \newcommand{\boxed}[1]{\fbox{\m@th$\displaystyle#1$}}


% Syntax: \colorboxed[<color model>]{<color specification>}{<math formula>}
\newcommand*{\colorboxed}{}
\def\colorboxed#1#{%
  \colorboxedAux{#1}%
}
\newcommand*{\colorboxedAux}[3]{%
  % #1: optional argument for color model
  % #2: color specification
  % #3: formula
  \begingroup
    \colorlet{cb@saved}{.}%
    \color#1{#2}%
    \boxed{%
      \color{cb@saved}%
      #3%
    }%
  \endgroup
}

%
\usepackage[utf8]{inputenc}
\usepackage[ngerman]{babel}
\usepackage[a4paper, top=1in, bottom=1.3in, rmargin=1.5in, left=1.4in, marginparwidth=80pt]{geometry}
\usepackage{etoc}
\usepackage{csquotes}
%\usepackage{emoji}
%\usepackage[tight]{minitoc}


\begin{document}
\title{\vspace*{-2.5em}Problem Set 4, Summary \& Tips}
\author{Vikram R. Damani\\
    Analysis~II}

\maketitle
%\dosectoc
%\sectoc

%\tableofcontents % //FIXME: make small toc

\section{Tipps}

\begin{nexercise}{Aufgabe 1}{1}\addcontentsline{toc}{subsection}{A1}
    Die Funktion $F:(0,\infty)\to\R$ sei durch
    \begin{align}
        F(\alpha):=\int_{0}^{\frac{1}{\alpha}} \frac{\arctan(\alpha t)}{t}\dt
    \end{align}
    gegeben. Zeigen Sie, dass die Ableitung $F'(\alpha)$
    identisch gleich Null ist, mittels
    \begin{enumerate}[label=\alph*.]
        \item Substitution.
        \item ($\heartsuit$) Ableitung unter dem Integral.
    \end{enumerate}
\end{nexercise}

\begin{tips}{1}
    Es gilt $F'(x)=0\implies F = \const.$
    \begin{enumerate}[label=\alph*.]
        \item Wir suchen eine geeignete Substitution $g(u)\defeq{f}(t)$ um zu zeigen, dass $F$ von $\alpha$ unabhängig  ist.
        \item Anwendung des Satz von Leibnitz.
    \end{enumerate}
\end{tips}\vspace*{1em}

\begin{nexercise}{Aufgabe 2 c.,d.,e.}{2}
    [Zeigen Sie, \dots]
    \begin{enumerate}[label=\alph*.]
        \setcounter{enumi}{2}
        \item dass an der Oberfläche des Zylinders die Strömung tangential verläuft und
        \item dass in grosser Entfernung vom Zylinder das Vektorfeld nahezu homogen ist.
        \item Bestimmen Sie ausserdem die Punkte maximaler und minimaler Geschwindigkeit auf der Zylinderoberfläche.
    \end{enumerate}
\end{nexercise}

\begin{tips}{2}
    \begin{enumerate}[label=\alph*.]
        \setcounter{enumi}{2}
        \item Ein Punkt an der Oberfläche des Zylinders $(x_0,y_0,z_0)$ ist Parametrisiert durch die Gleichung $x_0^2+y_0^2=a^2$ und der Tangentialvektor in der $x-y$ Ebene entlang der Oberfläche erfüllt $\vec{T}\,\bot\,(x_0,y_0,0)$.
        \item Man suche $\vve$ für Punkte im Abstand $R$ zu Null und $R\to\infty$.
        \item Aus c) lässt sich die Geschwindigkeit auf der Zylinderoberfläche berechnen.
    \end{enumerate}
\end{tips}\vspace*{1em}


\section{Theorie}
%\localtableofcontents

% \begin{displayquote}
%     \emph{Fuck it, there's not much to say}
% \end{displayquote}
% \begin{flushright}
%     Kottas, \today.
% \end{flushright}

% \begin{displayquote}
%     \emph{Yeah, but he's dead}
% \end{displayquote}
% \begin{flushright}
%     Kottas, \today.
% \end{flushright}

% \begin{displayquote}
%     \emph{Yeah, ants dont do taxes, do they\dots}
% \end{displayquote}
% \begin{flushright}
%     Kottas, \today.
% \end{flushright}

\begin{thmb}{\np{\emph{[Satz von \emph{Leibnitz} (Parameter in Integralgrenzen)].}}} Sei
    \begin{align}
        \Psi(x)\defeq\int_{u(x)}^{v(x)}f(x,t)\dt.
    \end{align}
    Dann ist die Ableitung
    \begin{align}
        \Psi'(x) & =\ddx \int_{u(x)}^{v(x)} f(t,x)\dt                                                                                       \\[5pt]
                 & =\colorboxed{antiquefuchsia}{\int_{u(x)}^{v(x)}\pdv{}{x}f(x,t)\dt + f(x,\r{v(x)})\,\r{v'(x)} - f(x,\r{u(x)})\,\r{u'(x)}}
    \end{align}

    \b{\emph{Spezialfälle}}:
    \begin{enumerate}[label=(\alph*)]
        \item $u(x)=a, \,v(x)=b\;\;\;\const.\implies u'=v'=0$ und es gilt also
              \begin{align}
                  \Psi'(x) =\colorboxed{antiquefuchsia}{\int_{a}^{b}\pdv{}{x}f(x,t)\dt}
              \end{align}
        \item $u(x)=a\;\const.\text{ und } \,v(x)=x,\text{ sowie }f(x,t)=g(t)$ unabh.\ von $x$ $\implies u'=0,\;g_x\equiv0,\;v'=1$ und es gilt also
              \begin{align}
                  \ddx\int_{0}^{x}g(t)\dt=g(x),
              \end{align}
              der Hauptsatz der Infinitesimalrechnung.
    \end{enumerate}
\end{thmb}\vspace*{1em}

\subsection{Vektorfelder}

\begin{defn}{\b{[Vektorfelder].}} Ein Vektorfeld weist jedem Punkt im Raum einen Vektor der gleichen Dimension zu.

    \textbf{Beispiele:}
    \begin{enumerate}[label=(\roman*)]
        \item Magnet/Elektrisches Feld
        \item Wind/Strömung
        \item Wärmefluss
              %\item Anything is a vectorfield if you're brave enough % including your mom
    \end{enumerate}
\end{defn}

\begin{figure}[htbp!]
    \centering
    \begin{tikzpicture}[>=latex, x=1.5cm, y=1.5cm, scale=1.5, font=\footnotesize]
        \def\length{sqrt(1+(x+y)^2)}
        \begin{axis}[
                axis lines=middle,
                view={0}{90},
                domain=-2.5:2.5,
                samples=18,
                axis equal image,
                ticklabel style={font=\tiny}
            ]
            \addplot3[
                blue!40!red,
                quiver={u={1/(\length)},
                        v={(x+y)/(\length)},
                        scale arrows=0.2,
                        every arrow/.append style={-latex}},
            ] (x,y,0);
        \end{axis}
    \end{tikzpicture}
\end{figure}

\begin{defn}{\b{[Skalarfeld]}} Sei $\mathbf{B}\subseteq\R^3$. Eine Funktion
    \begin{align}
        f\colon\quad \mathbf{B} & \longrightarrow\R    \\
        (x,y,z)                 & \longmapsto f(x,y,z)
    \end{align}
    in 3 Variablen ist ein Skalarfeld.
\end{defn}\vspace*{1em}

\begin{defn}{\b{[Vektorfeld]}} Ein Vektorfeld ist eine Funktion
    \begin{align}
        f\colon\quad \mathbf{B} & \longrightarrow\R^3                                \\
        \vec{r}                 & \longmapsto \vec{v}\,(\vec{r}\,)=\begin{pmatrix}
                                                                       v_1(x,y,z) \\[3pt]
                                                                       v_2(x,y,z) \\[3pt]
                                                                       v_3(x,y,z)
                                                                   \end{pmatrix}
    \end{align}
    \begin{rmk}{}{}
        Falls  $\mathbf{B}\subseteq\R^2$ und $\vec{v}: \mathbf{B}\to\R^2$, ist $\vec{v}$ ein \emph{ebenes Vektorfeld}.
    \end{rmk}
    \begin{rmk}{}{}
        Manche Vektorfelder sind ausserdem zeitabhänging, d.h.\ $\vve\,(\vr,t)\in\R^3$. Diese Vektorfelder nennt man instationär.
    \end{rmk}
\end{defn}\vspace*{1em}

\begin{defn}{\b{[Feldlinien]}}
    Eine Kurve $K\subset{B}$, die an jedem Punkt tangential zum Vektorfeld $\vve\,(\vr\,)$ ist, heisst Feldlinie von $\vve$.
\end{defn}\vspace*{1em}

\begin{defn}{\b{[Homogenes Vektorfeld]}} Ein konstantes Vektorfeld $\vve\,(\vr\,)=\va$ heisst homogen.
\end{defn}

\colorlet{Ecolor}{orange!90!black}
\colorlet{pluscolor}{red!60!black}
\colorlet{minuscolor}{blue!60!black}
\tikzstyle{anode}=[top color=red!20, bottom color=red!50]
\tikzstyle{cathode}=[top color=blue!20, bottom color=blue!40]
\tikzstyle{charge+}=[very thin,top color=red!50, bottom color=red!80]
\tikzstyle{charge-}=[very thin,top color=blue!40, bottom color=blue!70]
\tikzset{EFieldLine/.style={
            Ecolor, decoration={markings, mark=at position #1 with {\arrow{stealth}}}, postaction={decorate}}
}

\def\dph{0.3} % dipole height
\def\dpw{0.1} % dipole width
\def\dipole#1{
    \begin{scope}[shift={(#1)}]
        \draw[charge-] (-\dph,0) to[out=90,in=180] (0,\dpw) -- (0,-\dpw) to[out=180,in=-90] cycle;
        \draw[charge+] ( \dph,0) to[out=90,in=0] (0,\dpw) -- (0,-\dpw) to[out=  0,in=-90] cycle;
        \node[scale=0.7] at (-\dph/2,0) {$-$};
        \node[scale=0.7] at ( \dph/2,0) {$+$};
    \end{scope}
}

\def\height{5}
\def\width{3}
\def\platewidth{0.5}
\def\dielwidth{0.13*\width}
\def\nfieldlines{20}
\def\ncharges{7}

% capacitor with dipolar polarization
\begin{figure}[htbp!]
    \centering
    \begin{tikzpicture}

        % dielectric slab
        \draw[orange!60!black,fill=orange!80!brown!5]
        (\dielwidth,-0.03*\height) rectangle (\width-\dielwidth,1.03*\height)
        node[Ecolor, above=3cm, midway] {$\vec E$}
        node[above, pluscolor] {$+Q_\text{surf}$}
        node[above, minuscolor] at (1.3*\dielwidth,1.03*\height) {$-Q_\text{surf}$};

        % electric field
        \foreach \i [evaluate={\y=(\i-0.75)*\height/(\nfieldlines-0.5);}] in {1,...,\nfieldlines}{
                \draw[EFieldLine={0.54},very thick] (0,\y) --++ (\width,0);
            }

        % plates
        \draw[anode] (0,0) rectangle++ (-\platewidth,\height)
        node[left, pluscolor] {$+Q_\text{C}$};
        \draw[cathode] (\width,0) rectangle++ (\platewidth,\height)
        node[right, minuscolor] {$-Q_\text{C}$};
    \end{tikzpicture}
    \caption{Homogenes Vektorfeld eines Plattenkondensators.}
\end{figure}

\begin{defn}{\b{[Rotationsfeld]}} Bezeichnet $\vec{\omega}=\begin{pmatrix}
            \omega_1 & \omega_2 & \omega_3
        \end{pmatrix}\in\R^3$ ($\const.$) eine Winkelgeschwindigkeit um die Drehachse $\vec{\omega}$,
    so ist
    \begin{align}
        \vve\,(\vr\,)=\vec{\omega}\times\vr=\begin{pmatrix}
                                                \omega_2 z -\omega_3 y \\
                                                \omega_3 x-\omega_1 z  \\
                                                \omega_1 y-\omega_2 y
                                            \end{pmatrix}
    \end{align}
\end{defn}

\subsection{Differentialoperatoren}

\begin{defn}{\b{[Gradientenfeld]}}
    Sei $f$ ein Skalarfeld. Dann ist
    \begin{align}
        grad(f)=\grad{f(\vr\,)}=\begin{pmatrix}
                                    \pdv{f}{x} \\[4pt]
                                    \pdv{f}{y} \\[4pt]
                                    \pdv{f}{z}
                                \end{pmatrix}\in\R^3
    \end{align} das zugehörige Gradientenfeld.
\end{defn}

\begin{rmk}{}{} Zuweisungen die Funktionen in andere Funktionen umwandeln (Funktion~$\mapsto$~Funktion) nennt man Operatoren.
\end{rmk}\vspace*{1em}

\begin{defn}{\b{[Divergenz]}} Für ein Vektorfeld $\vve$ ist der Divergenzoperator wie folgt definiert
    \begin{align}
        \mathrm{div}(\vve)=\nabla\cdot{\vve}=(v_1)_x+(v_2)_y+(v_2)_z=f(x,y,z)
    \end{align} ein Skalarfeld.
\end{defn}

\begin{rmk}{}{}
    Der Laplace Operator kann auch als Divergenz des Gradientenfeldes geschrieben werden, also
    \begin{align}
        \Delta{f}=\nabla\cdot\grad{f}=(f_x)_x+(f_y)_y+(f_z)_z
    \end{align}
\end{rmk}

\begin{defn}{\b{[Rotation]}} Sei $\vve$ ein Vektorfeld. Dann ist
    \begin{align}
        \mathrm{rot}(\vve)=\vec\nabla\times\vve=\begin{pmatrix}
                                                    v_{3,y}-v_{2,z} \\
                                                    v_{1,z}-v_{3,x} \\
                                                    v_{2,x}-v_{1,y}
                                                \end{pmatrix}
    \end{align}
\end{defn}

Es gilt also
\begin{figure}[htbp!]
    \centering
    \includegraphics[scale=0.6]{imgs/skalarfeld-vektorfeld-relation.png}
\end{figure}

\end{document}

\begin{defn}{Funktionen in mehreren Variablen}
    Eine Funktion $f: D \to \R$ mit $D \subseteq \R^n$ ist eine Abbildung, die jedem $x \in D$ genau ein $f(x) \in \R$ zuordnet.
\end{defn}

\begin{tikzpicture}

    \begin{axis}[
            %axis background/.style={fill=green!10},
            %3d box=complete*,
            grid=major,
            %colorbar % show key
        ]
        \addplot3[surf] {{sin(deg(x)) * y*(1-y)}};
    \end{axis}

\end{tikzpicture}

