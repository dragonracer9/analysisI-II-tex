\newcommand{\blankpage}{\newpage\hbox{}\thispagestyle{empty}\newpage}
\newcommand{\emptyparagraph}{\paragraph{}\noindent}

\newcommand{\course}{\ifthenelse{\boolean{manuscript}}{manuscript}{course}\xspace}

% Comments
\newcommand{\ak}[1]{{\bf[AK: #1]}}

%--------Basic Math--------
\NewDocumentCommand{\floor}{m}{\left\lfloor #1 \right\rfloor}
\NewDocumentCommand{\ceil}{m}{\left\lceil #1 \right\rceil}
\NewDocumentCommand{\ip}{m}{\left\langle #1 \right\rangle}

%\newcommand*{\abs}[1]{\left| #1 \right|}
\newcommand*{\card}[1]{\left| #1 \right|}
%\NewDocumentCommand{\norm}{sm}{\IfBooleanTF{#1}{\|#2\|}{\left\| #2 \right\|}}

\newcommand*{\const}{\mathrm{const}}

\newcommand*{\defeq}{\overset{.}{=}}
\newcommand*{\eqdef}{\overset{.}{=}}

\DeclareMathOperator*{\argmax}{arg\,max}
\DeclareMathOperator*{\argmin}{arg\,min}


\DeclarePairedDelimiter\parentheses{(}{)}
\DeclarePairedDelimiter\brackets{[}{]}
\DeclarePairedDelimiter\braces{\{}{\}}


%--------Sets--------
\newcommand{\R}{\mathbb{R}}
\newcommand{\Rzero}{\mathbb{R}_{\geq 0}}
\newcommand{\Nat}{\mathbb{N}}
\newcommand{\NatZ}{\mathbb{N}_0}


%--------Symbols--------
%\renewcommand{\vec}[1]{\mathbold{#1}}
%\newcommand{\mat}[1]{\mathbold{#1}}
\newcommand{\rvec}[1]{\mathbf{#1}}
%\newcommand{\set}[1]{#1}
\newcommand{\spa}[1]{\mathcal{#1}}

\newcommand{\mean}[1]{\overline{#1}}
\newcommand{\compl}[1]{\overline{#1}}
\newcommand{\old}[1]{#1^{\mathrm{old}}}
\newcommand{\opt}[1]{#1^\star}

\newcommand{\altpi}{\Pi} % \vec{\uppi}



\NewDocumentCommand{\fnv}{oo}{v\IfValueT{#2}{_{#2}}\IfValueT{#1}{^{#1}}}
\RenewDocumentCommand{\v}{somo}{\IfBooleanTF{#1}{\fnv[\star][#4]\parentheses{#3}}{\fnv[#2][#4]\parentheses{#3}}}
\NewDocumentCommand{\fnq}{oo}{q\IfValueT{#2}{_{#2}}\IfValueT{#1}{^{#1}}}
\NewDocumentCommand{\q}{sommo}{\IfBooleanTF{#1}{\fnq[\star][#5]\parentheses{#3,#4}}{\fnq[#2][#5]\parentheses{#3,#4}}}
\NewDocumentCommand{\fnV}{oo}{V\IfValueT{#2}{_{#2}}\IfValueT{#1}{^{#1}}}
\NewDocumentCommand{\V}{somo}{\IfBooleanTF{#1}{\fnV[\star][#4]\parentheses{#3}}{\fnV[#2][#4]\parentheses{#3}}}
\NewDocumentCommand{\fnQ}{oo}{Q\IfValueT{#2}{_{#2}}\IfValueT{#1}{^{#1}}}
\NewDocumentCommand{\Q}{sommo}{\IfBooleanTF{#1}{\fnQ[\star][#5]\parentheses{#3,#4}}{\fnQ[#2][#5]\parentheses{#3,#4}}}
\NewDocumentCommand{\fna}{oo}{a\IfValueT{#2}{_{#2}}\IfValueT{#1}{^{#1}}}
\RenewDocumentCommand{\a}{sommo}{\IfBooleanTF{#1}{\fna[\star][#5]\parentheses{#3,#4}}{\fna[#2][#5]\parentheses{#3,#4}}}
\NewDocumentCommand{\fnA}{oo}{A\IfValueT{#2}{_{#2}}\IfValueT{#1}{^{#1}}}
\NewDocumentCommand{\fnAhat}{oo}{\hat{A}\IfValueT{#2}{_{#2}}\IfValueT{#1}{^{#1}}}
\NewDocumentCommand{\A}{sommo}{\IfBooleanTF{#1}{\fnA[\star][#5]\parentheses{#3,#4}}{\fnA[#2][#5]\parentheses{#3,#4}}}
\NewDocumentCommand{\fnj}{o}{J\IfValueT{#1}{_{#1}}}
\RenewDocumentCommand{\j}{mo}{\fnj[#2]\parentheses{#1}}
\NewDocumentCommand{\fnJ}{o}{\widehat{J}\IfValueT{#1}{_{#1}}}
\NewDocumentCommand{\J}{mo}{\fnJ[#2]\parentheses{#1}}

\NewDocumentCommand{\pset}{m}{\mathcal{P}\parentheses*{#1}}

\NewDocumentCommand{\pf}{mm}{{#1}_\sharp #2}

\NewDocumentCommand{\grad}{e_}{\boldsymbol{\nabla}\IfValueT{#1}{_{\!\!#1}\,}}
\NewDocumentCommand{\jac}{}{\mD}
\NewDocumentCommand{\hes}{}{\mH}
\NewDocumentCommand{\dive}{}{\grad\cdot}
\NewDocumentCommand{\lapl}{}{\Delta}

\NewDocumentCommand{\BigO}{m}{O\parentheses*{#1}}
\NewDocumentCommand{\BigOTil}{m}{\widetilde{O}\parentheses*{#1}}

\NewDocumentCommand{\transpose}{m}{#1^\top}
\NewDocumentCommand{\inv}{m}{#1^{-1}}
\RenewDocumentCommand{\det}{m}{\mathrm{det}\parentheses*{#1}}
\NewDocumentCommand{\tr}{m}{\mathrm{tr}\parentheses*{#1}}
\NewDocumentCommand{\diag}{om}{\mathrm{diag}\IfValueT{#1}{_{#1}}{}\braces{#2}}
\NewDocumentCommand{\msqrt}{m}{#1^{\nicefrac{1}{2}}}
\NewDocumentCommand{\vecop}{m}{\mathrm{vec}\brackets{#1}}

%--------Common vectors/matrices/sets--------
\newcommand{\vzero}{\vec{0}}
\newcommand{\vone}{\vec{1}}
\newcommand{\va}{\vec{a}}
\newcommand{\vap}{\vec{a'}}
\newcommand{\vas}{\vec{\opt{a}}}
\newcommand{\vb}{\vec{b}}
\newcommand{\vc}{\vec{c}}
\newcommand{\vd}{\vec{d}}
\newcommand{\ve}{\vec{e}}
\newcommand{\vf}{\vec{f}}
\newcommand{\vfhat}{\vec{\hat{f}}}
\newcommand{\vg}{\vec{g}}
\newcommand{\vh}{\vec{h}}
\newcommand{\vk}{\vec{k}}
\newcommand{\vm}{\vec{m}}
\newcommand{\vp}{\vec{p}}
\newcommand{\vq}{\vec{q}}
\newcommand{\vr}{\vec{r}}
\newcommand{\vs}{\vec{s}}
\newcommand{\vt}{\vec{t}}
\newcommand{\vu}{\vec{u}}
\newcommand{\vv}{\vec{v}}
\newcommand{\vvp}{\vec{v'}}
\newcommand{\vvs}{\vec{\opt{v}}}
\newcommand{\vw}{\vec{w}}
\newcommand{\vwhat}{\vec{\hat{w}}}
\newcommand{\vx}{\vec{x}}
\newcommand{\vxp}{\vec{x'}}
\newcommand{\vxs}{\vec{\opt{x}}}
\newcommand{\vy}{\vec{y}}
\newcommand{\vyp}{\vec{y'}}
\newcommand{\vz}{\vec{z}}
\newcommand{\valpha}{\vec{\alpha}}
\newcommand{\valphahat}{\vec{\hat{\alpha}}}
\newcommand{\vdelta}{\vec{\delta}}
\newcommand{\vDelta}{\vec{\Delta}}
\newcommand{\vepsilon}{\vec{\epsilon}}
\newcommand{\vvarepsilon}{\vec{\varepsilon}}
\newcommand{\veta}{\vec{\eta}}
\newcommand{\vlambda}{\vec{\lambda}}
\newcommand{\vmu}{\vec{\mu}}
\newcommand{\vmuhat}{\vec{\hat{\mu}}}
\newcommand{\vmup}{\vec{\mu'}}
\newcommand{\vnu}{\vec{\nu}}
\newcommand{\vomega}{\vec{\omega}}
\newcommand{\vphi}{\vec{\phi}}
\newcommand{\vpi}{\vec{\pi}}
\newcommand{\vpsi}{\vec{\psi}}
\newcommand{\vvarphi}{\vec{\varphi}}
\newcommand{\vvarphihat}{\vec{\hat{\varphi}}}
\newcommand{\vtheta}{\vec{\theta}}
\newcommand{\vthetahat}{\vec{\hat{\theta}}}
\newcommand{\vxi}{\vec{\xi}}
\newcommand{\mzero}{\mat{0}}
\newcommand{\mA}{\mat{A}}
\newcommand{\mB}{\mat{B}}
\newcommand{\mBs}{\mat{\opt{B}}}
\newcommand{\mC}{\mat{C}}
\newcommand{\mD}{\mat{D}}
\newcommand{\mF}{\mat{F}}
\newcommand{\mH}{\mat{H}}
\newcommand{\mI}{\mat{I}}
\newcommand{\mK}{\mat{K}}
\newcommand{\mL}{\mat{L}}
\newcommand{\mCalL}{\mat{\mathcal{L}}}
\newcommand{\mM}{\mat{M}}
\newcommand{\mP}{\mat{P}}
\newcommand{\mQ}{\mat{Q}}
\newcommand{\mS}{\mat{S}}
\newcommand{\mT}{\mat{T}}
\newcommand{\mU}{\mat{U}}
\newcommand{\mV}{\mat{V}}
\newcommand{\mW}{\mat{W}}
\newcommand{\mX}{\mat{X}}
\newcommand{\mLambda}{\mat{\Lambda}}
\newcommand{\mPhi}{\mat{\Phi}}
\newcommand{\mPi}{\mat{\Pi}}
\newcommand{\mSigma}{\mat{\Sigma}}
\newcommand{\mSigmap}{\mat{\Sigma'}}
\newcommand{\rG}{\rvec{G}}
\newcommand{\rQ}{\rvec{Q}}
\newcommand{\rU}{\rvec{U}}
\newcommand{\rV}{\rvec{V}}
\newcommand{\rW}{\rvec{W}}
\newcommand{\rX}{\rvec{X}}
\newcommand{\rXp}{\rvec{X'}}
\newcommand{\rY}{\rvec{Y}}
\newcommand{\rZ}{\rvec{Z}}
\newcommand{\sA}{\set{A}}
\newcommand{\sB}{\set{B}}
\newcommand{\sC}{\set{C}}
\newcommand{\sD}{\set{D}}
\newcommand{\sI}{\set{I}}
\newcommand{\sM}{\set{M}}
\newcommand{\sS}{\set{S}}
\newcommand{\sU}{\set{U}}
\newcommand{\sX}{\set{X}}
\newcommand{\sY}{\set{Y}}
\newcommand{\sZ}{\set{Z}}
\newcommand{\spA}{\spa{A}}
\newcommand{\spB}{\spa{B}}
\newcommand{\spC}{\spa{C}}
\newcommand{\spD}{\spa{D}}
\newcommand{\spF}{\spa{F}}
\newcommand{\spH}{\spa{H}}
\newcommand{\spL}{\spa{L}}
\newcommand{\spM}{\spa{M}}
\newcommand{\spO}{\spa{O}}
\newcommand{\spP}{\spa{P}}
\newcommand{\spQ}{\spa{Q}}
\newcommand{\spT}{\spa{T}}
\newcommand{\spW}{\spa{W}}
\newcommand{\spX}{\spa{X}}
\newcommand{\spY}{\spa{Y}}
\newcommand{\spZ}{\spa{Z}}
\newcommand{\fs}{\opt{f}}
\newcommand{\ps}{\opt{p}}
\newcommand{\qs}{\opt{q}}
\newcommand{\xs}{\opt{x}}
\newcommand{\ys}{\opt{y}}
\newcommand{\Bs}{\opt{B}}
\newcommand{\Qs}{\opt{Q}}
\newcommand{\sSs}{\opt{\sS}}
\newcommand{\hQs}{\opt{\hat{Q}}}
\newcommand{\Vs}{\opt{V}}
\newcommand{\pis}{\opt{\pi}}

\newcommand{\vF}{\rvec{F}}
\newcommand{\vS}{\rvec{S}}
\newcommand{\vT}{\rvec{T}}


\renewcommand\qedsymbol{$\blacksquare$}


\newcommand{\dx}{\mathrm{d}x}
\newcommand{\ddx}{\frac{\mathrm{d}}{\mathrm{d}x}}
\newcommand{\dt}{\mathrm{d}t}
\newcommand{\du}{\mathrm{d}u}
%\newcommand{\dv}{\mathrm{d}v}
\newcommand{\dy}{\mathrm{d}y}
\newcommand{\dz}{\mathrm{d}z}

\newcommand{\Rn}{\mathbb{R}^n}
\newcommand{\Rm}{\mathbb{R}^m}
\newcommand{\Rk}{\mathbb{R}^k}
\newcommand{\und}{\text{ und }}
\newcommand{\oder}{\text{ oder }}
\newcommand{\bydef}{\underset{def.}{=}}
\newcommand{\BH}{\underset{\textrm{B-H}}{=}}

\newcommand{\Follows}{\Longrightarrow\ }
\newcommand{\sameas}{\Longleftrightarrow}
\newcommandx{\Laplace}[2][1=f(t), 2=s]{\mathscr{L}\{#1\}(#2)}
\newcommandx{\LaplaceInv}[2][1=F(s), 2=t]{\mathscr{L}^{-1}\{#1\}(#2)}
\DeclareMathOperator{\arccosh}{Arcosh}
\DeclareMathOperator{\arcsinh}{Arsinh}
\DeclareMathOperator{\arctanh}{Artanh}
\DeclareMathOperator{\arcsech}{arcsech}
\DeclareMathOperator{\arccsch}{arcCsch}
\DeclareMathOperator{\arccoth}{arcCoth} 

\def\doubleunderline#1{\underline{\underline{#1}}}
\def\ez{\begin{flushright}\underline{ez.}\end{flushright}}
%\[
%   \Laplace[\cos(x)]=\int_{t=0}^{\infty}f(t)e^{-st}dt
%\]

\newcommand{\Z}{\mathbb{Z}}
\newcommand{\N}{\mathbb{N}}

\newcommand{\C}{\mathbb{C}}

\newcommand{\inttext}{\shortintertext}

% The following example defines \colorboxed as wrapper around amsmath's \boxed to set the frame color. It uses package xcolor for the color support to save the current color . before changing the color for the frame. Inside the box, the previous saved color is restored. This avoids a white background of \fcolorbox, since there is no "transparent" color.

% The macro also supports an optional argument for specifying the color model.

% Definition of \boxed in amsmath.sty:
% \newcommand{\boxed}[1]{\fbox{\m@th$\displaystyle#1$}}


% Syntax: \colorboxed[<color model>]{<color specification>}{<math formula>}
\newcommand*{\colorboxed}{}
\def\colorboxed#1#{%
  \colorboxedAux{#1}%
}
\newcommand*{\colorboxedAux}[3]{%
  % #1: optional argument for color model
  % #2: color specification
  % #3: formula
  \begingroup
    \colorlet{cb@saved}{.}%
    \color#1{#2}%
    \boxed{%
      \color{cb@saved}%
      #3%
    }%
  \endgroup
}
