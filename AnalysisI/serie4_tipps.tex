\documentclass[12pt]{article}

\usepackage[margin=1in]{geometry}
\usepackage[utf8]{inputenc}

\usepackage{amsmath}
\usepackage{amssymb}
\usepackage{amsfonts}
\usepackage{enumitem}
\usepackage{mathtools}
\usepackage{mathrsfs}
\usepackage{setspace}
\usepackage{xargs}
\usepackage{witharrows}

\newcommand{\Follows}{\Longrightarrow\ }
\newcommand{\sameas}{\Longleftrightarrow}
\newcommandx{\Laplace}[2][1=f(t), 2=s]{\mathscr{L}\{#1\}(#2)}

\def\doubleunderline#1{\underline{\underline{#1}}}
\def\ez{\begin{flushright}\underline{ez.}\end{flushright}}
%\[
%   \Laplace[\cos(x)]=\int_{t=0}^{\infty}f(t)e^{-st}dt
%\]

\newcommand{\R}{\mathbb{R}} % gives blackboard R
\newcommand{\Z}{\mathbb{Z}}
\newcommand{\N}{\mathbb{N}}
\newcommand{\Q}{\mathbb{Q}}
\newcommand{\C}{\mathbb{C}}

\newcommand{\inttext}{\shortintertext}

\newenvironment{definition}[2][Definition]{\begin{trivlist}
        \item[\hskip \labelsep {\bfseries #1}\hskip \labelsep {\bfseries #2.}]}{\end{trivlist}}
\newenvironment{lemma}[2][Lemma]{\begin{trivlist}
        \item[\hskip \labelsep {\bfseries #1}\hskip \labelsep {\bfseries #2.}]}{\end{trivlist}}
\newenvironment{exercise}[2][Exercise]{\begin{trivlist}
        \item[\hskip \labelsep {\bfseries #1}\hskip \labelsep {\bfseries #2.}]}{\end{trivlist}}
\newenvironment{problem}[2][Tipps \& Tricks zu]{\begin{trivlist}
        \item[\hskip \labelsep {\bfseries #1}\hskip \labelsep {\bfseries #2.}]}{\end{trivlist}}
\newenvironment{question}[2][Aufgabe]{\begin{trivlist}
        \item[\hskip \labelsep {\bfseries #1}\hskip \labelsep {\bfseries #2.}]}{\end{trivlist}}
\newenvironment{corollary}[2][Corollary]{\begin{trivlist}
        \item[\hskip \labelsep {\bfseries #1}\hskip \labelsep {\bfseries #2.}]}{\end{trivlist}}

\begin{document}
\title{Problem Set 4, Tips}
\author{Vikram Damani\\
    Analysis I}

\maketitle

\begin{question}{1}
    Berechnen Sie $f':\mathcal{D}(f)\to\R$ für
\end{question}

\begin{problem}{1}
Allgemeine Rechenregeln für Ableitungen:
\begin{itemize}
    \item Linearität:
          \begin{align}
              (f+g)'(x)       & =f'(x)+g'(x)                          \\
              (c\cdot{f})'(x) & =c\cdot{f}'(x), \qquad\forall{c}\in\R
          \end{align}
    \item Produktregel:
          \begin{align}
              (f\cdot{g})'(x) =f'(x)\cdot{g}(x)+g'(x)\cdot{f}(x)
          \end{align}
    \item Kettenregel:
          \begin{align}
              (f\circ{g})'(x) =f'(g(x))\cdot{g}'(x)
          \end{align}
    \item Quotientenregel:
          \begin{align}
              \left(\frac{f}{g}\right)'(x) =\frac{f'(x)\cdot{g}(x)-g'(x)\cdot{f}(x)}{g^2(x)}
          \end{align}
\end{itemize}
\end{problem}

\begin{question}{2}
    Ableitungen gerader/ungerader Funktionen
\end{question}

\begin{problem}{2}
Die Definitionen genügen um diese Aufgabe zu lösen.
\begin{definition}{[Gerade Funktion]}
    Eine Funktion heisst \textit{gerade}, falls $f(x)=f(-x)$.
\end{definition}
\begin{definition}{[Ungerade Funktion]}
    Eine Funktion heisst \textit{ungerade}, falls $-f(x)=f(-x)$.
\end{definition}
\end{problem}

\begin{question}{3}
    Es sei $P_0 = (x_0, y_0)$ ein von Ursprung verschiedener, aber sonst beliebiger Punkt der Parabel
    $y = x^2$ und $t_0$ sei die zugehörige Tangente.
    \begin{itemize}
        \item[(a)] Man finde den Punkt $P_1$ auf der Parabel, dessen zugehörige Tangente $t_1$ senkrecht zu $t_0$
              verläuft und bestimme den Schnittpunkt $S$ von $t_0$ und $t_1$ in Abhängigheit von $x_0$.
        \item[(b)] Man finde den Punkt $P_1$ auf der Parabel, dessen zugehörige Tangente $t_1$ senkrecht zu $t_0$
              verläuft und bestimme den Schnittpunkt $S$ von $t_0$ und $t_1$ in Abhängigheit von $x_0$.
    \end{itemize}
\end{question}

\begin{problem}{3}
Eine Tangente $t(x)$ an der Funktion $f(x)$ ist wie folgt definiert:
\begin{definition}{[Tangente $t(x)$]}
    Eine Tangnte $t(x)$ an $f(x)$ an der Stelle $x_0$ ist ene Gerade die den Graphen $\Gamma(f)$ mit Steigung $f'(x_0)$ an der Stelle $x_0$  berührt.
    \begin{align*}
        t: x \longmapsto y-f(x_0) & =f'(x_0)(x-x_0), \quad x\in\mathcal{D}(f)
    \end{align*}
    wobei $y-f(x_0)$ eine Verschiebung in $y$-Richtung und $x-x_0$ eine Verschiebung in $x$-Richtung ist.
\end{definition}

\begin{definition}{[Senkrechte Geraden]}
    Zwei Geraden $t_0$ und $t_1$ sind senkrecht zueinander, falls ihre Steigungen $m_0$ und $m_1$ folgende Bedingung erfüllen:
    \begin{align*}
        m_0\cdot{m_1}=-1
        \sameas f'(x_0)\cdot{f'(x_1)}=-1
        \sameas f'(x_0)=-\frac{1}{f'(x_1)}
    \end{align*}
\end{definition}

Der \textit{Schnittpunkt} $S$ zweier Geraden $t_0$ und $t_1$ ist gegeben durch die
Lösung des Gleichungssystems:
\begin{align*}
    t_0(x)                        & = t_1(x)                \\
    \sameas f(x_0)+f'(x_0)(x-x_0) & = f(x_1)+f'(x_1)(x-x_1)
\end{align*}
\end{problem}

\end{document}